% This file is available under NetHack General Public License. 
% The License is included in the file `license'. 

% Modified from nethack-343-src.tgz:/nethack-3.4.3/doc/Guidebook.tex 
% by Roy Clark (kralcyor) <kralcyor@kralcyor.info> in 2015. 

% Use LaTeX2e 
\documentclass[a4paper, 10pt]{article}

% To show Chinese 
\usepackage{ctex}
\setCJKmainfont[BoldFont={WenQuanYi Micro Hei}, ItalicFont={AR PL UKai CN}]{AR PL SungtiL GB}

\hyphenation{NET-H-A-C-KO-P-T-IONS}

\usepackage{hyperref}
\usepackage{fancyvrb}
\usepackage{supertabular}
\usepackage{cuted}
%\usepackage{listings}
\usepackage{enumitem}

% This file is available under NetHack General Public License. 
% The License is included in the file `license'. 
% This file is generated from zhTrans-template by generate_zhTrans_snippets.sh
% You may not modify this file directly. 
% Sun Nov 22 17:30:35 CST 2015

%%%%%%%%%%%%%%%%%%%%%%%%%%%%%%%%%%%%%%%%%%%%%%%%%%%%%%%%%%%%%%%%%%%%%%%%%%%%%%%%
%%% unique name 
% the Amulet of Yendor
\newcommand{\zhTransAmuletOfYendor}{壬铎护身符}
% the Valley of Gehennom
\newcommand{\zhTransValleyOfGehennom}{地狱之谷}
% the Mazes of Menace
\newcommand{\zhTransMazesOfMenace}{恐吓迷宫}
% Gnomish Mines
\newcommand{\zhTransGnomishMines}{侏儒矿坑}
% the Oracle of Delphi
\newcommand{\zhTransOracleOfDelpi}{德尔斐的祭司}
% the Dungeons of Doom
\newcommand{\zhTransDungeonsOfDoom}{毁灭地牢}

%%%%%%%%%%%%%%%%%%%%%%%%%%%%%%%%%%%%%%%%%%%%%%%%%%%%%%%%%%%%%%%%%%%%%%%%%%%%%%%%
%%% roles 
% Archeologist
\newcommand{\zhTransArcheologists}{考古学家}
% Barbarian
\newcommand{\zhTransBarbarians}{野蛮人}
% Caveman
\newcommand{\zhTransCavemen}{穴居人}
% Healer
\newcommand{\zhTransHealers}{医治师}
% Knight
\newcommand{\zhTransKnights}{骑士}
% Monk
\newcommand{\zhTransMonks}{僧侣}
% Priest
\newcommand{\zhTransPriests}{祭司}
% Ranger
\newcommand{\zhTransRangers}{漫游者}
% Rogue
\newcommand{\zhTransRogues}{流氓}
% Samurai
\newcommand{\zhTransSamurai}{武士}
% Tourist
\newcommand{\zhTransTourists}{游客}
% Valkyrie
\newcommand{\zhTransValkyries}{女武神}
% Wizard
\newcommand{\zhTransWizards}{巫师}

%%%%%%%%%%%%%%%%%%%%%%%%%%%%%%%%%%%%%%%%%%%%%%%%%%%%%%%%%%%%%%%%%%%%%%%%%%%%%%%%
%%% races 
% Dwarf
\newcommand{\zhTransDwarves}{矮人}
% Elf
\newcommand{\zhTransElves}{精灵}
% Gnome
\newcommand{\zhTransGnomes}{侏儒}
% Human
\newcommand{\zhTransHumans}{人类}
% Orc
\newcommand{\zhTransOrcs}{半兽人}

%%%%%%%%%%%%%%%%%%%%%%%%%%%%%%%%%%%%%%%%%%%%%%%%%%%%%%%%%%%%%%%%%%%%%%%%%%%%%%%%
%%% items in the status lines 
% Rank
\newcommand{\zhTransRank}{级别}
% Strength
\newcommand{\zhTransStrength}{力量值}
% Dexterity
\newcommand{\zhTransDexterity}{敏捷值}
% Constitution
\newcommand{\zhTransConstitution}{体格值}
% Intelligence
\newcommand{\zhTransIntelligence}{智力值}
% Wisdom
\newcommand{\zhTransWisdom}{智慧值}
% Charisma
\newcommand{\zhTransCharisma}{魅力值}
% Alignment
\newcommand{\zhTransAlignment}{阵营}
% Dungeon Level
\newcommand{\zhTransDungeonLevel}{地牢层数}
% Gold
\newcommand{\zhTransGold}{黄金}
% Hit Point
\newcommand{\zhTransHitPoints}{生命值}
% Power
\newcommand{\zhTransPower}{能量}
% Armor Class
\newcommand{\zhTransArmorClass}{护甲等级}
% Experience
\newcommand{\zhTransExperience}{经验}
% Time
\newcommand{\zhTransTime}{时间}
% Hunger Status
\newcommand{\zhTransHungerStatus}{饥饿状态}

%%%%%%%%%%%%%%%%%%%%%%%%%%%%%%%%%%%%%%%%%%%%%%%%%%%%%%%%%%%%%%%%%%%%%%%%%%%%%%%%
%%% alignment 
% Lawful
\newcommand{\zhTransLawful}{守序}
% Neutral
\newcommand{\zhTransNeutral}{中立}
% Chaotic
\newcommand{\zhTransChaotic}{混乱}

%%%%%%%%%%%%%%%%%%%%%%%%%%%%%%%%%%%%%%%%%%%%%%%%%%%%%%%%%%%%%%%%%%%%%%%%%%%%%%%%
%%% BUC 
% blessed
\newcommand{\zhTransBlessed}{被祝福的}
% uncursed
\newcommand{\zhTransUncursed}{未被诅咒的}
% cursed
\newcommand{\zhTransCursed}{被诅咒的}
% curse
\newcommand{\zhTransCurses}{诅咒}
% blessing
\newcommand{\zhTransBlessing}{祝福}

%%%%%%%%%%%%%%%%%%%%%%%%%%%%%%%%%%%%%%%%%%%%%%%%%%%%%%%%%%%%%%%%%%%%%%%%%%%%%%%%
%%% dungeon features 
% solid rock
\newcommand{\zhTransSolidRock}{坚硬的石头}
% wall
\newcommand{\zhTransWall}{墙}
% corner
\newcommand{\zhTransCorner}{墙角}
% doorway
\newcommand{\zhTransDoorway}{门口}
% door
\newcommand{\zhTransDoor}{门}
% iron bar
\newcommand{\zhTransIronBars}{铁条}
% tree
\newcommand{\zhTransTree}{树}
% floor
\newcommand{\zhTransFloor}{地面}
% corridor
\newcommand{\zhTransCorridor}{走道}
% stair
\newcommand{\zhTransStairs}{楼梯}
% ladder
\newcommand{\zhTransLadders}{梯子}
% altar
\newcommand{\zhTransAltar}{祭坛}
% grave
\newcommand{\zhTransGrave}{坟墓}
% throne
\newcommand{\zhTransThrone}{王座}
% sink
\newcommand{\zhTransSink}{水池}
% fountain
\newcommand{\zhTransFountain}{泉}
% pool
\newcommand{\zhTransPool}{水池}
% moat
\newcommand{\zhTransMoat}{护城河}
% ice
\newcommand{\zhTransIce}{冰}
% lava
\newcommand{\zhTransLava}{熔岩}
% drawbridge
\newcommand{\zhTransDrawbridge}{吊桥}
% air
\newcommand{\zhTransAir}{空气}
% cloud
\newcommand{\zhTransCloud}{云}
% under water
\newcommand{\zhTransUnderWater}{水面底下}
% trap
\newcommand{\zhTransTraps}{陷阱}

%%%%%%%%%%%%%%%%%%%%%%%%%%%%%%%%%%%%%%%%%%%%%%%%%%%%%%%%%%%%%%%%%%%%%%%%%%%%%%%%
%%% traps 
% arrow trap
\newcommand{\zhTransArrowTrap}{箭陷阱}
% dart trap
\newcommand{\zhTransDartTrap}{飞镖陷阱}
% falling rock trap
\newcommand{\zhTransFallingRockTrap}{落石陷阱}
% squeaky board
\newcommand{\zhTransSqueakyBoard}{发出吱吱声的地板}
% bear trap
\newcommand{\zhTransBearTrap}{捕熊器}
% land mine
\newcommand{\zhTransLandMine}{地雷}
% rolling boulder trap
\newcommand{\zhTransRollingBoulderTrap}{滚动巨石陷阱}
% sleeping gas trap
\newcommand{\zhTransSleepingGasTrap}{催眠陷阱}
% rust trap
\newcommand{\zhTransRustTrap}{锈蚀陷阱}
% fire trap
\newcommand{\zhTransFireTrap}{烈火陷阱}
% pit
\newcommand{\zhTransPit}{坑}
% spiked pit
\newcommand{\zhTransSpikedPit}{尖刺坑}
% hole
\newcommand{\zhTransHole}{洞}
% trap door
\newcommand{\zhTransTrapDoor}{陷阱门}
% teleportation trap
\newcommand{\zhTransTeleportationTrap}{瞬移陷阱}
% level teleporter
\newcommand{\zhTransLevelTeleporter}{层间瞬移陷阱}
% magic portal
\newcommand{\zhTransMagicPortal}{魔法入口}
% web
\newcommand{\zhTransWeb}{蜘蛛网}
% statue trap
\newcommand{\zhTransStatueTrap}{雕像陷阱}
% magic trap
\newcommand{\zhTransMagicTrap}{魔法陷阱}
% anti-magic field
\newcommand{\zhTransAntiMagicField}{反魔法力场}
% polymorph trap
\newcommand{\zhTransPolymorphTrap}{变形陷阱}

%%%%%%%%%%%%%%%%%%%%%%%%%%%%%%%%%%%%%%%%%%%%%%%%%%%%%%%%%%%%%%%%%%%%%%%%%%%%%%%%
%%% Encumbrance 
% Unburdened
\newcommand{\zhTransUnburdened}{不负重的}
% Burdened
\newcommand{\zhTransBurdened}{负重的}
% Stressed
\newcommand{\zhTransStressed}{受重压的}
% Strained
\newcommand{\zhTransStrained}{勉强的}
% Overtaxed
\newcommand{\zhTransOvertaxed}{负担过重的}
% Overloaded
\newcommand{\zhTransOverloaded}{过载的}

%%%%%%%%%%%%%%%%%%%%%%%%%%%%%%%%%%%%%%%%%%%%%%%%%%%%%%%%%%%%%%%%%%%%%%%%%%%%%%%%
%%% skills 
% none
\newcommand{\zhTransNone}{没有的}
% restricted
\newcommand{\zhTransRestricted}{受限制的}
% unskilled
\newcommand{\zhTransUnskilled}{不熟练的}
% basic
\newcommand{\zhTransBasic}{基本的}
% skilled
\newcommand{\zhTransSkilled}{熟练的}
% expert
\newcommand{\zhTransExpert}{内行的}
% master
\newcommand{\zhTransMaster}{大师}
% 这翻译的都什么鬼。
% grand master
\newcommand{\zhTransGrandMaster}{特级大师}

%%%%%%%%%%%%%%%%%%%%%%%%%%%%%%%%%%%%%%%%%%%%%%%%%%%%%%%%%%%%%%%%%%%%%%%%%%%%%%%%
%%% object classes 
% weapon
\newcommand{\zhTransWeapon}{武器}
% armor
\newcommand{\zhTransArmor}{护甲}
% food
\newcommand{\zhTransFood}{食物}
% scroll
\newcommand{\zhTransScroll}{卷轴}
% potion
\newcommand{\zhTransPotion}{药水}
% wand
\newcommand{\zhTransWand}{魔杖}
% ring
\newcommand{\zhTransRing}{戒指}
% spellbook
\newcommand{\zhTransSpellbook}{咒语书}
% tool
\newcommand{\zhTransTool}{工具}
% amulet
\newcommand{\zhTransAmulet}{护身符}
% gem
\newcommand{\zhTransGem}{宝石}
% boulder
\newcommand{\zhTransBoulders}{巨石}
% statue
\newcommand{\zhTransStatue}{雕像}

%%%%%%%%%%%%%%%%%%%%%%%%%%%%%%%%%%%%%%%%%%%%%%%%%%%%%%%%%%%%%%%%%%%%%%%%%%%%%%%%
%%% weapons 
% akly
\newcommand{\zhTransAklys}{阿克利斯标枪}
% arrow
\newcommand{\zhTransArrows}{箭}
% bec-de-corbin
\newcommand{\zhTransBecDeCorbin}{渡鸦嘴战锤}
% bow
\newcommand{\zhTransBow}{弓}
% broadsword
\newcommand{\zhTransBroadsword}{大刀}
% crossbow
\newcommand{\zhTransCrossbow}{驽}
% crossbow bolt
\newcommand{\zhTransCrossbowBolt}{驽箭}
% dagger
\newcommand{\zhTransDagger}{匕首}
% lucern hammer
\newcommand{\zhTransLucernHammer}{琉森锤}
% mace
\newcommand{\zhTransMaces}{狼牙棒}
% polearm
\newcommand{\zhTransPolearm}{长兵器}
% spear
\newcommand{\zhTransSpears}{矛}
% staff
\newcommand{\zhTransStaff}{棍棒}
% sword
\newcommand{\zhTransSwords}{剑}
% sling
\newcommand{\zhTransSling}{投石器}
% boomerang
\newcommand{\zhTransBoomerang}{回飞镖}
% two-handed sword
\newcommand{\zhTransTwoHandedSword}{双手剑}

%%%%%%%%%%%%%%%%%%%%%%%%%%%%%%%%%%%%%%%%%%%%%%%%%%%%%%%%%%%%%%%%%%%%%%%%%%%%%%%%
%%% armors 
% dragon scale mail
\newcommand{\zhTransDragonScaleMail}{龙鳞甲}
% plate mail
\newcommand{\zhTransPlateMail}{板链甲}
% crystal plate mail
\newcommand{\zhTransCrystalPlateMail}{水晶板链甲}
% bronze plate mail
\newcommand{\zhTransBronzePlateMail}{黄铜板链甲}
% splint mail
\newcommand{\zhTransSplintMail}{条板甲}
% banded mail
\newcommand{\zhTransBandedMail}{带链甲}
% dwarvish mithril-coat
\newcommand{\zhTransDwarvishMithrilCoat}{矮人秘银衣}
% elven mithril-coat
\newcommand{\zhTransElvenMithrilCoat}{精灵秘银衣}
% chain mail
\newcommand{\zhTransChainMail}{锁子甲}
% orcish chain mail
\newcommand{\zhTransOrcishChainMail}{半兽人锁子甲}
% scale mail
\newcommand{\zhTransScaleMail}{鳞甲}
% studded leather armor
\newcommand{\zhTransStuddedLeatherArmor}{带钉皮甲}
% ring mail
\newcommand{\zhTransRingMail}{环甲}
% orcish ring mail
\newcommand{\zhTransOrcishRingMail}{半兽人环甲}
% leather armor
\newcommand{\zhTransLeatherArmor}{皮甲}
% leather jacket
\newcommand{\zhTransLeatherJacket}{皮夹克}
% shield
\newcommand{\zhTransShield}{盾}

%%%%%%%%%%%%%%%%%%%%%%%%%%%%%%%%%%%%%%%%%%%%%%%%%%%%%%%%%%%%%%%%%%%%%%%%%%%%%%%%
%%% food 
% lembas wafer
\newcommand{\zhTransLembasWafers}{兰巴斯饼干}
% cram ration
\newcommand{\zhTransCramRation}{克元配给}
% food ration
\newcommand{\zhTransFoodRation}{食物配给}
% K-ration
\newcommand{\zhTransKRation}{K 配给}
% C-ration
\newcommand{\zhTransCRation}{C 配给}
% tin
\newcommand{\zhTransTin}{罐头}
% fortune cookie
\newcommand{\zhTransFortuneCookie}{幸运饼干}
% pancake
\newcommand{\zhTransPancake}{薄煎饼}
% cream pie
\newcommand{\zhTransCreamPie}{奶油派}
% candy bar
\newcommand{\zhTransCandyBar}{方糖块}
% lump of royal jelly
\newcommand{\zhTransLumpOfRoyalJelly}{蜂王浆团块}
% tripe ration
\newcommand{\zhTransTripeRation}{肚儿配给}
% slime mold
\newcommand{\zhTransSlimeMold}{黏菌}

%%%%%%%%%%%%%%%%%%%%%%%%%%%%%%%%%%%%%%%%%%%%%%%%%%%%%%%%%%%%%%%%%%%%%%%%%%%%%%%%
%%% other objects 
% ice box
\newcommand{\zhTransIceBox}{冰箱}
% scroll of identify
\newcommand{\zhTransScrollOfIdentify}{鉴定卷轴}
% scroll of mail
\newcommand{\zhTransScrollOfMail}{邮件卷轴}
% holy water
\newcommand{\zhTransHolyWater}{圣水}
% unholy water
\newcommand{\zhTransUnholyWater}{邪恶之水}
% pick-axe
\newcommand{\zhTransPickAxe}{镐}
% potion of polymorph
\newcommand{\zhTransPotionOfPolymorph}{变形药水}

%%%%%%%%%%%%%%%%%%%%%%%%%%%%%%%%%%%%%%%%%%%%%%%%%%%%%%%%%%%%%%%%%%%%%%%%%%%%%%%%
%%% monsters 
% FIXME check 
% angelic being
\newcommand{\zhTransAngelicBeing}{类天使生物}
% ant
\newcommand{\zhTransAnt}{蚂蚁}
% apelike creature
\newcommand{\zhTransApelikeCreature}{类猿生物}
% arachnid
\newcommand{\zhTransArachnid}{蛛形纲动物}
% bat
\newcommand{\zhTransBat}{蝙蝠}
% bird
\newcommand{\zhTransBird}{鸟类}
% black pudding
\newcommand{\zhTransBlackPudding}{黑布丁}
% blob
\newcommand{\zhTransBlobs}{液滴}
% centaur
\newcommand{\zhTransCentaur}{半人马}
% centipede
\newcommand{\zhTransCentipede}{蜈蚣}
% https://zh.wikipedia.org/wiki/%E9%9B%9E%E8%9B%87 
% cockatrice
\newcommand{\zhTransCockatrice}{鸡蛇}
% demons
\newcommand{\zhTransDemons}{恶魔}
% dog
\newcommand{\zhTransDog}{狗}
% dragon
\newcommand{\zhTransDragon}{龙}
% elemental
\newcommand{\zhTransElemental}{元素}
% eye
\newcommand{\zhTransEye}{眼}
% feline
\newcommand{\zhTransFeline}{猫科动物}
% fungi
\newcommand{\zhTransFungi}{真菌}
% ghost
\newcommand{\zhTransGhosts}{鬼魂}
% giant
\newcommand{\zhTransGiant}{巨人}
% giant humanoid
\newcommand{\zhTransGiantHumanoid}{巨型类人生物}
% https://zh.wikipedia.org/wiki/%E9%AD%94%E5%83%8F 
% golem
\newcommand{\zhTransGolem}{魔像}
% gremlin
\newcommand{\zhTransGremlin}{小妖精}
% horse
\newcommand{\zhTransHorse}{马}
% humanoid
\newcommand{\zhTransHumanoid}{类人生物}
% https://zh.wikipedia.org/wiki/%E5%B0%8F%E6%81%B6%E9%AD%94 
% imp
\newcommand{\zhTransImp}{小恶魔}
% jabberwock
\newcommand{\zhTransJabberwock}{甲伯沃基}
% jellies
\newcommand{\zhTransJellies}{胶冻}
% Keystone Kop
\newcommand{\zhTransKeystoneKop}{启斯东警察}
% https://zh.wikipedia.org/wiki/%E7%8B%97%E5%A4%B4%E4%BA%BA 
% kobold
\newcommand{\zhTransKobold}{狗头人}
% leprechaun
\newcommand{\zhTransLeprechaun}{矮妖精}
% https://zh.wikipedia.org/wiki/%E5%B7%AB%E5%A6%96 
% lich
\newcommand{\zhTransLich}{巫妖}
% light
\newcommand{\zhTransLight}{光}
% little dog
\newcommand{\zhTransLittleDog}{小狗}
% lizard
\newcommand{\zhTransLizard}{蜥蜴}
% long worm
\newcommand{\zhTransLongWorm}{长蠕虫}
% lurker above
\newcommand{\zhTransLurkerAbove}{顶部潜伏者}
% mail daemon
\newcommand{\zhTransMailDaemon}{邮件精灵}
% mimic
\newcommand{\zhTransMimic}{拟物怪}
% mind flayer
\newcommand{\zhTransMindFlayer}{夺心魔}
% minor demon
\newcommand{\zhTransMinorDemon}{低级恶魔}
% mold
\newcommand{\zhTransMold}{霉菌}
% mummy
\newcommand{\zhTransMummy}{木乃伊}
% naga
\newcommand{\zhTransNaga}{那伽}
% nymph
\newcommand{\zhTransNymph}{仙女}
% ooze
\newcommand{\zhTransOoze}{泥浆}
% ogre
\newcommand{\zhTransOrge}{食人魔}
% piercer
\newcommand{\zhTransPiercer}{锥状怪}
% pony
\newcommand{\zhTransPony}{矮种马}
% pudding
\newcommand{\zhTransPudding}{布丁}
% quadruped
\newcommand{\zhTransQuadruped}{四足动物}
% quantum mechanic
\newcommand{\zhTransQuantumMechanic}{量子力学}
% rodent
\newcommand{\zhTransRodent}{啮齿动物}
% rust monster
\newcommand{\zhTransRustMonster}{锈蚀怪}
% sea monster
\newcommand{\zhTransSeaMonster}{海怪}
% snake
\newcommand{\zhTransSnake}{蛇}
% sphere
\newcommand{\zhTransSphere}{球}
% trapper
\newcommand{\zhTransTrapper}{陷阱怪}
% troll
\newcommand{\zhTransTroll}{巨怪}
% https://zh.wikipedia.org/wiki/%E5%9C%9F%E5%B7%A8%E6%80%AA 
% umber hulk
\newcommand{\zhTransUmberHulk}{土巨怪}
% undead
\newcommand{\zhTransUndead}{不死生物}
% unicorn
\newcommand{\zhTransUnicorn}{独角兽}
% vampire
\newcommand{\zhTransVampire}{吸血鬼}
% vortex
\newcommand{\zhTransVortex}{旋涡}
% worm
\newcommand{\zhTransWorm}{蠕虫}
% wraith
\newcommand{\zhTransWraith}{死灵}
% xan
\newcommand{\zhTransXan}{三虫}
% xorn
\newcommand{\zhTransXorn}{索尔石怪}
% zombie
\newcommand{\zhTransZombie}{僵尸}
% zruty
\newcommand{\zhTransZruty}{祖鲁提}

%%%%%%%%%%%%%%%%%%%%%%%%%%%%%%%%%%%%%%%%%%%%%%%%%%%%%%%%%%%%%%%%%%%%%%%%%%%%%%%%
%%% miscellaneous
% artifact
\newcommand{\zhTransArtifact}{神器}
% attribute
\newcommand{\zhTransAttributes}{属性}
% beam
\newcommand{\zhTransBeam}{光线}
% beast
\newcommand{\zhTransBeast}{动物}
% blind
\newcommand{\zhTransBlind}{失明}
% Bones
\newcommand{\zhTransBones}{尸骨}
% bonuse
\newcommand{\zhTransBonuses}{加成}
% cancel
\newcommand{\zhTransCancel}{作废}
% charge
\newcommand{\zhTransCharge}{能量}
% computer fantasy game
\newcommand{\zhTransComputerFantasyGames}{电脑幻想游戏}
% Confused
\newcommand{\zhTransConfused}{迷糊}
% conduct
\newcommand{\zhTransConduct}{品行}
% credit
\newcommand{\zhTransCredit}{信用量}
% credit card
\newcommand{\zhTransCreditCard}{信用卡}
% dai-sho
\newcommand{\zhTransDaisho}{大小}
% Divine intervention
\newcommand{\zhTransDivineIntervention}{神的干预}
% dungeon
\newcommand{\zhTransDungeon}{地牢}
% dungeoneer
\newcommand{\zhTransDungeoneer}{地牢探索者}
% enchanted
\newcommand{\zhTransEnchanted}{被施过法术的}
% enchantment
\newcommand{\zhTransEnchantments}{魔法值}
% endgame
\newcommand{\zhTransEndgame}{最终试炼}
% experience level
\newcommand{\zhTransExperienceLevel}{经验等级}
% experience point
\newcommand{\zhTransExperiencePoints}{经验值}
% Fainting
\newcommand{\zhTransFainting}{昏晕}
% genocide
\newcommand{\zhTransGenocide}{灭绝}
% inventory
\newcommand{\zhTransInventory}{清单}
% invisible
\newcommand{\zhTransInvisible}{隐身的}
% Luck
\newcommand{\zhTransLuck}{幸运值}
% magic resistance
\newcommand{\zhTransMagicResistance}{魔法免疫}
% mana
\newcommand{\zhTransMana}{玛那}
% martial arts
\newcommand{\zhTransMartialArts}{战争艺术}
% MAUD
\newcommand{\zhTransMAUD}{毛德}
% Northland
\newcommand{\zhTransNorthlands}{北国}
% pets
\newcommand{\zhTransPets}{宠物}
% Polymorph
\newcommand{\zhTransPolymorph}{变形}
% quest
\newcommand{\zhTransQuest}{任务}
% Nippon
\newcommand{\zhTransNippon}{日本}
% Satiated
\newcommand{\zhTransSatiated}{饱足}
% shop
\newcommand{\zhTransShop}{黑店}
% shopkeeper
\newcommand{\zhTransShopkeeper}{黑店老板}
% shortest path
\newcommand{\zhTransShortestPath}{最短路径}
% shortest-path algorithm
\newcommand{\zhTransShortestPathAlg}{最短路径算法}
% Sokoban
\newcommand{\zhTransSokoban}{推箱子}
% steeds
\newcommand{\zhTransSteeds}{坐骑}
% Stunned
\newcommand{\zhTransStunned}{失去平衡}
% text adventure games
\newcommand{\zhTransTextAdventureGames}{文字冒险游戏}
% {\it Unearthed Arcana}
\newcommand{\zhTransUnearthedArcana}{《被发掘出的奥秘》}


\hypersetup{
pdftitle={进入\zhTransMazesOfMenace{}的指导:NetHack 指南},
pdfauthor={Eric S. Raymond 著/Roy Clark 译},
pdfsubject={Game Manual},
pdfkeywords={NetHack Guidebook, Chinese translation, the Mazes of Menace},
}

% 可能是经过多人陆续增加修改的缘故,原文档中格式有些不统一。
% 所以翻译时,试着进行了统一。
% {\it NetHack}
% {\it 人名},如“{\it Izchak Miller}”
% 怪物{\tt mon},如“{\tt h}”(dwarf, dwarf lord, dwarf king)
% 单个按键的命令{\tt c},扩展命令{\tt #command},如“{\tt \_}”和“{\tt #loot}”
% 按键{\tt k},如“{\tt ESC}”
% 以“OPTIONS=foo”形式设置的选项{\it foo},如“{\it fruit}”
% 以“BAR=”形式设置的选项{\tt BAR},如“{\tt SOUND}”
% 编译时选项 OPTION,如“AUTOPICKUP\_EXCEPTIONS”
% 平台 port,如“OS/2”
% {\tt 原封不动地引用的文字},可能为屏幕上出现的消息、选项的例子、命令的例子等,如“{\tt Elbereth}”

%\documentstyle[titlepage]{article}

\textheight 220mm
\textwidth 160mm
\oddsidemargin 0mm
\evensidemargin 0mm
\topmargin 0mm

\newcommand{\nd}{\noindent}

\newcommand{\tb}[1]{\tt #1 \hfill}
\newcommand{\bb}[1]{\bf #1 \hfill}
\newcommand{\ib}[1]{\it #1 \hfill}

\newcommand{\blist}[1]
{\begin{list}{$\bullet$}
    {\leftmargin 30mm \topsep 2mm \partopsep 0mm \parsep 0mm \itemsep 1mm
     \labelwidth 28mm \labelsep 2mm
     #1}}

\newcommand{\elist}{\end{list}}

% this will make \tt underscores look better, but requires that
% math subscripts will never be used in this document
\catcode`\_=12

\begin{document}
%
% input file: guidebook.mn
% $Revision: 1.61.2.19 $ $Date: 2003/12/03 03:00:50 $
%
%.ds h0 "
%.ds h1 %.ds h2 \%
%.ds f0 "

%.mt
%\title{\LARGE A Guide to the Mazes of Menace:\\
%\Large Guidebook for {\it NetHack\/}}
\title{\LARGE 进入\zhTransMazesOfMenace{}的指导:\\
\Large {\it NetHack\/} 指南}

%.au
%\author{Eric S. Raymond\\
%(Extensively edited and expanded for 3.4)}
\author{Eric S. Raymond 著/Roy Clark 译\\
(针对 3.4 版本作了大量编辑和扩展)}
%\date{December 2, 2003}
\date{2003 年 12 月 2 日}

\maketitle

%.hn 1
%\section{Introduction}
\section{介绍}

%.pg

%Recently, you have begun to find yourself unfulfilled and distant 
%in your daily occupation.  Strange dreams of prospecting, stealing, 
%crusading, and combat have haunted you in your sleep for many months, 
%but you aren't sure of the reason.  You wonder whether you have in 
%fact been having those dreams all your life, and somehow managed to 
%forget about them until now.  Some nights you awaken suddenly
%and cry out, terrified at the vivid recollection of the strange and 
%powerful creatures that seem to be lurking behind every corner of the 
%dungeon in your dream.  Could these details haunting your dreams be real?  
%As each night passes, you feel the desire to enter the mysterious caverns 
%near the ruins grow stronger.  Each morning, however, you quickly put 
%the idea out of your head as you recall the tales of those who entered 
%the caverns before you and did not return.  Eventually you can resist 
%the yearning to seek out the fantastic place in your dreams no longer.  
%After all, when other adventurers came back this way after spending time 
%in the caverns, they usually seemed better off than when they passed 
%through the first time.  And who was to say that all of those who did 
%not return had not just kept going?
最近,你开始发现自己在日常生活中感到不满意和缺乏热情。
关于探索、潜入、远征和战斗的奇怪的梦,已经在你睡着时在你的脑中萦回了数月,但你无法确定原因。
你怀疑是否事实上你一直有着这些梦,只是不知怎么地设法忘记了它们,直到现在才又出现。
一些晚上你突然醒来并大喊,清晰地记得奇怪和强大的生物似乎潜伏在梦中\zhTransDungeon{}的各一个角落,
你对此感到害怕。这些盘踞在你梦中的情形可能是真的吗?
随着每一个夜晚过去,你感到想要进入废墟附近的洞穴的欲望越来强烈。
然而每天早晨,你回想起关于那些在你之前进入洞穴并且还没有回来的人的传闻,
于是快速将这些想法赶出你的头脑。
最终你再也无法抗拒对探索那个你梦中奇幻地方的渴望。毕竟,
当其他冒险者在花费时间在洞穴之后回来时,他们通常看起来比他们刚进入洞穴时要好。
而且谁能说所有那些没有回来的人不是正好还在继续前进?
%.pg

%Asking around, you hear about a bauble, called the Amulet of Yendor by some,
%which, if you can find it, will bring you great wealth.  One legend you were
%told even mentioned that the one who finds the amulet will be granted
%immortality by the gods.  The amulet is rumored to be somewhere beyond the
%Valley of Gehennom, deep within the Mazes of Menace.  Upon hearing the
%legends, you immediately realize that there is some profound and 
%undiscovered reason that you are to descend into the caverns and seek 
%out that amulet of which they spoke.  Even if the rumors of the amulet's 
%powers are untrue, you decide that you should at least be able to sell the 
%tales of your adventures to the local minstrels for a tidy sum, especially 
%if you encounter any of the terrifying and magical creatures of 
%your dreams along the way.  You spend one last night fortifying yourself 
%at the local inn, becoming more and more depressed as you watch the odds 
%of your success being posted on the inn's walls getting lower and lower.  
询问了附近的人们,你听说了一个无价的东西,它被一些人称为\zhTransAmuletOfYendor
(the Amulet of Yendor),
只要能得到它,它将会给你带来大量的财富。
一个你听到的传说甚至提到,找到这个护身符的人会被众神赋予永生。
该护身符据传闻在超过\zhTransValleyOfGehennom(the Valley of Gehennom)的某处,
深深地位于\zhTransMazesOfMenace(the Mazes of Menace)之中。
听到这些传说,你立刻意识到有一些极大和未知的原因,
使得你要下降进入洞穴,寻找他们所说的护身符。
即使关于护身符力量的传言是不真实的,你判断出你至少应该能够将你的冒险故事卖给当地的吟游诗人,
从而获得大量钱财,特别是如果你在途中遇到了你梦中的任何一种可怕和有魔力的生物。
你花了一个晚上在当地客栈补充体力,看到你的成功被张贴到客栈的可能性变得越来越小,
你变得越来越沮丧。

%.pg
%\nd In the morning you awake, collect your belongings, and 
%set off for the dungeon.  After several days of uneventful 
%travel, you see the ancient ruins that mark the entrance to the 
%Mazes of Menace.  It is late at night, so you make camp at the entrance 
%and spend the night sleeping under the open skies.  In the morning, you 
%gather your gear, eat what may be your last meal outside, and enter the 
%dungeon\ldots
第二天早晨醒来,你收好你的物品出发去\zhTransDungeon{}。
几天平静无事的行程后,你看到了远古的废墟——进入\zhTransMazesOfMenace{}的标志。
此时已经很晚了,于是你在入口处扎营,将晚上剩余的时间用于在露天下睡觉。
早晨,你整理好装备,吃完可能是在外面吃的最后一餐,然后进入了\zhTransDungeon{}……

%.hn 1
%\section{What is going on here?}
\section{这里发生了什么?}

%.pg
%You have just begun a game of {\it NetHack}.  Your goal is to grab as much
%treasure as you can, retrieve the Amulet of Yendor, and escape the
%Mazes of Menace alive.  
你恰好开始了一场 {\it NetHack} 游戏。
你的目标是尽可能多地获取财宝,取得\zhTransAmuletOfYendor
(the Amulet of Yendor),并活着逃离\zhTransMazesOfMenace
(the Mazes of Menace)。

%.pg
%Your abilities and strengths for dealing with the hazards of adventure
%will vary with your background and training:
你处理冒险中面临的危险的才智和能力随着你的背景和所受的训练而变化:
\footnote{以下十三项指的是你的角色可能的职业(role)——翻译者。}

%.pg
%
\blist{}
%\item[\bb{Archeologists}]%
\item[\bb{\zhTransArcheologists{\rm (}Archeologists{\rm )}}]%
%understand dungeons pretty well; this enables them
%to move quickly and sneak up on the local nasties.  They start equipped
%with the tools for a proper scientific expedition.
对于\zhTransDungeon{}非常了解;这允许他们快速移动和偷偷靠近当地的危险怪物。
他们出发时装备了用于真正科学考察的工具。
%.pg
%
%\item[\bb{Barbarians}]%
\item[\bb{\zhTransBarbarians{\rm (}Barbarians{\rm )}}]%
%are warriors out of the hinterland, hardened to battle.
%They begin their quests with naught but uncommon strength, a trusty hauberk,
%and a great two-handed sword.
是来自内地的战士,善于战斗。
他们开始探索时,除了不寻常的力量、可以信任的锁子甲和一柄厉害的\zhTransTwoHandedSword(two-handed sword)以外什么都没有带。
%.pg
%
%\item[\bb{Cavemen {\rm and} Cavewomen}]
\item[\bb{\zhTransCavemen{\rm (}Cavemen {\rm 和} Cavewomen{\rm )}}]
%start with exceptional strength, but unfortunately, neolithic weapons.
带着突出的力量和不幸的新石器时代的武器开始。
%.pg
%
%\item[\bb{Healers}]%
\item[\bb{\zhTransHealers{\rm (}Healers{\rm )}}]%
%are wise in medicine and apothecary.  They know the
%herbs and simples that can restore vitality, ease pain, anesthetize,
%and neutralize
%poisons; and with their instruments, they can divine a being's state
%of health or sickness.  Their medical practice earns them quite reasonable
%amounts of money, with which they enter the dungeon.
在医学和药物学方面十分博学。
% What the hell is the difference between herbs and simples? 
他们熟知可以恢复精力、减少疼痛、麻醉和中和毒物的草药;
借助他们的器械,他们可以推断生物的健康或生病状态。
他们的医学业务为他们挣得了可观的金钱,他们带上它进入了\zhTransDungeon{}。
%.pg
%
%\item[\bb{Knights}]%
\item[\bb{\zhTransKnights{\rm (}Knights{\rm )}}]%
%are distinguished from the common skirmisher by their
%devotion to the ideals of chivalry and by the surpassing excellence of
%their armor.
以他们对骑士精神典范的忠诚和他们卓越的\zhTransArmor{},与一般的散兵区分出来。
%.pg
%
%\item[\bb{Monks}]%
\item[\bb{\zhTransMonks{\rm (}Monks{\rm )}}]%
%are ascetics, who by rigorous practice of physical and mental
%disciplines have become capable of fighting as effectively without weapons
%as with.  They wear no armor but make up for it with increased mobility.
是苦行者,通过严格的肉体和精神的修行,已经有能力不使用武器进行战斗,如同使用了武器一样有效。
他们不装\zhTransArmor{},作为补偿有着机动性上的提升。
%.pg
%
%\item[\bb{Priests {\rm and} Priestesses}]%
\item[\bb{\zhTransPriests{\rm (}Priests {\rm 和} Priestesses{\rm )}}]%
%are clerics militant, crusaders
%advancing the cause of righteousness with arms, armor, and arts
%thaumaturgic.  Their ability to commune with deities via prayer
%occasionally extricates them from peril, but can also put them in it.
是好战的牧师和十字军战士,通过武器、\zhTransArmor{}和法术推进正义目标。
他们通过祈祷(prayer)与神交流的能力有时可以将他们从危险中解救出来,但也可以使他们陷入其中。
%.pg
%
%\item[\bb{Rangers}]%
\item[\bb{\zhTransRangers{\rm (}Rangers{\rm )}}]%
%are most at home in the woods, and some say slightly out
%of place in a dungeon.  They are, however, experts in archery as well
%as tracking and stealthy movement.
最熟悉森林的环境,有人认为他们在\zhTransDungeon{}中有些不合适。
不过他们精通射箭,在追踪和暗中移动上也是如此。
%.pg
%
%\item[\bb{Rogues}]%
\item[\bb{\zhTransRogues{\rm (}Rogues{\rm )}}]%
%are agile and stealthy thieves, with knowledge of locks,
%traps, and poisons.  Their advantage lies in surprise, which they employ
%to great advantage.
是敏捷和悄无声息的小偷,拥有关于锁具、陷阱和毒药的知识。
他们的优势在于出奇不意,他们可以利用它获得极大便利。
%.pg
%
%\item[\bb{Samurai}]%
\item[\bb{\zhTransSamurai{\rm (}Samurai{\rm )}}]%
%are the elite warriors of feudal Nippon.  They are lightly
%armored and quick, and wear the %
%{\it dai-sho}, two swords of the deadliest
%keenness.
是封建时代\zhTransNippon{}的精锐战士。
他们仅装备了轻型\zhTransArmor{},因而行动十分敏捷;他们还佩带了{\it \zhTransDaisho}——两把有着最致命的锐利刀锋的刀。
%.pg
%
%\item[\bb{Tourists}]%
\item[\bb{\zhTransTourists{\rm (}Tourists{\rm )}}]%
%start out with lots of gold (suitable for shopping with),
%a credit card, lots of food, some maps, and an expensive camera.  Most
%monsters don't like being photographed.
携带着大量的金钱(适合用于购物)、一张信用卡(credit card)、大量的食物、一些地图和一架昂贵的相机(expensive camera)开始旅程。
大部分怪物不喜欢被拍照。
%.pg
%
%\item[\bb{Valkyries}]%
\item[\bb{\zhTransValkyries{\rm (}Valkyries{\rm )}}]%
%are hardy warrior women.  Their upbringing in the harsh
%Northlands makes them strong, inures them to extremes of cold, and instills
%in them stealth and cunning.
是强壮的战斗女性。在严厉的\zhTransNorthlands{}中所受的教育使她们变得强壮,使她们习惯于极端的寒冷,并且灌输进了她们的隐密和灵巧。
%.pg
%
%\item[\bb{Wizards}]%
\item[\bb{\zhTransWizards{\rm (}Wizards{\rm )}}]%
%start out with a knowledge of magic, a selection of magical
%items, and a particular affinity for dweomercraft.  Although seemingly weak
%and easy to overcome at first sight, an experienced Wizard is a deadly foe.
带着关于魔法的知识、若干精选的魔法物品和对魔法的特殊亲和性开始。
尽管乍一看似乎很弱,可以轻易战胜,不过一个富有经验的\zhTransWizards{}是一个致命的敌人。
\elist

%.pg
%You may also choose the race of your character:
你同样可以选择你的角色的种族(race):

%.pg
%
\blist{}
%\item[\bb{Dwarves}]%
\item[\bb{\zhTransDwarves{\rm (}Dwarves{\rm )}}]%
%are smaller than humans or elves, but are stocky and solid
%individuals.  Dwarves' most notable trait is their great expertise in mining
%and metalwork.  Dwarvish armor is said to be second in quality not even to the
%mithril armor of the Elves.
比\zhTransHumans{}和\zhTransElves{}体型小,不过他们是矮而壮实和坚强的个体。
\zhTransDwarves{}最显著的特点是他们在采矿和金属加工上的巨大专业知识。
\zhTransDwarves{}的\zhTransArmor{}被认为是质量第二好的,仅次于\zhTransElves{}的秘银铠甲。
%.pg
%
%\item[\bb{Elves}]%
\item[\bb{\zhTransElves{\rm (}Elves{\rm )}}]%
%are agile, quick, and perceptive; very little of what goes
%on will escape an Elf.  The quality of Elven craftsmanship often gives
%them an advantage in arms and armor.
是敏捷、迅速和观察敏锐的;很少有什么发生的事会被一个\zhTransElves{}错过。
\zhTransElves{}技艺的优良经常给他们在武器和\zhTransArmor{}上带来好处。
%.pg
%
%\item[\bb{Gnomes}]%
\item[\bb{\zhTransGnomes{\rm (}Gnomes{\rm )}}]%
%are smaller than but generally similar to dwarves.  Gnomes are
%known to be expert miners, and it is known that a secret underground mine
%complex built by this race exists within the Mazes of Menace, filled with
%both riches and danger.
比\zhTransDwarves{}小,不过此外一般与\zhTransDwarves{}相似。
\zhTransGnomes{}作为专业的矿工而闻名,并且已知在\zhTransMazesOfMenace{}中存在着一个
由该种族建造的秘密的复杂地下矿井,充满了财宝和危险。
%.pg
%
%\item[\bb{Humans}]%
\item[\bb{\zhTransHumans{\rm (}Humans{\rm )}}]%
%are by far the most common race of the surface world, and
%are thus the norm by which other races are often compared.  Although
%they have no special abilities, they can succeed in any role.
显然是地表世界上最常见的种族,因而经常是比较其他种族时的基准。
尽管没有特殊能力,他们可以胜任任何职业(role)。
%.pg
%
%\item[\bb{Orcs}]%
\item[\bb{\zhTransOrcs{\rm (}Orcs{\rm )}}]%
%are a cruel and barbaric race that hate every living thing
%(including other orcs).  Above all others, Orcs hate Elves with a passion
%unequalled, and will go out of their way to kill one at any opportunity.
%The armor and weapons fashioned by the Orcs are typically of inferior quality.
是残忍和野蛮的种族,他们憎恨所有生物(包括其他\zhTransOrcs)。
首先,\zhTransOrcs{}对\zhTransElves{}有着强烈无比的憎恨,会利用任何机会
特意去杀死一个\zhTransElves。
由\zhTransOrcs{}制造的\zhTransArmor{}和武器通常是质量较差的。
\elist

%.hn 1
%\section{What do all those things on the screen mean?}
\section{屏幕上所有这些东西都意味着什么?}
%.pg
%On the screen is kept a map of where you have been and what you have 
%seen on the current dungeon level; as you explore more of the level, 
%it appears on the screen in front of you.
屏幕上维持着一张地图,上面有着当前一层\zhTransDungeon{}上你已经到过的地方和
你已经看到过的东西;
随着你继续探索这一层,其余部分会逐渐出现在你面前的屏幕上。

%.pg
%When {\it NetHack\/}'s ancestor {\it rogue\/} first appeared, its screen
%orientation was almost unique among computer fantasy games.  Since
%then, screen orientation has become the norm rather than the
%exception; {\it NetHack\/} continues this fine tradition.  Unlike text
%adventure games that accept commands in pseudo-English sentences and
%explain the results in words, {\it NetHack\/} commands are all one or two
%keystrokes and the results are displayed graphically on the screen.  A
%minimum screen size of 24 lines by 80 columns is recommended; if the
%screen is larger, only a $21\times80$ section will be used for the map.
当 {\it NetHack\/} 的祖先 {\it rogue\/} 首次出现时,它的屏幕样式
在\zhTransComputerFantasyGames{}中几乎是完全独特的。
自从那时起,这类屏幕样式已经成为常态而非例外;
{\it NetHack\/}保持了这一优良传统。
与接收以伪英语文字构成的命令,并用文字解释结果的\zhTransTextAdventureGames{}不同,
{\it NetHack\/}的命令全部为一个或两个按键,
并且结果以图形的方式显示在屏幕上。
推荐最小的屏幕尺寸为 24 行 80 列;如果屏幕比这更大,只有$21\times80$的区域
会被用于地图。

%.pg
%{\it NetHack\/} can even be played by blind players, with the assistance of
%Braille readers or speech synthesisers.  Instructions for configuring
%{\it NetHack\/} for the blind are included later in this document.
{\it NetHack\/} 甚至可以在盲文阅读器或者语音合成器的帮助下由盲人玩家来玩。
关于如何配置用于盲人的{\it NetHack\/}的指导被包含在这一文档的后面。

%.pg
%{\it NetHack\/} generates a new dungeon every time you play it; even the
%authors still find it an entertaining and exciting game despite
%having won several times.
每一次你玩{\it NetHack\/}时,它会生成一个新的\zhTransDungeon{};几位作者甚至在已经赢了
数次后仍然觉得它是一个有趣和刺激的游戏。

%.pg
%{\it NetHack\/} offers a variety of display options.  The options available to
%you will vary from port to port, depending on the capabilities of your
%hardware and software, and whether various compile-time options were
%enabled when your executable was created.  The three possible display
%options are: a monochrome character interface, a color character interface,
%and a graphical interface using small pictures called tiles.  The two
%character interfaces allow fonts with other characters to be substituted,
%but the default assignments use standard ASCII characters to represent
%everything.  There is no difference between the various display options
%with respect to game play.  Because we cannot reproduce the tiles or
%colors in the Guidebook, and because it is common to all ports, we will
%use the default ASCII characters from the monochrome character display
%when referring to things you might see on the screen during your game.
{\it NetHack\/}提供各种显示选项。可用的选项会随着不同的移植而不同,
这取决于你的硬件和软件能力,以及当你的可执行文件生成时各种编译时选项是否开启。
三种可能的显示选项为:单色字符界面、彩色字符界面和使用被称为瓦片(tiles)
的小图像的图形界面。两种字符界面允许替换带有其他字符的字体,不过默认设置
使用标准的 ASCII 字符表现所有东西。从游戏的角度,各种显示选项相互之间没有区别。
因为我们不能在这份指南中再现瓦片或颜色,并且因为它对于所有移植是共同的,
所以我们将在提及你在游戏中可能在屏幕上所看到的东西时,使用单色字符界面中的
默认 ASCII 字符。
%.pg
%In order to understand what is going on in {\it NetHack}, first you must
%understand what {\it NetHack\/} is doing with the screen.  The {\it NetHack\/}
%screen replaces the ``You see \ldots'' descriptions of text adventure games.
%Figure 1 is a sample of what a {\it NetHack\/} screen might look like.
%The way the screen looks for you depends on your platform.
为了理解{\it NetHack}中发生了什么,首先你必须明白{\it NetHack\/}正在对屏幕
做什么。{\it NetHack\/}的屏幕替代了\zhTransTextAdventureGames{}中的“你看到……”
(“You see \ldots”)描述。图 1 是{\it NetHack\/}的屏幕可能看起来是什么样子
的示例。你所看到的屏幕看起来如何取决于你的平台。

\vbox{
\begin{verbatim}
        The bat bites!

                ------
                |....|    ----------
                |.<..|####...@...$.|
                |....-#   |...B....+
                |....|    |.d......|
                ------    -------|--



        Player the Rambler     St:12 Dx:7 Co:18 In:11 Wi:9 Ch:15  Neutral
        Dlvl:1  $:0  HP:9(12) Pw:3(3) AC:10 Exp:1/19 T:257 Weak
\end{verbatim}
\begin{center}
%Figure 1
图 1
\end{center}
}

%.hn 2
%\subsection*{The status lines (bottom)}
\subsection*{状态行(底部)}

%.pg
%The bottom two lines of the screen contain several cryptic pieces of
%information describing your current status.  If either status line
%becomes longer than the width of the screen, you might not see all of
%it.  Here are explanations of what the various status items mean
%(though your configuration may not have all the status items listed
%below):
屏幕的最下面两行包含了几个晦涩的信息片段,它们描述了你的目前状态。
如果任一行长过屏幕的宽度,你可能无法看到它的全部。
这里解释了各种状态项目的含义(不过你的配置可能没有所有下面列举的状态项目):

%.lp
\blist{}
%\item[\bb{Rank}]
\item[\bb{\zhTransRank{\rm (}Rank{\rm )}}]
%Your character's name and professional ranking (based on the
%experience level, see below).
你角色的名字和职业\zhTransRank(基于\zhTransExperienceLevel,见下文)。
%.lp
%\item[\bb{Strength}]
\item[\bb{\zhTransStrength{\rm (}Strength{\rm )}}]
%A measure of your character's strength; one of your six basic
%attributes.  A human character's attributes can range from 3 to 18 inclusive;
%non-humans may exceed these limits
%(occasionally you may get super-strengths of the form 18/xx, and magic can
%also cause attributes to exceed the normal limits).  The
%higher your strength, the stronger you are.  Strength affects how
%successfully you perform physical tasks, how much damage you do in
%combat, and how much loot you can carry.
关于你角色的力量的衡量;它是你的六个基本\zhTransAttributes(attribute)之一。
一个人类角色的\zhTransAttributes{}可从 3 到 18(含 3 和 18);
非人类角色可能超过这些限制(有时你可能获得形式为 18/xx 的超常\zhTransStrength,
另外魔法也可以使\zhTransAttributes{}超过正常范围)。
你的\zhTransStrength{}越高,你越强壮。\zhTransStrength{}影响你有多成功地执行体力任务、
在战斗中你能造成多少伤害和你能携带多少战利品。
%.lp
%\item[\bb{Dexterity}]
\item[\bb{\zhTransDexterity{\rm (}Dexterity{\rm )}}]
%Dexterity affects your chances to hit in combat, to avoid traps, and
%do other tasks requiring agility or manipulation of objects.
\zhTransDexterity{}影响你在战斗中命中的机会、避开陷阱的机会和其他需要灵活性
或处理物品的任务。
%.lp
%\item[\bb{Constitution}]
\item[\bb{\zhTransConstitution{\rm (}Constitution{\rm )}}]
%Constitution affects your ability to recover from injuries and other
%strains on your stamina.
\zhTransConstitution{}影响你从受伤和其他对体力的损害中恢复的能力。
%.lp
%\item[\bb{Intelligence}]
\item[\bb{\zhTransIntelligence{\rm (}Intelligence{\rm )}}]
%Intelligence affects your ability to cast spells and read spellbooks.
\zhTransIntelligence{}影响你施放咒语(spell)和阅读咒语书(spellbook)的能力。
%.lp
%\item[\bb{Wisdom}]
\item[\bb{\zhTransWisdom{\rm (}Wisdom{\rm )}}]
%Wisdom comes from your practical experience (especially when dealing with
%magic).  It affects your magical energy.
\zhTransWisdom{}来自于你的实践经验(尤其是当处理魔法时)。
它影响你的魔法能量。
%.lp
%\item[\bb{Charisma}]
\item[\bb{\zhTransCharisma{\rm (}Charisma{\rm )}}]
%Charisma affects how certain creatures react toward you.  In
%particular, it can affect the prices shopkeepers offer you.
\zhTransCharisma{}影响某些生物对你的反应。
特别地,它可以影响\zhTransShopkeeper{}(shopkeeper)给你的价格。
%.lp
%\item[\bb{Alignment}]
\item[\bb{\zhTransAlignment{\rm (}Alignment{\rm )}}]
%
%{\it Lawful}, {\it Neutral\/} or {\it Chaotic}.  Often, Lawful is
%taken as good and Chaotic is evil, but legal and ethical do not always
%coincide.  Your alignment influences how other
%monsters react toward you.  Monsters of a like alignment are more likely
%to be non-aggressive, while those of an opposing alignment are more likely
%to be seriously offended at your presence.
{\it Lawful}(\zhTransLawful)、{\it Neutral\/}(\zhTransNeutral)
或者{\it Chaotic}(\zhTransChaotic)。
通常地,\zhTransLawful{}被看成是好的,而\zhTransChaotic{}是邪恶的,
但法律和道德并不总是一致的。
你的\zhTransAlignment{}影响其他的怪物如何对你反应。
相同\zhTransAlignment{}的怪物更有可能没有攻击性,而相反\zhTransAlignment{}的
怪物更可能在你出现时被严重地冒犯。
%.lp
%\item[\bb{Dungeon Level}]
\item[\bb{\zhTransDungeonLevel{\rm (}Dungeon Level{\rm )}}]
%How deep you are in the dungeon.  You start at level one and the number
%increases as you go deeper into the dungeon.  Some levels are special,
%and are identified by a name and not a number.  The Amulet of Yendor is
%reputed to be somewhere beneath the twentieth level.
你在\zhTransDungeon{}中有多深。你开始时位于第一层,随着你深入\zhTransDungeon{},层数会相应增加。
有些层是特殊的,会用一个名字进行标识而非数字。
\zhTransAmuletOfYendor{}被认为位于二十层以下的某处。
%.lp
%\item[\bb{Gold}]
\item[\bb{\zhTransGold{}数{\rm (}Gold{\rm )}}]
%The number of gold pieces you are openly carrying.  Gold which you have
%concealed in containers is not counted.
你敞开携带的黄金的数目。被你藏在容器(container)中的黄金没有计算在内。
%.lp
%\item[\bb{Hit Points}]
\item[\bb{\zhTransHitPoints{\rm (}Hit Points{\rm )}}]
%Your current and maximum hit points.  Hit points indicate how much
%damage you can take before you die.  The more you get hit in a fight,
%the lower they get.  You can regain hit points by resting, or by using
%certain magical items or spells.  The number in parentheses is the maximum
%number your hit points can reach.
你目前的以及最大的\zhTransHitPoints。
\zhTransHitPoints{}显示在你死之前可以承受多少伤害。
在战斗中你被打中越多,它就越低。
你可以通过休息或者通过某些魔法物品或咒语来恢复\zhTransHitPoints。
在括号中的数字是你的\zhTransHitPoints{}能达到的最大值。
%.lp
%\item[\bb{Power}]
\item[\bb{\zhTransPower{\rm (}Power{\rm )}}]
%Spell points.  This tells you how much mystic energy ({\it mana\/})
%you have available for spell casting.  Again, resting will regenerate the
%amount available.
与咒语相关的数值。
这告诉你有多少可以用于施放咒语的神秘能量({\it \zhTransMana\/},{\it mana\/})。
同样地,休息会恢复可用的数量。
%.lp
%\item[\bb{Armor Class}]
\item[\bb{\zhTransArmorClass{\rm (}Armor Class{\rm )}}]
%A measure of how effectively your armor stops blows from unfriendly
%creatures.  The lower this number is, the more effective the armor; it
%is quite possible to have negative armor class.
关于你的\zhTransArmor{}能多有效地阻止不友好生物打击的度量。
这个数字越小,\zhTransArmor{}越有效;相当有可能有一个负数的\zhTransArmorClass。
%.lp
%\item[\bb{Experience}]
\item[\bb{\zhTransExperience{\rm (}Experience{\rm )}}]
%Your current experience level and experience points.  As you
%adventure, you gain experience points.  At certain experience point
%totals, you gain an experience level.  The more experienced you are,
%the better you fight and withstand magical attacks.  Many dungeons
%show only your experience level here.
你目前的\zhTransExperienceLevel{}和\zhTransExperiencePoints。
随着冒险的进行,你会获得\zhTransExperiencePoints。
当总共达到了某些\zhTransExperiencePoints,你的\zhTransExperienceLevel{}就会增加一。
你越有经验,你战斗得越好,且越能经受魔法攻击。
许多\zhTransDungeon{}在这里只显示你的\zhTransExperienceLevel。
%.lp
%\item[\bb{Time}]
\item[\bb{\zhTransTime{\rm (}Time{\rm )}}]
%The number of turns elapsed so far, displayed if you have the
%{\it time\/} option set.
至今已流逝的回合数,当你设置了{\it time}选项时显示。
%.lp
%\item[\bb{Hunger Status}]
\item[\bb{\zhTransHungerStatus{\rm (}Hunger Status{\rm )}}]
%Your current hunger status, ranging from %
%{\it Satiated\/} down to {\it Fainting}.  If your hunger status is normal,
%it is not displayed.
你目前的\zhTransHungerStatus,从{\it Satiated\/}(\zhTransSatiated)到{\it Fainting}(\zhTransFainting)。
如果你的\zhTransHungerStatus{}是正常的,它将不会被显示。
%.pg
%Additional status flags may appear after the hunger status:
%{\it Conf\/} when you're confused, {\it FoodPois\/} or {\it Ill\/}
%when sick, {\it Blind\/}
%when you can't see, {\it Stun\/} when stunned, and {\it Hallu\/} when
%hallucinating.
额外的状态标志可能出现在\zhTransHungerStatus{}后面:
当你\zhTransConfused{}时出现{\it Conf\/},当你生病时出现{\it FoodPois\/}或者
{\it Ill\/},当你看不见东西时出现{\it Blind\/},当你\zhTransStunned{}时出现
{\it Stun\/},当你有幻觉时出现{\it Hallu\/}。
\elist

%.hn 2
%\subsection*{The message line (top)}
\subsection*{消息行(顶部)}

%.pg
%The top line of the screen is reserved for messages that describe
%things that are impossible to represent visually.  If you see a
%``{\tt --More--}'' on the top line, this means that {\it NetHack\/} has
%another message to display on the screen, but it wants to make certain
%that you've read the one that is there first.  To read the next message,
%just press the space bar.
屏幕的顶部预留给一些消息,这些消息描述不可能以图形方式表现的事情。
如果你在最上面一行看到“{\tt --More--}”(更多),这意味着{\it NetHack\/} 
有另一消息需要在屏幕上显示,不过它想要确定你已经阅读完最先出现的消息。
要阅读下一条消息,只需按下空格键。

%.hn 2
%\subsection*{The map (rest of the screen)}
\subsection*{地图(屏幕的余下部分)}

%.pg
%The rest of the screen is the map of the level as you have explored it
%so far.  Each symbol on the screen represents something.  You can set
%various graphics
%options to change some of the symbols the game uses; otherwise, the
%game will use default symbols.  Here is a list of what the default
%symbols mean:
屏幕的剩余部分是当前层上你至今已经探索过的部分的地图。
屏幕上的每一个符号代表某一东西。
你可以设置各种图形选项来改变游戏使用的一些符号;否则,游戏会使用默认的符号。
这里是默认符号含义的列表:

\blist{}
%.lp
%\item[\tb{- {\rm and} |}]
\item[\tb{- {\rm 和} |}]
%The walls of a room, or an open door.  Or a grave ({\tt |}).
房间的\zhTransWall(wall),或者打开的\zhTransDoor(door)。
或者\zhTransGrave(grave,{\tt |})。
%.lp
\item[\tb{.}]
%The floor of a room, ice, or a doorless doorway.
房间的\zhTransFloor(floor)、\zhTransIce(ice)或者没有门的\zhTransDoorway(doorway)。
%.lp
\item[\tb{\#}]
%A corridor, or iron bars, or a tree, or possibly a kitchen sink (if
%your dungeon has sinks), or a drawbridge.
\zhTransCorridor(corridor),或者\zhTransIronBars(iron bar),或者一棵\zhTransTree(tree),或者可能是一具厨房\zhTransSink(sink)
(如果你的\zhTransDungeon{}有\zhTransSink{}的话),或者一座吊桥(drawbridge)。
%.lp
\item[\tb{>}]
%Stairs down: a way to the next level.
向下的\zhTransStairs(stair):通往下一层的路。
%.lp
\item[\tb{<}]
%Stairs up: a way to the previous level.
向上的楼梯:通往前一层的路。
%.lp
\item[\tb{+}]
%A closed door, or a spellbook containing a spell you may be able to learn.
关上的门,或一本\zhTransSpellbook(spellbook),它包含了一个你可能有能力学习的咒语(spell)。
%.lp
\item[\tb{@}]
%Your character or a human.
你的角色或者一个\zhTransHumans。
%.lp
\item[\tb{\$}]
%A pile of gold.
一堆\zhTransGold(gold)。
%.lp
%\item[\tb{\^}]
\item[\tb{\^{}}]
%A trap (once you have detected it).
\zhTransTraps(trap)(一旦你探测到了它)。
%.lp
\item[\tb{)}]
%A weapon.
\zhTransWeapon(weapon)。
%.lp
\item[\tb{[}]
%A suit or piece of armor.
一套或一件\zhTransArmor(armor)。
%.lp
\item[\tb{\%}]
%Something edible (not necessarily healthy).
一些可以吃的(不一定是健康的)。
%.lp
\item[\tb{?}]
%A scroll.
\zhTransScroll(scroll)。
%.lp
\item[\tb{/}]
%A wand.
\zhTransWand(wand)。
%.lp
\item[\tb{=}]
%A ring.
\zhTransRing(ring)。
%.lp
\item[\tb{!}]
%A potion.
\zhTransPotion(potion)。
%.lp
\item[\tb{(}]
%A useful item (pick-axe, key, lamp \ldots).
有用的东西(\zhTransPickAxe(pick-axe)、钥匙(key)、灯(lamp)……)。
%.lp
\item[\tb{"}]
%An amulet or a spider web.
\zhTransAmulet(amulet)或蜘蛛网(spider web)。
%.lp
\item[\tb{*}]
%A gem or rock (possibly valuable, possibly worthless).
\zhTransGem(gem)或石头(rock)(可能贵重的,也可能没有价值的)。
%.lp
\item[\tb{`}]
%A boulder or statue.
\zhTransBoulders(boulder)或\zhTransStatue(statue)。
%.lp
\item[\tb{0}]
%An iron ball.
铁球(iron ball)。
%.lp
\item[\tb{_}]
%An altar, or an iron chain.
\zhTransAltar(altar),或铁链(iron chain)。
%.lp
\item[\tb{\{}]
%A fountain.
\zhTransFountain(fountain)。
%.lp
\item[\tb{\}}]
%A pool of water or moat or a pool of lava.
一池水(pool of water)或\zhTransMoat(moat)或一池\zhTransLava(lava)。
%.lp
\item[\tb{$\backslash$}]
%An opulent throne.
华丽的\zhTransThrone(throne)。
%.lp
%\item[\tb{a-zA-Z {\rm \& other symbols}}]
\item[\tb{a-zA-Z {\rm 以及其他符号}}]
%Letters and certain other symbols represent the various inhabitants
%of the Mazes of Menace.  Watch out, they can be nasty and vicious.
%Sometimes, however, they can be helpful.
字母和一些其他符号代表各种\zhTransMazesOfMenace{}的居民。
当心,他们可能是危险和有恶意的。
然而有时他们可以是很有帮助的。
%.lp
\item[\tb{I}]
%This marks the last known location of an invisible or otherwise unseen
%monster.  Note that the monster could have moved.  The `F' and `m' commands
%may be useful here.
这标记了一个不可见或者未被看见的怪物的最后已知位置。
注意这个怪物可能已经移动了。
“{\tt F}” 和 “{\tt m}” 命令可能在此是有帮助的。

\elist
%.pg
%You need not memorize all these symbols; you can ask the game what any
%symbol represents with the `{\tt /}' command (see the next section for
%more info).
你不需要记住所有这些符号;你可以用“{\tt /}”命令来询问游戏任一符号代表了什么
(更多信息见下一节)。

%.hn 1
%\section{Commands}
\section{命令}

%.pg
%Commands are initiated by typing one or two characters.  Some commands,
%like ``{\tt search}'', do not require that any more information be collected
%by {\it NetHack\/}.  Other commands might require additional information, for
%example a direction, or an object to be used.  For those commands that
%require additional information, {\it NetHack\/} will present you with either 
%a menu of choices, or with a command line prompt requesting information.  Which
%you are presented with will depend chiefly on how you have set the
%`{\it menustyle\/}'
%option.
命令(command)由键入一个或两个字符开始。
某些命令,例如“{\tt s}”(搜索,search),不需要由{\it NetHack\/}收集更多信息。
另一些命令可能需要额外的信息,例如方向,或者使用的物品。
对于那些需要额外信息的命令,{\it NetHack\/}会向你显示一个选择菜单,
或者一个获取信息的命令行提示符。
具体向你显示哪一种主要取决于你如何设置“{\it menustyle\/}”(菜单风格)选项。

%.pg
%For example, a common question in the form ``{\tt What do you want to
%use? [a-zA-Z\ ?*]}'', asks you to choose an object you are carrying.
%Here, ``{\tt a-zA-Z}'' are the inventory letters of your possible choices.
%Typing `{\tt ?}' gives you an inventory list of these items, so you can see
%what each letter refers to.  In this example, there is also a `{\tt *}'
%indicating that you may choose an object not on the list, if you
%wanted to use something unexpected.  Typing a `{\tt *}' lists your entire
%inventory, so you can see the inventory letters of every object you're
%carrying.  Finally, if you change your mind and decide you don't want
%to do this command after all, you can press the `ESC' key to abort the
%command.
举列来说,一个形式为“{\tt What do you want to use? [a-zA-Z\ ?*]}”
(你想使用什么?)的常见问题,会让你选择一个正携带着的物品。
这里的“{\tt a-zA-Z}”是你可能选择的\zhTransInventory{}(inventory)字母。
键入“{\tt ?}”会给出这些物品的\zhTransInventory{}列表,
让你可以看到每一个字母指代的是什么。
在这个例子中,“{\tt *}”也被列举出来,它指明你可以选择一个不在列表中的物品,
如果你想要使用某些意料之外的物品的话。
键入“{\tt *}”会列出你的全部\zhTransInventory,因而你可以看到所有
你携带的物品的\zhTransInventory{}字母。
最后,如果你改变主意,决定不使用这一命令,你可以按下“{\tt ESC}”键来中止这一命令。

%.pg
%You can put a number before some commands to repeat them that many
%times; for example, ``{\tt 10s}'' will search ten times.  If you have the
%{\it number\_pad\/}
%option set, you must type `{\tt n}' to prefix a count, so the example above
%would be typed ``{\tt n10s}'' instead.  Commands for which counts make no
%sense ignore them.  In addition, movement commands can be prefixed for
%greater control (see below).  To cancel a count or a prefix, press the
%`ESC' key.
你可以在某些命令前面加上一个数字,来使它们重复指定的次数;
例如,“{\tt 10s}”会搜索十次。
如果你设置了{\it number\_pad\/}(小键盘)选项,你需要在次数前加上“{\tt n}”前缀,
所以上述例子则变为“{\tt n10s}”。
那些次数对它们没有意义的命令会忽略这些数字。
另外,移动命令可以被冠以次数来获得更好的控制(见下面)。
要取消次数或前缀的话,按“{\tt ESC}”键。

%.pg
%The list of commands is rather long, but it can be read at any time
%during the game through the `{\tt ?}' command, which accesses a menu of
%helpful texts.  Here are the commands for your reference:
命令的列表是非常长的,不过通过“{\tt ?}”命令,可以在游戏中的任何时间阅读它,
该命令会访问一个有帮助文本的菜单。
下面是供你参考的命令:

\blist{}
%.lp
\item[\tb{?}]
%Help menu:  display one of several help texts available.
帮助菜单:显示数个可用的帮助文本中的一个。
%.lp
\item[\tb{/}]
%Tell what a symbol represents.  You may choose to specify a location
%or type a symbol (or even a whole word) to explain.
%Specifying a location is done by moving the cursor to a particular spot
%on the map and then pressing one of `{\tt .}', `{\tt ,}', `{\tt ;}',
%or `{\tt :}'.  `{\tt .}' will explain the symbol at the chosen location,
%conditionally check for ``{\tt More info?}'' depending upon whether the
%{\it help\/}
%option is on, and then you will be asked to pick another location;
%`{\tt ,}' will explain the symbol but skip any additional
%information; `{\tt ;}' will skip additional info and also not bother asking
%you to choose another location to examine; `{\tt :}' will show additional
%info, if any, without asking for confirmation.  When picking a location,
%pressing the {\tt ESC} key will terminate this command, or pressing `{\tt ?}'
%will give a brief reminder about how it works.
告知一个符号所代表的含义。
你可以选择指定一个位置或者输入一个符号(甚至一整个单词)进行解释。
指定一个位置通过移动光标到地图上一个特定的点,然后按“{\tt .}”、“{\tt ,}”、
“{\tt ;}”或“{\tt :}”其中之一来完成。
“{\tt .}”会解释被选择位置的符号,接着取决于{\it help\/}(帮助)选项是否开启,
它会询问是否显示更多信息(“{\tt More info?}”),然后让你选择另一个位置;
“{\tt ,}”会解释该符号但略过所有附加信息;
“{\tt ;}”会略过附加信息,并且不会让你选择要查看的下一个位置;
“{\tt :}”会不经确认就显示附加信息,如果有的话。
当选择位置时,按下{\tt ESC}键会终止这一命令,按下“{\tt ?}”则会给出它如何工作的
简明提示。

%.pg
%Specifying a name rather than a location
%always gives any additional information available about that name.
指定一个名字而非一个位置时总会给出有关这一名字的任何可用的附加信息。
%.lp
\item[\tb{\&}]
%Tell what a command does.
说明一个命令做什么。
%.lp
\item[\tb{<}]
%Go up to the previous level (if you are on a staircase or ladder).
向上前往前一层(如果你站在楼梯或梯子上的话)。
%.lp
\item[\tb{>}]
%Go down to the next level (if you are on a staircase or ladder).
向下前往下一层(如果你站在楼梯或梯子上的话)。
%.lp
\item[\tb{[yuhjklbn]}]
%Go one step in the direction indicated (see Figure 2).  If you sense
%or remember
%a monster there, you will fight the monster instead.  Only these
%one-step movement commands cause you to fight monsters; the others
%(below) are ``safe.''
向指示的方向前进一步(见图 2)。
如果你感觉到或记得有一个怪物在那里,你会攻击该怪物,而非前进。
只有这些一步移动命令会导致你攻击怪物;其他移动命令(下面)是“安全”的。
%.sd
\begin{center}
\begin{tabular}{cc}
\verb+   y  k  u   + & \verb+   7  8  9   +\\
\verb+    \ | /    + & \verb+    \ | /    +\\
\verb+   h- . -l   + & \verb+   4- . -6   +\\
\verb+    / | \    + & \verb+    / | \    +\\
\verb+   b  j  n   + & \verb+   1  2  3   +\\
%                     & (if {\it number\_pad\/} set)
                     & (如果设置了{\it number\_pad\/}的话)
\end{tabular}
\end{center}
%.ed
\begin{center}
%Figure 2
图 2
\end{center}
%.lp
\item[\tb{[YUHJKLBN]}]
%Go in that direction until you hit a wall or run into something.
向该方向前进直到你撞到墙上或者遇上什么东西。
%.lp
\item[\tb{m[yuhjklbn]}]
%Prefix:  move without picking up objects or fighting (even if you remember
%a monster there)
前缀:移动,且不捡起物品或不战斗(即使你记得那里有一个怪物)
%.lp
\item[\tb{F[yuhjklbn]}]
%Prefix:  fight a monster (even if you only guess one is there)
前缀:攻击一个怪物(即使你仅仅是猜测那里有怪物)
%.lp
\item[\tb{M[yuhjklbn]}]
%Prefix:  Move far, no pickup.
前缀:远距离移动,且不捡东西。
%.lp
\item[\tb{g[yuhjklbn]}]
%Prefix:  Move until something interesting is found.
前缀:一直移动直到发现某些有趣的事。
%.lp
%\item[\tb{G[yuhjklbn] {\rm or} <CONTROL->[yuhjklbn]}]
\item[\tb{G[yuhjklbn] {\rm 或} <CONTROL->[yuhjklbn]}]
%Prefix:  Same as `{\tt g}', but forking of corridors is not considered
%interesting.
前缀:与“{\tt g}”相同,不过不认为过道的分岔是有趣的。
%.lp
\item[\tb{_}]
%Travel to a map location via a shortest-path algorithm.  The shortest path
%is computed over map locations the hero knows about (e.g. seen or
%previously traversed).  If there is no known path, a guess is made instead.
%Stops on most of 
%the same conditions as the `G' command, but without picking up
%objects, similar to the `M' command.  For ports with mouse 
%support, the command is also invoked when a mouse-click takes place on a 
%location other than the current position.
依据\zhTransShortestPathAlg{}走到地图上的一个位置。
该\zhTransShortestPath{}由英雄已知的(即已看到或者先前穿过)的地图位置计算得出。
如果没有已知路径,那么将会使用一个猜测的路径。
会在大多数与“{\tt G}”命令相同的条件下停下来,不过不会捡起物品,这与“{\tt M}”命令相似。
对于有鼠标支持的移植,当在一个非当前位置的地方用鼠标点击时,该命令同样会被调用。
%.lp
\item[\tb{.}]
%Rest, do nothing for one turn.
休息,在一回合里什么也不干。
%.lp
\item[\tb{a}]
%Apply (use) a tool (pick-axe, key, lamp \ldots).
使用(apply)一个工具(\zhTransPickAxe(pick-axe)、钥匙(key)、灯(lamp)……)。
%.lp
\item[\tb{A}]
%Remove one or more worn items, such as armor.
%Use `{\tt T}' (take off) to take off only one piece of armor 
%or `{\tt R}' (remove) to take off only one accessory.
移除一件或多件已穿戴的物品,例如\zhTransArmor{}。
使用“{\tt T}”(take off,脱下)来脱掉仅一件\zhTransArmor{},
或“{\tt R}”(remove,移除)来脱掉仅一件装饰物。
%.lp
\item[\tb{\^{}A}]
%Redo the previous command.
重复上一条命令。
%.lp
\item[\tb{c}]
%Close a door.
关上(close)一扇门。
%.lp
\item[\tb{C}]
%Call (name) an individual monster.
命名(call)一个怪物个体。
%.lp
\item[\tb{\^{}C}]
%Panic button.  Quit the game.
应急按钮。退出该游戏。
%.lp
\item[\tb{d}]
%Drop something.\\
%{\tt d7a} --- drop seven items of object
%{\it a}.
放下(drop)某些东西。\\
{\tt d7a} --- 放下七件物品{\it a}。
%.lp
\item[\tb{D}]
%Drop several things.  In answer to the question
%``{\tt What kinds of things do you want to drop? [!\%= BUCXaium]}''
%you should type zero or more object symbols possibly followed by
%`{\tt a}' and/or `{\tt i}' and/or `{\tt u}' and/or `{\tt m}'.
%In addition, one or more of
%the blessed/uncursed/cursed groups may be typed.\\
放下(drop)数件物品。
要回答“{\tt What kinds of things do you want to drop? [!\%= BUCXaium]}”
(你要放下哪几种物品?)这一问题,你应该按下零个或更多个物品种类的符号,
之后可能包括按下“{\tt a}”和/或“{\tt i}”和/或“{\tt u}”和/或“{\tt m}”。
另外,一个或多个\zhTransBlessed/\zhTransUncursed/\zhTransCursed
(blessed/uncursed/cursed)
组也可以被键入。\\
%.sd
%.si
%{\tt DB}  --- drop all objects known to be blessed.\\
%{\tt DU}  --- drop all objects known to be uncursed.\\
%{\tt DC}  --- drop all objects known to be cursed.\\
%{\tt DX}  --- drop all objects of unknown B/U/C status.\\
%{\tt Da}  --- drop all objects, without asking for confirmation.\\
%{\tt Di}  --- examine your inventory before dropping anything.\\
%{\tt Du}  --- drop only unpaid objects (when in a shop).\\
%{\tt Dm}  --- use a menu to pick which object(s) to drop.\\
%{\tt D\%u} --- drop only unpaid food.
{\tt DB}  --- 放下所有已知\zhTransBlessed{}物品。\\
{\tt DU}  --- 放下所有已知\zhTransUncursed{}物品。\\
{\tt DC}  --- 放下所有已知\zhTransCursed{}物品。\\
{\tt DX}  --- 放下所有未知 B/U/C 状态的物品。\\
{\tt Da}  --- 不用确认即放下所有物品。\\
{\tt Di}  --- 在放下任何东西前检查你的\zhTransInventory。\\
{\tt Du}  --- 仅放下没有付款的物品(当在\zhTransShop{}里时)。\\
{\tt Dm}  --- 使用菜单来挑选要放下的物品。\\
{\tt D\%u} --- 只放下没有付款的食物。
%.ei
%.ed
%.lp
\item[\tb{\^{}D}]
%Kick something (usually a door).
踢某个东西(通常是门)。
%.lp
\item[\tb{e}]
%Eat food.
吃(eat)食物。
%.lp
% Make sure Elbereth is not hyphenated below, the exact spelling matters.
% (Only specified here to parallel Guidebook.mn; use of \tt font implicity
% prevents automatic hyphenation in TeX and LaTeX.)
\hyphenation{Elbereth}		%override the deduced syllable breaks
\item[\tb{E}]
%Engrave a message on the floor.
%Engraving the word ``{\tt Elbereth}'' will cause most monsters to not attack
%you hand-to-hand (but if you attack, you will rub it out); this is
%often useful to give yourself a breather.  (This feature may be compiled out
%of the game, so your version might not have it.)\\
在地面上刻写(engrave)一条消息。
刻写“{\tt Elbereth}”一词会使大多数怪物在接近你的时候不攻击你
(不过你进行攻击的话会擦掉刻写的字);
这经常是有用的,可以给你自己一个喘息的机会。
(这一特性可能在编译时并没有包含在游戏中,所以你的版本里可能没有它。)\\
%.sd
%.si
%{\tt E-} --- write in the dust with your fingers.
{\tt E-} --- 用你的手指在尘土中写字。
%.ei
%.ed
%.Ip
\item[\tb{f}]
%Fire one of the objects placed in your quiver.  You may select
%ammunition with a previous `{\tt Q}' command, or let the computer pick
%something appropriate if {\it autoquiver\/} is true.
发射(fire)放在你的箭筒中的一个物品。
你可以预先用“{\tt Q}”命令选择好弹药,或者在{\it autoquiver\/}(自动装备箭筒)
开启时由电脑自动挑选合适的东西。
%.lp
\item[\tb{i}]
%List your inventory (everything you're carrying).
列出你的\zhTransInventory(inventory)(所有你携带的东西)。
%.lp
\item[\tb{I}]
%List selected parts of your inventory.\\
列出选出来的你的部分\zhTransInventory。\\
%.sd
%.si
%{\tt I*} --- list all gems in inventory;\\
%{\tt Iu} --- list all unpaid items;\\
%{\tt Ix} --- list all used up items that are on your shopping bill;\\
%{\tt I\$} --- count your money.
{\tt I*} --- 列出\zhTransInventory{}中的所有宝石;\\
{\tt Iu} --- 列出所有没有付款的物品;\\
{\tt Ix} --- 列出所有在你的购物账单上用掉的东西;\\
{\tt I\$} --- 计算你的金钱。
%.ei
%.ed
%.lp
\item[\tb{o}]
%Open a door.
打开(open)门。
%.lp
\item[\tb{O}]
%Set options.  A menu showing the current option values will be
%displayed.  You can change most values simply by selecting the menu
%entry for the given option (ie, by typing its letter or clicking upon
%it, depending on your user interface).  For the non-boolean choices,
%a further menu or prompt will appear once you've closed this menu.
%The available options
%are listed later in this Guidebook.  Options are usually set before the
%game rather than with the `{\tt O}' command; see the section on options below.
设置选项(option)。会显示一张有着当前选项的值的菜单。
对于大多数选项的值,你只需选择指定选项所对应的菜单条目
(即键入它的字母或者点击它,这取决于你的用户界面)就可以改变它。
对于非布尔类型的选项,当你关闭当前菜单时一个进一步的菜单或提示符会出现。
可用的选项会列在这份指南的后面部分。
选项通常在游戏之前设置,而非使用“{\tt O}”命令;
参见后面关于选项的部分。
%.lp
\item[\tb{p}]
%Pay your shopping bill.
支付(pay)你的购物账单。
%.lp
\item[\tb{P}]
%Put on a ring or other accessory (amulet, blindfold).
戴上(put on)戒指或者其他装饰物品(护身符,眼罩(blindfold))。
%.lp
\item[\tb{\^{}P}]
%Repeat previous message.  Subsequent {\tt \^{}P}'s repeat earlier messages.
%The behavior can be varied via the msg_window option.
重复上一条消息。
后续的{\tt \^{}P}会重复更早的消息。
通过 {\it msg\_window} 选项可以改变具体行为。
%.lp
\item[\tb{q}]
%Quaff (drink) something (potion, water, etc).
喝(quaff)某些东西(药水、水等)。
%.lp
\item[\tb{Q}]
%Select an object for your quiver.  You can then throw this using
%the `f' command.  (In versions prior to 3.3 this was the command to quit
%the game, which has now been moved to `{\tt \#quit}'.)
为你的箭筒(quiver)选择一个物品。
随后你可以使用“{\tt f}”命令发射这一物品。
(在早于 3.3 的版本中,这是用来退出游戏的命令,现在该功能已被移动到了
“{\tt \#quit}”。)
%.lp
\item[\tb{r}]
%Read a scroll or spellbook.
阅读(read)卷轴或咒语书。
%.lp
\item[\tb{R}]
%Remove an accessory (ring, amulet, etc).
移除(remove)一件装饰品(戒指、护身符等)。
%.lp
\item[\tb{\^{}R}]
%Redraw the screen.
重绘(redraw)屏幕。
%.lp
\item[\tb{s}]
%Search for secret doors and traps around you.  It usually takes several
%tries to find something.
在你周围搜索(search)暗门或陷阱。
通常需要尝试多次以找到些什么。
%.lp
\item[\tb{S}]
%Save (and suspend) the game.  The game will be restored automatically the
%next time you play.
保存(save)(并暂停)游戏。
下一次你玩的时候游戏会自动恢复。
%.lp
\item[\tb{t}]
%Throw an object or shoot a projectile.
扔(throw)一个物品,或发射一个发射物。
%.lp
\item[\tb{T}]
%Take off armor.
脱下(take off)\zhTransArmor{}。
%.lp
\item[\tb{\^{}T}]
%Teleport, if you have the ability.
瞬间移动(teleport),如果你有这一能力的话。
%.lp
\item[\tb{v}]
%Display version number.
显示版本(version)号。
%.lp
\item[\tb{V}]
%Display the game history.
显示游戏的历史。
%.lp
\item[\tb{w}]
%Wield weapon.\\
手持(wield)武器。\\
%.sd
%.si
%{\tt w-} --- wield nothing, use your bare hands.
{\tt w-} --- 什么也不拿,空手作战。
%.ei
%.ed
%.lp
\item[\tb{W}]
%Wear armor.
穿戴(wear)\zhTransArmor。
%.lp
\item[\tb{x}]
%Exchange your wielded weapon with the item in your alternate
%weapon slot.  The latter is used as your secondary weapon when engaging in
%two-weapon combat.  Note that if one of these slots is empty,
%the exchange still takes place.
将你手持的武器与在你备用武器槽里的武器交换。
当你在双武器作战(two-weapon combat)时,后者会成为你的次要武器。
注意,如果你的其中一个武器槽是空的话,交换仍然会进行。
%.lp
\item[\tb{X}]
%Enter explore (discovery) mode, explained in its own section later.
进入探索模式(explore mode),这将会在后面它自己的部分中进行解释。
%.lp
\item[\tb{\^{}X}]
%Display your name, role, race, gender, and alignment as well as
%the various deities in your game.
显示你的名字、职业、种族、性别和\zhTransAlignment,以及你游戏中的各个神。
%.lp
\item[\tb{z}]
%Zap a wand.  To aim at yourself, use `{\tt .}' for the direction.
挥动(zap)魔杖。
要指向你自己,使用“{\tt .}”作为方向。
%.lp
\item[\tb{Z}]
%Zap (cast) a spell.  To cast at yourself, use `{\tt .}' for the direction.
施放(zap)咒语。
要向你自己施放,使用“{\tt .}”作为方向。
%.lp
\item[\tb{\^{}Z}]
%Suspend the game (UNIX versions with job control only).
挂起游戏(只针对有作业控制的 UNIX 版本)。
%.lp
\item[\tb{:}]
%Look at what is here.
查看这里有什么。
%.lp
\item[\tb{;}]
%Show what type of thing a visible symbol corresponds to.
查看一个可见符号对应于什么物品种类。
%.lp
\item[\tb{,}]
%Pick up some things. May be preceded by `{\tt m}' to force a selection menu.
捡起一些东西。
可以加上前缀“{\tt m}”来强制打开选择菜单。
%.lp
\item[\tb{@}]
%Toggle the {\it autopickup\/} option on and off.
切换{\it autopickup\/}(自动捡起)选项打开或关闭。
%.lp
\item[\tb{\^{}}]
%Ask for the type of a trap you found earlier.
询问一个你先前发现的陷阱的类型。
%.lp
\item[\tb{)}]
%Tell what weapon you are wielding.
告知你正握持的武器。
%.lp
\item[\tb{[}]
%Tell what armor you are wearing.
告知你正穿戴的\zhTransArmor{}。
%.lp
\item[\tb{=}]
%Tell what rings you are wearing.
告知你正穿戴的戒指。
%.lp
\item[\tb{"}]
%Tell what amulet you are wearing.
告知你正穿戴的护身符。
%.lp
\item[\tb{(}]
%Tell what tools you are using.
告知你正在使用的工具。
%.lp
\item[\tb{*}]
%Tell what equipment you are using; combines the preceding five type-specific
%commands into one.
告知你正在使用的装备;将上述五个指定具体类型的命令合并成了一个。
%.lp
\item[\tb{\$}]
%Count your gold pieces.
计算你的黄金数。
%.lp
\item[\tb{+}]
%List the spells you know.  Using this command, you can also rearrange
%the order in which your spells are listed.  They are shown via a menu,
%and if you select a spell in that menu, you'll be re-prompted for
%another spell to swap places with it, and then have opportunity to
%make further exchanges.
列出你知道的咒语。
使用这一命令,你还可以重新排列咒语列出的顺序。
它们由菜单显示,如果你在菜单中选择了一个咒语,提示符会重新出现,
让你选择与它交换位置的另一个咒语,之后你还有机会做进一步交换。
%.lp
\item[\tb{$\backslash$}]
%Show what types of objects have been discovered.
显示你已经发现的物品的种类。
%.lp
\item[\tb{!}]
%Escape to a shell.
脱离到一个 shell 中。
%.lp
\item[\tb{\#}]
%Perform an extended command.  As you can see, the authors of {\it NetHack\/}
%used up all the letters, so this is a way to introduce the less frequently
%used commands.
%What extended commands are available depends on what features
%the game was compiled with.
执行一个扩展命令(extended command)。
正如你所看到的,{\it NetHack\/}作者们用完了所有字母,
所以这是一种引进那些不经常使用的命令的方法。
有哪些扩展命令可用取决于游戏编译进了哪些特性。
%.lp
\item[\tb{\#adjust}]
%Adjust inventory letters (most useful when the
%{\it fixinv\/}
%option is ``on'').
(调整\footnote{该括号内的内容为该扩展命令中的单词的含义,下同——翻译者。})
调整\zhTransInventory{}字母(当{\it fixinv\/}(固定\zhTransInventory)
选项打开时时最有用)。
%.lp
\item[\tb{\#chat}]
%Talk to someone.
(聊天)与某人交谈。
%.lp
\item[\tb{\#conduct}]
%List which challenges you have adhered to.  See the section below entitled
%``Conduct'' for details.
(\zhTransConduct)列出你自愿接受的挑战。
详情参见下面标题为“\zhTransConduct”(conduct)的部分。
%.lp
\item[\tb{\#dip}]
%Dip an object into something.
(浸泡)将一个物品浸泡到某些东西中。
%.lp
\item[\tb{\#enhance}]
%Advance or check weapons and spell skills.
(增强)提升或检查武器和咒语的技能(skill)。
%.lp
\item[\tb{\#force}]
%Force a lock.
(用力打开)强行撬开锁。
%.lp
\item[\tb{\#invoke}]
%Invoke an object's special powers.
(召唤)召唤一个物品的特殊力量。
%.lp
\item[\tb{\#jump}]
%Jump to another location.
(跳)跳到另一个位置。
%.lp
\item[\tb{\#loot}]
%Loot a box or bag on the floor beneath you, or the saddle 
%from a horse standing next to you.
(掠夺)搜刮在你脚下地面上的盒子(box)或包(bag),或者站在你旁边的马的鞍(saddle)。
%.lp
\item[\tb{\#monster}]
%Use a monster's special ability (when polymorphed into monster form).
(怪物)使用怪物的特殊能力(当\zhTransPolymorph{}为怪物形态时)。
%.lp
\item[\tb{\#name}]
%Name an item or type of object.
(命名)命名一件物品或物品的种类。
%.lp
\item[\tb{\#offer}]
%Offer a sacrifice to the gods.
(献祭)给神献祭牺牲。
%.lp
\item[\tb{\#pray}]
%Pray to the gods for help.
(祈祷)向神祈祷以获得帮助。
%.lp
\item[\tb{\#quit}]
%Quit the program without saving your game.
(退出)在不保存你的游戏的情况下退出程序。
%.lp
\item[\tb{\#ride}]
%Ride (or stop riding) a monster.
(骑)骑(或停止骑)一个怪物。
%.lp
\item[\tb{\#rub}]
%Rub a lamp or a stone.
(摩擦)摩擦灯或石头。
%.lp
\item[\tb{\#sit}]
%Sit down.
(坐)坐下。
%.lp
\item[\tb{\#turn}]
%Turn undead.
(驱赶)驱赶\zhTransUndead。
%.lp
\item[\tb{\#twoweapon}]
%Toggle two-weapon combat on or off.  Note that you must
%use suitable weapons for this type of combat, or it will
%be automatically turned off.
(双武器)切换双武器战斗(two-weapon combat)打开或关闭。
注意你必须使用合适的武器来进行这种类型的战斗,否则它会被自动关闭。
%.lp
\item[\tb{\#untrap}]
%Untrap something (trap, door, or chest).
(解除陷阱)将某东西(陷阱、门或箱子)中的机关解除。
%.lp
\item[\tb{\#version}]
%Print compile time options for this version of {\it NetHack}.
(版本)打印这一版本{\it NetHack}的编译时选项。
%.lp
\item[\tb{\#wipe}]
%Wipe off your face.
(擦试)擦干清你的脸。
%.lp
\item[\tb{\#?}]
%Help menu:  get the list of available extended commands.
帮助菜单:给出可用扩展命令的列表。
\elist

%.pg
%\nd If your keyboard has a meta key (which, when pressed in combination
%with another key, modifies it by setting the `meta' [8th, or `high']
%bit), you can invoke many extended commands by meta-ing the first
%letter of the command.
\nd 如果你的键盘有 meta 键(当与另一键组合按下时,会设置这一键的“meta”
(第 8 位,或“高位”)比特),你可以通过 meta 键加上命令的第一个字母的方式
来调用许多扩展命令。
%- In {\it NT, OS/2, PC\/ {\rm and} ST NetHack},
%- the `Alt' key can be used in this fashion;
%- on the Amiga set the {\it altmeta\/} option to get this behavior.
%In {\it NT, OS/2, {\rm and} PC NetHack},
%the `Alt' key can be used in this fashion.
在 NT、OS/2、PC 和 ST 移植的{\it NetHack}中,
“{\tt Alt}”键使用这一特性。
\blist{}
%.lp 
\item[\tb{M-?}]
%{\tt\#?} (not supported by all platforms)
{\tt\#?} (并非所有平台都支持)
%.lp
\item[\tb{M-2}]
%{\tt\#twoweapon} (unless the {\it number\_pad\/} option is enabled)
{\tt\#twoweapon} (除非启用了{\it number\_pad\/}选项)
%.lp
\item[\tb{M-a}]
{\tt\#adjust}
%.lp
\item[\tb{M-c}]
{\tt\#chat}
%.lp
\item[\tb{M-d}]
{\tt\#dip}
%.lp
\item[\tb{M-e}]
{\tt\#enhance}
%.lp
\item[\tb{M-f}]
{\tt\#force}
%.lp
\item[\tb{M-i}]
{\tt\#invoke}
%.lp
\item[\tb{M-j}]
{\tt\#jump}
%.lp
\item[\tb{M-l}]
{\tt\#loot}
%.lp
\item[\tb{M-m}]
{\tt\#monster}
%.lp
\item[\tb{M-n}]
{\tt\#name}
%.lp
\item[\tb{M-o}]
{\tt\#offer}
%.lp
\item[\tb{M-p}]
{\tt\#pray}
%.Ip
\item[\tb{M-q}]
{\tt\#quit}
%.lp
\item[\tb{M-r}]
{\tt\#rub}
%.lp
\item[\tb{M-s}]
{\tt\#sit}
%.lp
\item[\tb{M-t}]
{\tt\#turn}
%.lp
\item[\tb{M-u}]
{\tt\#untrap}
%.lp
\item[\tb{M-v}]
{\tt\#version}
%.lp
\item[\tb{M-w}]
{\tt\#wipe}
\elist

%.pg
%\nd If the {\it number\_pad\/} option is on, some additional letter commands
%are available:
\nd 如果开启了{\it number\_pad\/}选项,一些额外的字母命令将是可用的:
\blist{}
%.lp 
\item[\tb{h}]
%Help menu:  display one of several help texts available, like ``{\tt ?}''.
帮助(help)菜单:显示数个可用的帮助文本中的一个,与“{\tt ?}”相似。
%.lp
\item[\tb{j}]
%Jump to another location.  Same as ``{\tt \#jump}'' or ``{\tt M-j}''.
跳到另一位置。与“{\tt \#jump}”或“{\tt M-j}”相同。
%.lp
\item[\tb{k}]
%Kick something (usually a door).  Same as `{\tt \^{}D}'.
踢(kick)某个东西(通常是门)。与“{\tt \^{}D}”相同。
%.lp
\item[\tb{l}]
%Loot a box or bag on the floor beneath you, or the saddle 
%from a horse standing next to you.  Same as ``{\tt \#loot}'' or ``{\tt M-l}''.
搜刮在你脚下地面上的盒子或包,或者站在你旁边的马的鞍。
与“{\tt \#loot}”或“{\tt M-l}”相同。
%.lp
\item[\tb{N}]
%Name an object or type of object.  Same as ``{\tt \#name}'' or ``{\tt M-n}''.
命名一件物品或物品的种类。
与“{\tt \#name}”或“{\tt M-n}”相同。
%.lp
\item[\tb{u}]
%Untrap a trap, door, or chest.  Same as ``{\tt \#untrap}'' or ``{\tt M-u}''.
将某东西(陷阱、门或箱子)中的机关解除。与“{\tt \#untrap}”或“{\tt M-u}”相同。
\elist

%.hn 1
%\section{Rooms and corridors}
\section{房间和过道}

%.pg
%Rooms and corridors in the dungeon are either lit or dark.
%Any lit areas within your line of sight will be displayed;
%dark areas are only displayed if they are within one space of you. 
%Walls and corridors remain on the map as you explore them.
在\zhTransDungeon{}中的房间(room)和过道(corridor)不是明亮的就是黑暗的。
在你视线范围内的任何明亮区域都会显示出来;
黑暗的区域只有它们在距你一格的范围内才会显示。
墙壁(wall)与过道会以你探索它们时的状态保留在地图上。

%.pg
%Secret corridors are hidden.  You can find them with the `{\tt s}' (search)
%command.
秘密过道是隐藏着的。
你可以用“{\tt s}”(search,搜索)命令找出它们。

%.hn 2
%\subsection*{Doorways}
\subsection*{门口(Doorways)}

%.pg
%Doorways connect rooms and corridors.  Some doorways have no doors;
%you can walk right through.  Others have doors in them, which may be
%open, closed, or locked.  To open a closed door, use the `{\tt o}' (open)
%command; to close it again, use the `{\tt c}' (close) command.
门口连接房间与过道。
有些门口没有门(door);你可以直接走过去。
另一些有门,这些门可能是打开的、关上的或者锁上的。
使用“{\tt o}”(open,打开)命令来打开一扇关上的门;
如需要重新关上它,使用“{\tt c}”(close,关闭)命令。

%.pg
%You can get through a locked door by using a tool to pick the lock
%with the `{\tt a}' (apply) command, or by kicking it open with the
%`{\tt \^{}D}' (kick) command.
你可以用“{\tt a}”(apply,使用)命令来使用一个工具拨开门的锁,
或者用“{\tt \^{}D}”(kick,踢)命令把门踢开,
从而通过一扇锁上的门。

%.pg
%Open doors cannot be entered diagonally; you must approach them
%straight on, horizontally or vertically.  Doorways without doors are
%not restricted in this fashion.
打开的门不能从对角线方向通过;
你必须水平或垂直地正对着它们通过。
没有门的门口没有这样的限制。

%.pg
%Doors can be useful for shutting out monsters.  Most monsters cannot
%open doors, although a few don't need to (ex.\ ghosts can walk through
%doors).
门可以用来将怪物关在门外。
大多数怪物不能打开门,虽然一些怪物不需要打开门
(例如鬼魂(ghost)可以径直穿过门)。

%.pg
%Secret doors are hidden.  You can find them with the `{\tt s}' (search)
%command.  Once found they are in all ways equivalent to normal doors.
暗门(secret door)是隐藏着的。你可以用“{\tt s}”(search,搜索)命令找出它们。
一旦被找到了,它们与普通的门是完全相同的。

%.hn 2
% TODO replace dungeon feature with zhTrans-commands 
%\subsection*{Traps (`{\tt \^{}}')}
\subsection*{\zhTransTraps(Traps,“{\tt \^{}}”)}

%.pg
%There are traps throughout the dungeon to snare the unwary delver.
%For example, you may suddenly fall into a pit and be stuck for a few
%turns trying to climb out.  Traps don't appear on your map until you
%see one triggered by moving onto it, see something fall into it, or you
%discover it with the `{\tt s}' (search) command.  Monsters can fall prey to
%traps, too, which can be a very useful defensive strategy.
在\zhTransDungeon{}中到处都是\zhTransTraps,用来捕获粗心的搜索者。
例如,你可能会突然落入一个\zhTransPit{}(pit)中,
需要试着从里面爬出而被困在里面少许回合。
\zhTransTraps{}不会出现在你的地图上,直到你走进去而看到它被触发、
看到某些东西落入里面或者你通过“{\tt s}”(search,搜索)命令找出它。
怪物同样会陷入\zhTransTraps{}中,这可以作为一个非常有用的防御策略。

%.pg
%There is a special pre-mapped branch of the dungeon based on the
%classic computer game ``{\tt Sokoban}.''  The goal is to push the boulders
%into the pits or holes.  With careful foresight, it is possible to
%complete all of the levels according to the traditional rules of
%Sokoban.  Some allowances are permitted in case the player gets stuck;
%however, they will lower your luck.
存在着一个特殊的预先绘好了地图的\zhTransDungeon{}分支,它基于经典电脑游戏“{\tt \zhTransSokoban}”
(Sokoban)。
它的目标是将\zhTransBoulders{}(boulder)推入\zhTransPit{}或\zhTransHole{}(hole)中。
凭借着仔细的考虑,可以根据\zhTransSokoban{}的传统规则将所有层完成。
万一玩家被困住了,一些补救措施是允许的;不过它们会降低你的\zhTransLuck(luck)。

%\subsection*{Stairs (`{\tt <}', `{\tt >}')}
\subsection*{\zhTransStairs(Stairs,“{\tt <}”和“{\tt >}”)}

%.pg
%In general, each level in the dungeon will have a staircase going up
%(`{\tt <}') to the previous level and another going down (`{\tt >}')
%to the next
%level.  There are some exceptions though.  For instance, fairly early
%in the dungeon you will find a level with two down staircases, one
%continuing into the dungeon and the other branching into an area
%known as the Gnomish Mines.  Those mines eventually hit a dead end,
%so after exploring them (if you choose to do so), you'll need to
%climb back up to the main dungeon.
一般来说,\zhTransDungeon{}中的每一层都会有一座向上(“{\tt >}”)进往前一层的\zhTransStairs,
和另一座向下(“{\tt <}”)去往下一层的\zhTransStairs。
不过有一些例外。
例如,在\zhTransDungeon{}相当早的地方,你会找到这样的一层,它有两座向下的\zhTransStairs{},
一座会继续深入\zhTransDungeon{},而另一座会分岔进入一个被称为\zhTransGnomishMines{}
(Gnomish Mines)的地方。
那些矿坑最终会存在一个尽头,所以探索完它们后(如果你选择这样做的话),
你需要往回向上爬,以回到主\zhTransDungeon{}。

%.pg
%When you traverse a set of stairs, or trigger a trap which sends you
%to another level, the level you're leaving will be deactivated and
%stored in a file on disk.  If you're moving to a previously visited
%level, it will be loaded from its file on disk and reactivated.  If
%you're moving to a level which has not yet been visited, it will be
%created (from scratch for most random levels, from a template for
%some ``special'' levels, or loaded from the remains of an earlier game
%for a ``bones'' level as briefly described below).  Monsters are only
%active on the current level; those on other levels are essentially
%placed into stasis.
当你穿过一系列\zhTransStairs,或者触发了一个会将你送往另一层的\zhTransTraps,
你离开的这一层会被冻结并保存在硬盘上的一个文件中。
如果你移动到先前访问过的一层,它会从硬盘上它的文件中载入并重新激活。
如果你移动到先前没有访问过的一层,它会被创建
(对于大多数随机的层会从头创建,对于一些“特殊”的层会从模板创建,
对于“\zhTransBones”(bones)层会从早先的一次游戏的遗留物中加载,
最后这一情况将会在后面进行简要介绍)。
怪物只有在当前层才会活动;那些其他层上的怪物事实上是停滞的。
 
%.pg
%Ordinarily when you climb a set of stairs, you will arrive on the
%corresponding staircase at your destination.  However, pets (see below)
%and some other monsters will follow along if they're close enough when
%you travel up or down stairs, and occasionally one of these creatures
%will displace you during the climb.  When that occurs, the pet or other
%monster will arrive on the staircase and you will end up nearby.
通常当你爬过一系列\zhTransStairs{}时,你会到达目的地的相应\zhTransStairs。
然而,\zhTransPets(pet,见下面)和一些其他怪物如果与你足够近的话,
会在你向上或向下穿过楼梯时会跟随你,
有时它们之一还会在爬的时候替换掉你。
当这种情况发生时,\zhTransPets{}或其他怪物会到达\zhTransStairs,
而你会停在旁边。

%\subsection*{Ladders (`{\tt <}', `{\tt >}')}
\subsection*{\zhTransLadders(Ladders,“{\tt <}”和“{\tt >}”)}

%.pg
%Ladders serve the same purpose as staircases, and the two types of
%inter-level connections are nearly indistinguishable during game play.
\zhTransLadders{}有着与\zhTransStairs{}相同的用途,
并且这两种层间连接在游戏中几乎是无法分辨的。

%.hn 2
%\subsection*{Shops and shopping}
\subsection*{\zhTransShop(Shops)与购物(shopping)}

%.pg
%Occasionally you will run across a room with a shopkeeper near the door
%and many items lying on the floor.  You can buy items by picking them
%up and then using the `{\tt p}' command.  You can inquire about the price
%of an item prior to picking it up by using the ``{\tt \#chat}'' command
%while standing on it.  Using an item prior to paying for it will incur a
%charge, and the shopkeeper won't allow you to leave the shop until you
%have paid any debt you owe.
有时你会走入到一间有着\zhTransShopkeeper{}(shopkeeper)站在门边,
以及许多物品放在地面上的房间。
你可以通过捡起物品然后使用“{\tt p}”命令来购买它们。
你可以通过站在一个物品上并使用“{\tt \#chat}”(闲聊)命令,
来在捡起它前询问它的价格。
在付钱之前就使用一个物品会产生费用,除非你付清所有欠下的债务,
\zhTransShopkeeper{}不会让你离开\zhTransShop。

%.pg
%You can sell items to a shopkeeper by dropping them to the floor while
%inside a shop.  You will either be offered an amount of gold and asked
%whether you're willing to sell, or you'll be told that the shopkeeper
%isn't interested (generally, your item needs to be compatible with the
%type of merchandise carried by the shop).
你可以通过在\zhTransShop{}里面将物品放到地面上这样的方式,
来将物品出售给\zhTransShopkeeper。
\zhTransShopkeeper{}会给你一个报价并询问你是否愿意出售,
或者他会告诉你他对这物品没有兴趣
(一般地,你的物品需要与该\zhTransShop{}出售的货物类型相容)。

%.pg
%If you drop something in a shop by accident, the shopkeeper will usually
%claim ownership without offering any compensation.  You'll have to buy
%it back if you want to reclaim it.
如果你在\zhTransShop{}里不小心放下了东西,
\zhTransShopkeeper{}通常会声明它们归他所有,且不会给任何补偿。
如果你要取回的话,你需要购买它们。

%.pg
%Shopkeepers sometimes run out of money.  When that happens, you'll be
%offered credit instead of gold when you try to sell something.  Credit
%can be used to pay for purchases, but it is only good in the shop where
%it was obtained; other shopkeepers won't honor it.  (If you happen to
%find a ``credit card'' in the dungeon, don't bother trying to use it in
%shops; shopkeepers will not accept it.)
\zhTransShopkeeper{}有时会用完了现金。
当这种情况发生时,在你试图卖出某些东西时,你会得到\zhTransCredit{}(credit)而非黄金。
\zhTransCredit{}可以用来购买物品,但只在获得它的\zhTransShop{}才有效;
其他\zhTransShopkeeper{}不会兑现它。
(如果你在\zhTransDungeon{}中找到了一张“\zhTransCreditCard”(credit card),
不要费心试着在\zhTransShop{}里使用它;\zhTransShopkeeper{}是不会承认它的。)

%.pg
%The {\tt \$} command, which reports the amount of gold you are carrying
%(in inventory, not inside bags or boxes), will also show current shop
%debt or credit, if any.  The {\tt Iu} command lists unpaid items
%(those which still belong to the shop) if you are carrying any.
%The {\tt Ix} command shows an inventory-like display of any unpaid
%items which have been used up, along with other shop fees, if any.
{\tt \$}命令会报告你所携带的黄金数目
(在\zhTransInventory{}中的,而非在包或盒子中的),
也会显示当前\zhTransShop{}的欠款或\zhTransCredit,如果有的话。
{\tt Iu}命令会列出未付款(unpaid)的物品(同样指属于该\zhTransShop{}的),
如果你有携带着的话。
{\tt Ix}命令会以\zhTransInventory{}风格的方式显示用掉的未付款物品,
以及其他的\zhTransShop{}费用,如果有的话。

%.hn 3
%\subsubsection*{Shop idiosyncracies}
\subsubsection*{\zhTransShop{}的特性}

%.pg
%Several aspects of shop behavior might be unexpected.
一些\zhTransShop{}的行为可能是出忽意料的。

\begin{itemize}
% note: a bullet is the default item label so we could omit [$\bullet$] here
%.lp \(bu 2
\item[$\bullet$]
%The price of a given item can vary due to a variety of factors.
给定物品的价格可能会由于多种因素而变化。
%.lp \(bu 2
\item[$\bullet$]
%A shopkeeper treats the spot immediately inside the door as if it were
%outside the shop.
\zhTransShopkeeper{}将门内紧挨着门的一格视为是在\zhTransShop{}外面的。
%.lp \(bu 2
\item[$\bullet$]
%While the shopkeeper watches you like a hawk, he will generally ignore
%any other customers.
尽管\zhTransShopkeeper{}像鹰一样盯着你,他却一般会忽略其他顾客。
%.lp \(bu 2
\item[$\bullet$]
%If a shop is ``closed for inventory'', it will not open of its own accord.
如果一间\zhTransShop{}“因清点库存而关闭”(closed for inventory),
那么它是不会自行打开的。
%.lp \(bu 2
\item[$\bullet$]
%Shops do not get restocked with new items, regardless of inventory depletion.
即使存量耗尽,\zhTransShop{}也不会用新的物品来补充库存。
\end{itemize}

%.hn 1
%\section{Monsters}
\section{怪物}

%.pg
%Monsters you cannot see are not displayed on the screen.  Beware!
%You may suddenly come upon one in a dark place.  Some magic items can
%help you locate them before they locate you (which some monsters can do
%very well).
你看不见的怪物(monster)不会显示在屏幕上。
注意!你可能会在黑暗的地方突然遇上一个。
某些魔法物品可以在它们找出你(某些怪物对此十分擅长)
之前帮助你找出它们。

%.pg
%The commands `{\tt /}' and `{\tt ;}' may be used to obtain information
%about those
%monsters who are displayed on the screen.  The command `{\tt C}' allows you
%to assign a name to a monster, which may be useful to help distinguish
%one from another when multiple monsters are present.  Assigning a name
%which is just a space will remove any prior name.
“{\tt /}”和“{\tt ;}”命令可以用来获得屏幕上显示出的怪物的信息。
命令“{\tt C}”允许你给一个怪物取名字,这可以在多个怪物出现时
帮助你将一个怪物与另一个区分出来。
取只有一个空格的名字会移除之前的任何名字。

%.pg
%The extended command ``{\tt \#chat}'' can be used to interact with an adjacent
%monster.  There is no actual dialog (in other words, you don't get to
%choose what you'll say), but chatting with some monsters such as a
%shopkeeper or the Oracle of Delphi can produce useful results.
扩展命令“{\tt \#chat}”(闲聊)可以用来与一个相邻的怪物交互。
实际上并没有真正的对话发生
(换句话说,你不需要选择你将要说的话),
不过与诸如\zhTransShopkeeper{}或\zhTransOracleOfDelpi(the Oracle of Delphi)
之类的怪物交谈可以产生有用的结果。

%.hn 2
%\subsection*{Fighting}
\subsection*{战斗}

%.pg
%If you see a monster and you wish to fight it, just attempt to walk
%into it.  Many monsters you find will mind their own business unless
%you attack them.  Some of them are very dangerous when angered.
%Remember:  discretion is the better part of valor.
如果你看到一个怪物并希望攻击它,只需要试着走到它所在的位置。
许多你找到的怪物只关心它们自己的事,除非你进攻它们。
它们中的一些在生气的时候是非常危险的。
记住:谨慎即大勇(discretion is the better part of valor)。

%.pg
%If you can't see a monster (if it is invisible, or if you are blinded),
%the symbol `I' will be shown when you learn of its presence.
%If you attempt to walk into it, you will try to fight it just like
%a monster that you can see; of course,
%if the monster has moved, you will attack empty air.  If you guess
%that the monster has moved and you don't wish to fight, you can use the `m'
%command to move without fighting; likewise, if you don't remember a monster
%but want to try fighting anyway, you can use the `F' command.
如果你看不到一个怪物(它是\zhTransInvisible(invisible),
或者你正处于\zhTransBlind{}(blinded)状态),符号“{\tt I}”会在你察觉到它时显示。
如果你试着走进它,你会试着攻击它,正如一个你可以看见的怪物那样;
当然,如果该怪物已经移动了,你则会打空。
如果你推测该怪物已经移动了,且你不希望进攻,你可以使用“{\tt m}”命令移动而不攻击;
类似地,如果你不记得有怪物存在,但无论如何都要进攻,
你可以使用“{\tt F}”命令。

%.hn 2
%\subsection*{Your pet}
\subsection*{你的宠物(pet)}

%.pg
%You start the game with a little dog (`{\tt d}'), cat (`{\tt f}'),
%or pony (`{\tt u}'), which follows
%you about the dungeon and fights monsters with you.  Like you, your
%pet needs food to survive.  It usually feeds itself on fresh carrion
%and other meats.  If you're worried about it or want to train it, you
%can feed it, too, by throwing it food.  A properly trained pet can be
%very useful under certain circumstances.
你带着一只\zhTransLittleDog(little dog,“{\tt d}”)、
一只猫(cat,“{\tt f}”)或者一匹\zhTransPony(pony,“{\tt u}”)开始游戏,
它们会跟随你到\zhTransDungeon{}去,并与你一起对怪物作战。
与你一样,你的宠物需要食物来生存。
它通常会用新鲜的死肉和其他肉来喂饱自己。
如果你担心它或想要训练它,你也可以喂它,通过向它扔食物的方式。
一只被正确训练的宠物在一些情形下将是非常有帮助的。

%.pg
%Your pet also gains experience from killing monsters, and can grow
%over time, gaining hit points and doing more damage.  Initially, your
%pet may even be better at killing things than you, which makes pets
%useful for low-level characters.
你的宠物同样会从杀死怪物中获得经验,由此会随时间而长大,
从而能获得\zhTransHitPoints{}并对怪物造成更多的伤害。
最开始,你的宠物甚至比你在杀死怪物上要更擅长,
这使得宠物对于低等级的角色是有帮助的。

%.pg
%Your pet will follow you up and down staircases if it is next to you
%when you move.  Otherwise your pet will be stranded and may become
%wild.  Similarly, when you trigger certain types of traps which alter
%your location (for instance, a trap door which drops you to a lower
%dungeon level), any adjacent pet will accompany you and any non-adjacent
%pet will be left behind.  Your pet may trigger such traps itself; you
%will not be carried along with it even if adjacent at the time.
如果当你上或下楼梯时你的宠物在你的旁边,那么它会跟随你一起移动。
否则你的宠物会受困,并且有可能变成野生的。
相似地,当你触发了某种会更变你的位置的\zhTransTraps{}
(例如,\zhTransTrapDoor(trap door)会把你扔到更深的\zhTransDungeon{}层中)时,
相邻的宠物会跟着你,非相邻的则会被留下。
你的宠物可能会自己触发这样的\zhTransTraps;
即使此时你与它是相邻的,你也不会被一同带走。

%.hn 2
%\subsection*{Steeds}
\subsection*{\zhTransSteeds(Steeds)}

%.pg
%Some types of creatures in the dungeon can actually be ridden if you
%have the right equipment and skill.  Convincing a wild beast to let
%you saddle it up is difficult to say the least.  Many a dungeoneer
%has had to resort to magic and wizardry in order to forge the alliance.
%Once you do have the beast under your control however, you can
%easily climb in and out of the saddle with the `{\tt \#ride}' command.  Lead
%the beast around the dungeon when riding, in the same manner as
%you would move yourself.  It is the beast that you will see displayed
%on the map.
如果你有合适的装备和技能的话,\zhTransDungeon{}中的一些种类的生物事实上可以用来骑乘。
退一步说,说服一只野生的\zhTransBeast{}让你给它安上鞍(saddle)是困难的。
许多\zhTransDungeoneer{}需要借助于魔法和巫术以便于促成这一结盟。
不过一旦你控制了一只\zhTransBeast{},
你可以用“{\tt \#ride}”(骑)命令来方便地爬上和爬下鞍座。
当骑乘时,引导这匹\zhTransBeast{}在\zhTransDungeon{}中移动的方法与你自己移动的方法相同。
显示在地图上的将会是该\zhTransBeast{}。

%.pg
%Riding skill is managed by the `{\tt \#enhance}' command.  See the section
%on Weapon proficiency for more information about that.
骑乘技能由“{\tt \#enhance}”(增强)命令进行管理。
浏览下面关于武器熟练度的部分以获得更多有关信息。

%.hn 2
%\subsection*{Bones levels}
\subsection*{\zhTransBones{}层(Bones levels)}

%.pg
%You may encounter the shades and corpses of other adventurers (or even
%former incarnations of yourself!) and their personal effects.  Ghosts
%are hard to kill, but easy to avoid, since they're slow and do little
%damage.  You can plunder the deceased adventurer's possessions;
%however, they are likely to be cursed.  Beware of whatever killed the
%former player; it is probably still lurking around, gloating over its
%last victory.
你可能会遇到其他冒险者的鬼魂和尸体
(甚至自己以前的化身!)以及他们的个人财产。
\zhTransGhosts(ghost)是较难杀死的,不过很容易避开,
毕竟它们移动缓慢且只造成少量伤害。
你可以掠夺已死的冒险者的财产;
不过它们很有可能是\zhTransCursed(cursed)。
当心杀死之前玩家的东西;
它很可能仍然潜伏在附近,为它上一次的胜利而沾沾自喜。

%.hn 1
%\section{Objects}
\section{物品}

%.pg
%When you find something in the dungeon, it is common to want to pick
%it up.  In {\it NetHack}, this is accomplished automatically by walking over
%the object (unless you turn off the {\it autopickup\/}
%option (see below), or move with the `{\tt m}' prefix (see above)), or
%manually by using the `{\tt ,}' command.
当你在\zhTransDungeon{}中发现了些什么,想要捡起它是很自然的。
在{\it NetHack}中,这可以通过走到该物品(object)上方而自动实现
(除非你关闭了{\it autopickup\/}(自动捡起)选项(见后面),
或者使用了“{\tt m}”前缀来移动(见前面)),
或者通过“{\tt ,}”命令来手动完成。
%.pg
%If you're carrying too many items, {\it NetHack\/} will tell you so and you
%won't be able to pick up anything more.  Otherwise, it will add the object(s)
%to your pack and tell you what you just picked up.
如果你携带了太多东西,{\it NetHack\/}会将这一情况告诉你,
此时你将无法再捡起更多的东西。
如果并非这种情形的话,它会将这个(些)物品加入到你的背包中,
并告诉你刚刚所捡起的东西。
%.pg
%As you add items to your inventory, you also add the weight of that object
%to your load.  The amount that you can carry depends on your strength and
%your constitution.  The
%stronger you are, the less the additional load will affect you.  There comes
%a point, though, when the weight of all of that stuff you are carrying around
%with you through the dungeon will encumber you.  Your reactions
%will get slower and you'll burn calories faster, requiring food more frequently
%to cope with it.  Eventually, you'll be so overloaded that you'll either have
%to discard some of what you're carrying or collapse under its weight.
在你将物品加入你的\zhTransInventory{}(inventory)的同时,
你也将该物品的重量加入到你的负重当中。
你所能携带的量取决于你的\zhTransStrength{}(strength)和\zhTransConstitution(constitution)。
你越强壮,该额外负荷对你的影响越小。
不过这最终会到达一点,此时你在穿越\zhTransDungeon{}时携带的所有东西的重量会拖累你。
你的反应会变慢,你会更快地燃烧卡路里,会更频繁地需要食物以应付它。
最终,你会变得如此过载,以至于你不得不扔掉一些你所携带的东西,
或者被它的重量所压垮。
%.pg
%NetHack will tell you how badly you have loaded yourself.  The symbols
%`Burdened', `Stressed', `Strained', `Overtaxed' and `Overloaded' are
%displayed on the bottom line display to indicate your condition.
{\it NetHack\/}会告诉你你的负荷有多严重。
“{\it Burdened}”(\zhTransBurdened)、“{\it Stressed}”(\zhTransStressed)、
“{\it Strained}”(\zhTransStrained)、“{\it Overtaxed}”(\zhTransOvertaxed)
和“{\it Overloaded}”(\zhTransOverloaded)这些符号会显示在
底部行上,以显示你的状态。

%.pg
%When you pick up an object, it is assigned an inventory letter.  Many
%commands that operate on objects must ask you to find out which object
%you want to use.  When {\it NetHack\/} asks you to choose a particular object
%you are carrying, you are usually presented with a list of inventory
%letters to choose from (see Commands, above).
当你捡起一个物品时,它会被分配给一个\zhTransInventory{}字母。
许多操作物品的命令必须让你找出你想使用的物品。
当{\it NetHack\/}让你选择一个你携带的特定物品时,
通常会向你显示一个可供选择的\zhTransInventory{}字母的列表(见上面的命令部分)。

%.pg
%Some objects, such as weapons, are easily differentiated.  Others, like
%scrolls and potions, are given descriptions which vary according to
%type.  During a game, any two objects with the same description are
%the same type.  However, the descriptions will vary from game to game.
一些物品,如武器(weapon),是容易区分出来的。
另一些,如卷轴(scroll)和药水(potion),给出的是会随种类而变的描述。
在一个游戏中,有着相同描述的任何两个物品有着相同的类型。
不过,这些描述在一个游戏与另一个游戏之间会变化。

%.pg
%When you use one of these objects, if its effect is obvious, {\it NetHack\/}
%will remember what it is for you.  If its effect isn't extremely
%obvious, you will be asked what you want to call this type of object
%so you will recognize it later.  You can also use the ``{\tt \#name}''
%command for the same purpose at any time, to name all objects of a
%particular type or just an individual object.
%When you use ``{\tt \#name}'' on an object which has already been named,
%specifying a space as the value will remove the prior name instead
%of assigning a new one.
当你使用这些物品之一时,如果它的效果是明白的,
{\it NetHack\/}会替你记住它是什么。
如果它的效果不是特别明显,你会被询问你想要将这一物品的种类命名为什么,
从而之后你可以识别它。
你同样可以在任何时候使用“{\tt \#name}”(命名)命令来命名一种特定类型的所有物品
或者单个物品,以达到和前述相同的目的。
当你对一个已经被命名的物品使用“{\tt \#name}”时,
将名字指定为一个空格会移除之前的名字,而非取一个新的名字。

%.hn 2
%\subsection*{Curses and Blessings}
\subsection*{\zhTransCurses(Curses)和\zhTransBlessing(Blessings)}

%.pg
%Any object that you find may be cursed, even if the object is
%otherwise helpful.  The most common effect of a curse is being stuck
%with (and to) the item.  Cursed weapons weld themselves to your hand
%when wielded, so you cannot unwield them.  Any cursed item you wear
%is not removable by ordinary means.  In addition, cursed arms and armor
%usually, but not always, bear negative enchantments that make them
%less effective in combat.  Other cursed objects may act poorly or
%detrimentally in other ways.
你找到的任何物品都有可能是\zhTransCursed(cursed),即使该物品是有用处的。
\zhTransCursed{}最常见效果是被这一物品缠住。
\zhTransCursed{}武器在握持时会将它们与你的手焊(weld)在一起,以致于你不能松开它们。
你穿戴的任何\zhTransCursed{}物品无法通过正常手段移除。
另外,\zhTransCursed{}武器和\zhTransArmor{}通常,不过并不总是,
具有负数的\zhTransEnchantments(enchantment),使得它们在战斗中不那么有效。
其他\zhTransCursed{}物品可能在其他方面表现糟糕或者有害。

%.pg
%Objects can also be blessed.  Blessed items usually work better or
%more beneficially than normal uncursed items.  For example, a blessed
%weapon will do more damage against demons.
物品同样可以是\zhTransBlessed(blessed)。
\zhTransBlessed{}物品通常比正常的\zhTransUncursed{}(uncursed)物品工作得更好或更有效。
例如,\zhTransBlessed{}武器会对\zhTransDemons{}(demon)造成更多的伤害。

%.pg
%There are magical means of bestowing or removing curses upon objects,
%so even if you are stuck with one, you can still have the curse
%lifted and the item removed.  Priests and Priestesses have an innate
%sensitivity to this property in any object, so they can more easily avoid
%cursed objects than other character roles.
存在着可以赋予或者移除物品上的\zhTransCurses{}的魔法方法,
所以即使你被缠住了,你仍然可以使\zhTransCurses{}消除从而将物品移除。
\zhTransPriests{}(Priest 和 Priestess)对任何物品的这种状态有先天的敏感性,
所以他们可以比其他职业更容易避开\zhTransCursed{}物品。

%.pg
%An item with unknown status will be reported in your inventory with no prefix.
%An item which you know the state of will be distinguished in your inventory
%by the presence of the word ``cursed'', ``uncursed'' or ``blessed'' in the
%description of the item.
一个未知状态的物品会以没有前缀的形式显示在你的\zhTransInventory{}中。
一个你知道它的状态的物品可以通过在该物品的描述中
“{\tt cursed}”(\zhTransCursed)、“{\tt uncursed}”(\zhTransUncursed)
或“{\tt blessed}”(\zhTransBlessed)等词的出现而在\zhTransInventory{}中将它区分出来。

%.hn 2
%\subsection*{Weapons (`{\tt )}')}
\subsection*{武器(Weapons,“{\tt )}”)}

%.pg
%Given a chance, most monsters in the Mazes of Menace will gratuitously try to
%kill you.  You need weapons for self-defense (killing them first).  Without a
%weapon, you do only 1--2 hit points of damage (plus bonuses, if any).
%Monk characters are an exception; they normally do much more damage with
%bare hands than they do with weapons.
只要有机会,大多数\zhTransMazesOfMenace{}中的怪物会没有理由地想要杀死你。
你需要武器来自卫(首先得杀死它们)。
没有武器的话,你仅能造成 1--2 点\zhTransHitPoints{}的伤害
(加上\zhTransBonuses,如果有的话)。
\zhTransMonks{}(Monk)角色是一个例外;他们通常空手造成的伤害要比使用武器时造成的多。

%.pg
%There are wielded weapons, like maces and swords, and thrown weapons,
%like arrows and spears.  To hit monsters with a weapon, you must wield it and
%attack them, or throw it at them.  You can simply elect to throw a spear.
%To shoot an arrow, you should first wield a bow, then throw the arrow.
%Crossbows shoot crossbow bolts.  Slings hurl rocks and (other) stones
%(like gems).
有握持着使用的武器,如\zhTransMaces{}(mace)和\zhTransSwords(sword),
也有投掷使用的武器,如\zhTransArrows(arrow)和\zhTransSpears(spear)。
要使用一件武器攻击怪物,你必须握住它再向它们进攻,或者将它扔向它们。
你可以简单地选择投掷\zhTransSpears。
要发射一支\zhTransArrows,你应该先手持一柄\zhTransBow(bow),
然后再射该\zhTransArrows。
\zhTransCrossbow(crossbow)发射\zhTransCrossbowBolt(crossbow bolt)。
\zhTransSling(sling)投掷石头(rock)和(其他)石子(例如宝石(gem))。

%.pg
%Enchanted weapons have a ``plus'' (or ``to hit enhancement'' which can be
%either positive or negative) that adds to your chance to
%hit and the damage you do to a monster.  The only way to determine a weapon's
%enchantment is to have it magically identified somehow.
%Most weapons are subject to some type of damage like rust.  Such
%``erosion'' damage can be repaired.
\zhTransEnchanted{}武器有“附加值”(或者“命中提升值”,它可以是正的或负的),
它可以增加你命中怪物的机会和你对怪物造成的伤害。
确定一件武器\zhTransEnchantments{}(enchantment)的唯一方法是以某种方式对它进行魔法鉴定。
大多数武器会受某些类型的损害,例如生锈(rust)。
这样的“腐蚀性”损害是可以被修复的。

%.pg
%The chance that an attack will successfully hit a monster, and the amount
%of damage such a hit will do, depends upon many factors.  Among them are:
%type of weapon, quality of weapon (enchantment and/or erosion), experience
%level, strength, dexterity, encumbrance, and proficiency (see below).  The
%monster's armor class---a general defense rating, not necessarily due to
%wearing of armor---is a factor too; also, some monsters are particularly
%vulnerable to certain types of weapons.
一次攻击可以成功地命中怪物的机会,以及这样一次攻击会造成的伤害,
由许多因素决定。
它们中包括:武器的类型、武器的质量(\zhTransEnchantments{}和/或腐蚀(erosion))、
\zhTransExperienceLevel、\zhTransStrength、\zhTransDexterity、负重程度
和技能熟练度(见下面)。
怪物的\zhTransArmorClass——一种基本的防御等级,不一定归因于所穿的护甲
——同样也是一个因素;
同样地,一些怪物对于特定类型的武器是特别脆弱的。

%.pg
%Many weapons can be wielded in one hand; some require both hands.
%When wielding a two-handed weapon, you can not wear a shield, and
%vice versa.  When wielding a one-handed weapon, you can have another
%weapon ready to use by setting things up with the `{\tt x}' command, which
%exchanges your primary (the one being wielded) and alternate weapons.
%And if you have proficiency in the ``two weapon combat'' skill, you
%may wield both weapons simultaneously as primary and secondary; use the
%`{\tt \#twoweapon}' extended command to engage or disengage that.  Only
%some types of characters (barbarians, for instance) have the necessary
%skill available.  Even with that skill, using two weapons at once incurs
%a penalty in the chance to hit your target compared to using just one
%weapon at a time.
许多武器可以用单手握持;一些则需要两只手。
当握着双手大小的武器时,你不能穿\zhTransShield(shield),反之亦然。
当你握持着单手大小的武器时,你可以有另一件武器准备好,
可以用“{\tt x}”命令装备使用,该命令会交换你的主要(当前握持的)与备用武器。
如果你有“双武器战斗”(two weapon combat)技能的熟练度,
你可同时握持这两样武器,作为主要的和次要的;
使用“{\tt \#twoweapon}”扩展命令来进入或者离开这种状态。
只有某些种类的角色(例如\zhTransBarbarians(barbarian))具有可用的必须技能。
即使有该技能,同时使用两件武器相比于只使用一件武器会在
击中你的目标的机会上有惩罚。

%.pg
%There might be times when you'd rather not wield any weapon at all.
%To accomplish that, wield `{\tt -}', or else use the `{\tt A}' command which
%allows you to unwield the current weapon in addition to taking off
%other worn items.
可能有时候你不想握持任何武器。
为了做到如此,握持“{\tt -}”,
或者使用“{\tt A}”命令,它允许你解除当前武器,此外还可以脱下其他穿着的物品。

%.pg
%Those of you in the audience who are AD\&D players, be aware that each
%weapon which existed in AD\&D does roughly the same damage to monsters in
%{\it NetHack}.  Some of the more obscure weapons (such as the %
%{\it aklys}, {\it lucern hammer}, and {\it bec-de-corbin\/}) are defined
%in an appendix to {\it Unearthed Arcana}, an AD\&D supplement.
那些身为专家级龙与地下城(AD\&D)玩家的读者,请注意每一件在 AD\&D 中出现
的武器在{\it NetHack}中会对怪物造成相近的伤害。
某些更加费解的武器(例如{\it\zhTransAklys}(aklys)、
{\it\zhTransLucernHammer}(lucern hammer)
和{\it\zhTransBecDeCorbin}(bec-de-corbin))
被定义在 AD\&D 的补充\zhTransUnearthedArcana({\it Unearthed Arcana})的附录中。

%.pg
%The commands to use weapons are `{\tt w}' (wield), `{\tt t}' (throw),
%`{\tt f}' (fire, an alternative way of throwing), `{\tt Q}' (quiver),
%`{\tt x}' (exchange), `{\tt \#twoweapon}', and `{\tt \#enhance}' (see below).
使用武器的命令有“{\tt w}”(握持,wield)、“{\tt t}”(投掷,throw)、
“{\tt f}”(发射,fire,投掷的另一种方式)、“{\tt Q}”(装备箭筒,quiver)、
“{\tt x}”(交换,exchange)、“{\tt \#twoweapon}”(双武器)
和“{\tt \#enhance}”(增强,见下面)。

%.hn 3
%\subsection*{Throwing and shooting}
\subsection*{投掷(throwing)与发射(shooting)}

%.pg
%You can throw just about anything via the `{\tt t}' command.  It will prompt
%for the item to throw; picking `{\tt ?}' will list things in your inventory
%which are considered likely to be thrown, or picking `{\tt *}' will list
%your entire inventory.  After you've chosen what to throw, you will
%be prompted for a direction rather than for a specific target.  The
%distance something can be thrown depends mainly on the type of object
%and your strength.  Arrows can be thrown by hand, but can be thrown
%much farther and will be more likely to hit when thrown while you are
%wielding a bow.
你可以用“{\tt t}”命令来扔任何东西。
它会弹出提示符来询问要扔的物品;
选择“{\tt ?}”会列出你的\zhTransInventory{}中看起来适合用来扔的东西,
选择“{\tt *}”则会列出你的全部\zhTransInventory。
当你选择好要扔的物品之后,你会被提示给一个方向而非某个特定目标。
物品可以被扔出的距离主要取决于物品的类型和你的\zhTransStrength。
\zhTransArrows{}可以用手来扔,不过在你持\zhTransBow{}进行发射时会射得更远,
并且更有可能命中。

%.pg
%You can simplify the throwing operation by using the `{\tt Q}' command to
%select your preferred ``missile'', then using the `{\tt f}' command to
%throw it.  You'll be prompted for a direction as above, but you don't
%have to specify which item to throw each time you use `{\tt f}'.  There is
%also an option,
%{\it autoquiver},
%which has {\it NetHack\/} choose another item to automatically fill your
%quiver when the inventory slot used for `{\tt Q}' runs out.
你可以使用“{\tt Q}”命令选择你偏好的“弹药”,然后用“{\tt f}”命令来发射它,
从而简化发射操作。
你仍会像上面那样被要求给出一个方向,不过每一次使用“{\tt f}”时
不需要指定要发射的物品。
这里还有一个选项{\it autoquiver}(自动装备箭筒),
它会让{\it NetHack\/}在用于“{\tt Q}”的\zhTransInventory{}槽空了时
自动选择另一物品来填充你的箭筒。

%.pg
%Some characters have the ability to fire a volley of multiple items in a
%single turn.  Knowing how to load several rounds of ammunition at
%once---or hold several missiles in your hand---and still hit a
%target is not an easy task.  Rangers are among those who are adept
%at this task, as are those with a high level of proficiency in the
%relevant weapon skill (in bow skill if you're wielding one to
%shoot arrows, in crossbow skill if you're wielding one to shoot bolts,
%or in sling skill if you're wielding one to shoot stones).
%The number of items that the character has a chance to fire varies from
%turn to turn.  You can explicitly limit the number of shots by using a
%numeric prefix before the `{\tt t}' or `{\tt f}' command.
%For example, ``{\tt 2f}'' (or ``{\tt n2f}'' if using
%{\it number\_pad\/}
%mode) would ensure that at most 2 arrows are shot
%even if you could have fired 3.  If you specify
%a larger number than would have been shot (``{\tt 4f}'' in this example),
%you'll just end up shooting the same number (3, here) as if no limit
%had been specified.  Once the volley is in motion, all of the items
%will travel in the same direction; if the first ones kill a monster,
%the others can still continue beyond that spot.
一些角色有能力在一个回合中一齐发射多个物品。
知道如何同时装载多轮弹药——或者在你手上保持数个弹药——
并且仍然命中目标不是一件容易的任务。
\zhTransRangers{}(Ranger)属于那些精于此任务的人,
同样地还属于那些在相关武器技能中
(在\zhTransBow{}技能中如果你持\zhTransBow{}发射\zhTransArrows{}的话,
在\zhTransCrossbow{}技能中如果你持\zhTransCrossbow{}发射\zhTransCrossbowBolt
的话,在\zhTransSling{}技能中如果你持\zhTransSling{}发射石头的话)
有高熟练度的人。
角色有机会发射的物品数量随不同回合而变。
你可以在“{\tt t}”或“{\tt f}”命令前加上一个数字前缀来明确地限制发射的数量。
例如,“{\tt 2f}”(或“{\tt n2f}”,如果使用{\it number\_pad\/}(小键盘)模式)
会确保最多只有 2 枚\zhTransArrows{}被发射,即使你有能力发射 3 枚。
如果你指定一个比可以发射的要大的数(在这个例子中如“{\tt 4f}”),
你最终只会射出相同的数目(3,此处),与没有指定限制时一样。
一旦齐射已经发射,所有这些物品会以相同方向前进;
如果第一发杀死了一个怪物,其余的仍会继续前进而超过那一点。

%.hn 3
%\subsection*{Weapon proficiency}
\subsection*{武器熟练度(weapon proficiency)}

%.pg
%You will have varying degrees of skill in the weapons available.
%Weapon proficiency, or weapon skills, affect how well you can use
%particular types of weapons, and you'll be able to improve your skills
%as you progress through a game, depending on your role, your experience
%level, and use of the weapons.
你会在不同的可用武器中有着不同的技能程度。
武器熟练度,或者武器技能(skill),会影响你特定类型的武器使用得有多好,
并且随着游戏的进行,你将能够改进你的技能,
这取决于你的职业、\zhTransExperienceLevel{}和武器的使用程度。

%.pg
%For the purposes of proficiency, weapons have
%been divided up into various groups such as daggers, broadswords, and
%polearms.  Each role has a limit on what level of proficiency a character
%can achieve for each group.  For instance, wizards can become highly
%skilled in daggers or staves but not in swords or bows.
出于熟练度的考虑,武器被分成了各种各样的分类,例如\zhTransDagger(dagger)、
\zhTransBroadsword(broadsword)和\zhTransPolearm(polearm)。
每一种职业有着一个角色在各个分类中可以达到多高熟练程度的限制。
例如,\zhTransWizards{}(Wizard)可以在\zhTransDagger{}或\zhTransStaff(staff)中获得很高技能,
不过在\zhTransSwords{}或\zhTransBow{}中却做不到。

%.pg
%The `{\tt \#enhance}' extended command is used to review current weapons proficiency
%(also spell proficiency) and to choose which skill(s) to improve when
%you've used one or more skills enough to become eligible to do so.  The
%skill rankings are ``none'' (sometimes also referred to as ``restricted'',
%because you won't be able to advance), ``unskilled'', ``basic'', ``skilled'',
%and ``expert''.  Restricted skills simply will not appear in the list
%shown by `{\tt \#enhance}'.  (Divine intervention might unrestrict a particular
%skill, in which case it will start at unskilled and be limited to basic.)
%Some characters can enhance their barehanded combat or martial arts skill
%beyond expert to ``master'' or ``grand master''.
“{\tt \#enhance}”(增强)扩展命令被用来回顾当前的武器熟练度
(也包括咒语熟练度)以及当你已经使用一种或多种技能足够多,从而有资格提升技能时,
选择哪一种技能来提升。
技能的等级为“{\tt none}”(\zhTransNone,有时被提及为“{\tt restricted}”
(\zhTransRestricted),因为你无法提升它)、“{\tt unskilled}”(\zhTransUnskilled)、
“{\tt basic}”(\zhTransBasic)、“{\tt skilled}”(\zhTransSkilled)
和“{\tt expert}”(\zhTransExpert)。
\zhTransRestricted{}技能不会出现在“{\tt \#enhance}”显示的列表中。
(\zhTransDivineIntervention{}可能会解除某一特定技能的限制,
这种情况下该技能会以\zhTransUnskilled{}开始并被限制为\zhTransBasic。)
一些角色可以将他们的空手搏斗(barehanded combat)或\zhTransMartialArts
(martial arts)技能提升到超过\zhTransExpert,而达到
“{\tt master}”(\zhTransMaster)或者“{\tt grand master}”(\zhTransGrandMaster)。

%.pg
%Use of a weapon in which you're restricted or unskilled
%will incur a modest penalty in the chance to hit a monster and also in
%the amount of damage done when you do hit; at basic level, there is no
%penalty or bonus; at skilled level, you receive a modest bonus in the
%chance to hit and amount of damage done; at expert level, the bonus is
%higher.  A successful hit has a chance to boost your training towards
%the next skill level (unless you've already reached the limit for this
%skill).  Once such training reaches the threshold for that next level,
%you'll be told that you feel more confident in your skills.  At that
%point you can use `{\tt \#enhance}' to increase one or more skills.  Such skills
%are not increased automatically because there is a limit to your total
%overall skills, so you need to actively choose which skills to enhance
%and which to ignore.
使用一件你受限制(restricted)或不熟练(unskilled)的武器会在命中怪物的机会上
有一不大的惩罚,当你击中怪物时造成的伤害量也同样有惩罚;
在基本(basic)水平时既没有惩罚也没有\zhTransBonuses;
在熟练(skilled)水平时,你会在命中机会和施加的伤害量上得到
一适度的\zhTransBonuses;
在内行(expert)水平时,\zhTransBonuses{}会更高。
一次成功的命中有机会增强你向下一技能水平的训练
(除非你已经达到了该技能的上限)。
一旦这样的训练达到了下一水平的阈值,你会被告知你在你的技能上感到更有自信
(you feel more confident in your skills)。
此时你可以使用“{\tt \#enhance}”来提升一种或多种技能。
这种技能是不会自动提升的,因为存在着你的所有技能的上限,
所以你需要积极地选择哪些技能来增强,哪些则忽略。

%.hn 2
%\subsection*{Armor (`{\tt [}')}
\subsection*{\zhTransArmor{}(Armor,“{\tt [}”)}

%.pg
%Lots of unfriendly things lurk about; you need armor to protect
%yourself from their blows.  Some types of armor offer better
%protection than others.  Your armor class is a measure of this
%protection.  Armor class (AC) is measured as in AD\&D, with 10 being
%the equivalent of no armor, and lower numbers meaning better armor.
%Each suit of armor which exists in AD\&D gives the same protection in
%{\it NetHack}.  Here is an (incomplete) list of the armor classes provided by
%various suits of armor:
大量不友好的生物正在潜伏着;
你需要\zhTransArmor{}来保护你免受它们的重击。
一些种类的\zhTransArmor{}比其他一些提供更好的保护。
你的\zhTransArmorClass{}是这种保护的一种量度。
\zhTransArmorClass(armor class,AC)由与 AD\&D 中相同的方式进行衡量,
10 等同于没有穿任何护甲,越小的数字意味着越好的护甲。
在 AD\&D 中出现的每一套护甲在{\it NetHack}中提供相同的保护。
下面是一份(不完全的)各种护甲提供的\zhTransArmorClass{}的列表:

%\begin{center}
%\begin{tabular}{lllll}
%dragon scale mail      & 1 & \makebox[20mm]{}  & plate mail            & 3\\
%crystal plate mail     & 3 &                   & bronze plate mail     & 4\\
%splint mail            & 4 &                   & banded mail           & 4\\
%dwarvish mithril-coat  & 4 &                   & elven mithril-coat    & 5\\
%chain mail             & 5 &                   & orcish chain mail     & 6\\
%scale mail             & 6 &                   & studded leather armor & 7\\
%ring mail              & 7 &                   & orcish ring mail      & 8\\
%leather armor          & 8 &                   & leather jacket        & 9\\
%no armor               & 10
%\end{tabular}
%\end{center}

\begin{center}
\begin{tabular}{ll}
\zhTransDragonScaleMail(dragon scale mail)               & 1\\
\zhTransPlateMail(plate mail)                            & 3\\
\zhTransCrystalPlateMail(crystal plate mail)             & 3\\
\zhTransBronzePlateMail(bronze plate mail)               & 4\\
\zhTransSplintMail(splint mail)                          & 4\\
\zhTransBandedMail(banded mail)                          & 4\\
\zhTransDwarvishMithrilCoat(dwarvish mithril-coat)       & 4\\
\zhTransElvenMithrilCoat(elven mithril-coat)             & 5\\
\zhTransChainMail(chain mail)                            & 5\\
\zhTransOrcishChainMail(orcish chain mail)               & 6\\
\zhTransScaleMail(scale mail)                            & 6\\
\zhTransStuddedLeatherArmor(studded leather armor)       & 7\\
\zhTransRingMail(ring mail)                              & 7\\
\zhTransOrcishRingMail(orcish ring mail)                 & 8\\
\zhTransLeatherArmor(leather armor)                      & 8\\
\zhTransLeatherJacket(leather jacket)                    & 9\\
没有护甲                                                   & 10
\end{tabular}
\end{center}

%.pg
%\nd You can also wear other pieces of armor (ex.\ helmets, boots,
%shields, cloaks)
%to lower your armor class even further, but you can only wear one item
%of each category (one suit of armor, one cloak, one helmet, one
%shield, and so on) at a time.
\nd 你还可以装戴其他种类的护甲(如头盔(helmet)、鞋子(boot)、\zhTransShield(shield)、斗蓬(cloak))
来进一步降低你的\zhTransArmorClass,不过你只能同时穿每一种类别中的一件物品
(一套铠甲、一件斗蓬、一顶头盔、一件\zhTransShield{}等等)。

%.pg
%If a piece of armor is enchanted, its armor protection will be better
%(or worse) than normal, and its ``plus'' (or minus) will subtract from
%your armor class.  For example, a +1 chain mail would give you
%better protection than normal chain mail, lowering your armor class one
%unit further to 4.  When you put on a piece of armor, you immediately
%find out the armor class and any ``plusses'' it provides.  Cursed
%pieces of armor usually have negative enchantments (minuses) in
%addition to being unremovable.
如果一件护甲是\zhTransEnchanted(enchanted),它的保护会比一般的更好(或更差),
且它的“增加量”(或减少量)会从你的\zhTransArmorClass{}中减去。
例如,一件 +1 \zhTransChainMail{}相比一般的\zhTransChainMail{}会给你更好的保护,
使你的\zhTransArmorClass{}进一步减少一个单位,达到 4。
当你装上一件护甲时,你会立刻知道\zhTransArmorClass{}和它提供的任何“增加值”。
\zhTransCursed{}护甲通常有负的\zhTransEnchantments(减少值),并且不能被移除。

%.pg
%Many types of armor are subject to some kind of damage like rust.  Such
%damage can be repaired.  Some types of armor may inhibit spell casting.
许多类型的护甲会遭受某些种类的损害,如生锈。
这样的损害是可以被修复的。
一些类型的护甲可能会抑制咒语的施放(spell casting)。

%.pg
%The commands to use armor are `{\tt W}' (wear) and `{\tt T}' (take off).
%The `{\tt A}' command can also be used to take off armor as well as other
%worn items.
使用护甲的命令有“{\tt W}”(穿上,wear)和“{\tt T}”(脱下,take off)。
“{\tt A}”命令同样可以用来脱下护甲以及其他穿戴的物品。

%.hn 2
%\subsection*{Food (`{\tt \%}')}
\subsection*{食物(Food,“{\tt \%}')}

%.pg
%Food is necessary to survive.  If you go too long without eating you
%will faint, and eventually die of starvation.
%Some types of food will spoil, and become unhealthy to eat,
%if not protected.
%Food stored in ice boxes or tins (``cans'')
%will usually stay fresh, but ice boxes are heavy, and tins
%take a while to open.
食物是生存所必需的。
如果长时间没有进食你会昏迷(faint),并最终会饿死(die of starvation)。
一些种类的食物如果没有受保护会变质,变得如果食用将不利于健康。
储存在\zhTransIceBox(ice box)或\zhTransTin(tin)中的食物通常会保持新鲜,
不过\zhTransIceBox{}是很重的,而\zhTransTin{}得花费时间来打开。

%.pg
%When you kill monsters, they usually leave corpses which are also
%``food.''  Many, but not all, of these are edible; some also give you
%special powers when you eat them.  A good rule of thumb is ``you are
%what you eat.''
当你杀死怪物时,它们通常会留下尸体(corpse),这同样是“食物”。
这些中的许多,不过并非全部,是可以吃的;
一些在你吃它们时还会给予你特殊能力。
一个好的经验是“你吃什么你就会成为什么。”(you are what you eat.)

%.pg
%Some character roles and some monsters are vegetarian.  Vegetarian monsters
%will typically never eat animal corpses, while vegetarian players can,
%but with some rather unpleasant side-effects.
一些角色职业和一些怪物是素食者(vegetarian)。
素食怪物典型地从不吃动物的尸体,而素食玩家可以吃,
不过会有一些非常不愉快的副作用。

%.pg
%You can name one food item after something you like to eat with the
%{\it fruit\/} option.
你可以使用{\it fruit\/}(水果)选项来以你喜欢吃的东西来命名一种食物。

%.pg
%The command to eat food is `{\tt e}'.
吃食物的命令为“{\tt e}”。

%.hn 2
%\subsection*{Scrolls (`{\tt ?}')}
\subsection*{卷轴(Scrolls,“{\tt ?}”)}

%.pg
%Scrolls are labeled with various titles, probably chosen by ancient wizards
%for their amusement value (ex.\ ``READ ME,'' or ``THANX MAUD'' backwards).
%Scrolls disappear after you read them (except for blank ones, without
%magic spells on them).
卷轴用各种各样的标题进行了标记,很可能由远古的\zhTransWizards{}们选择,
以用于他们消遣(例如“READ ME”(读我)或
倒着写的“THANX MAUD”(感谢\zhTransMAUD))。
在你阅读完之后卷轴就会消失(除非是没有魔法咒语在上面的空白(blank)卷轴)。

%.pg
%One of the most useful of these is the %
%{\it scroll of identify}, which
%can be used to determine what another object is, whether it is cursed or
%blessed, and how many uses it has left.  Some objects of subtle
%enchantment are difficult to identify without these.
最有用处的卷轴之一是{\it \zhTransScrollOfIdentify}(scroll of identify),
它可以用来鉴定其他物品是什么、是\zhTransCursed{}还是\zhTransBlessed,
以及还有多少次可以使用。
一些有着难以察觉的\zhTransEnchantments{}(enchantment)的物品如果没有这种卷轴将会是难以鉴定的。

%.pg
%A mail daemon may run up and deliver mail to you as a %
%{\it scroll of mail} (on versions compiled with this feature).
%To use this feature on versions where {\it NetHack\/}
%mail delivery is triggered by electronic mail appearing in your system mailbox,
%you must let {\it NetHack\/} know where to look for new mail by setting the
%``MAIL'' environment variable to the file name of your mailbox.
%You may also want to set the ``MAILREADER'' environment variable to the
%file name of your favorite reader, so {\it NetHack\/} can shell to it when you
%read the scroll.
%On versions of {\it NetHack\/} where mail is randomly
%generated internal to the game, these environment variables are ignored.
%You can disable the mail daemon by turning off the
%{\it mail\/} option.
一个\zhTransMailDaemon(mail daemon)可能会跑进来,
并给你送达一封{\it \zhTransScrollOfMail}
(scroll of mail)(在那些编译了该特性的版本上)。
为了在那些当电子邮件出现在你的系统邮箱时{\it NetHack\/}会被触发以投递邮件的
版本上使用该特性,你必须通过设置“MAIL”(邮件)环境变量为你邮箱的文件名,
以便让{\it NetHack\/}知道去何处找新邮件。
你可能还希望设置“MAILREADER”(邮件阅读器)环境变量为你喜爱的阅读器的文件名,
从而在你阅读该卷轴时{\it NetHack\/}会脱离到该程序。
在那些邮件在内部被随机生成并被投递到游戏中的版本,这些环境变量是被忽略的。
你可以通过关闭{\it mail\/}选项来禁用\zhTransMailDaemon。

%.pg
%The command to read a scroll is `{\tt r}'.
阅读卷轴的命令为“{\tt r}”。

%.hn 2
%\subsection*{Potions (`{\tt !}')}
\subsection*{药水(Potions,“{\tt !}”)}

%.pg
%Potions are distinguished by the color of the liquid inside the flask.
%They disappear after you quaff them.
药水通过在烧瓶中液体的颜色进行区分。当你喝下后它们会消失。

%.pg
%Clear potions are potions of water.  Sometimes these are
%blessed or cursed, resulting in holy or unholy water.  Holy water is
%the bane of the undead, so potions of holy water are good things to
%throw (`{\tt t}') at them.  It is also sometimes very useful to dip
%(``{\tt \#dip}'') an object into a potion.
透明的药水(clear potion)是水(potion of water)。
有时它们是\zhTransBlessed{}或者\zhTransCursed,
成为\zhTransHolyWater(holy water)和\zhTransUnholyWater(unholy water)。
\zhTransHolyWater{}是\zhTransUndead{}的灾难,
所以\zhTransHolyWater{}是适合扔(throw,“{\tt t}”)向它们的东西。
有时将一个物品浸泡(“{\tt \#dip}”)入一瓶药水中也是十分有用的。

%.pg
%The command to drink a potion is `{\tt q}' (quaff).
喝一瓶药水的命令为“{\tt q}”(狂饮,quaff)。

%.hn 2
%\subsection*{Wands (`{\tt /}')}
\subsection*{魔杖(Wands,“{\tt /}”)}

%.pg
%Magic wands usually have multiple magical charges.  Some wands are
%directional---you must give a direction in which to zap them.  You can also
%zap them at yourself (just give a `{\tt .}' or `{\tt s}' for the direction).
%Be warned, however, for this is often unwise.  Other wands are
%nondirectional---they don't require a direction.  The number of charges in a
%wand is random and decreases by one whenever you use it.
魔杖通常有多次魔法\zhTransCharge(charge)。
一些魔杖是方向性的——你必须给出一个挥动它们的方向。
你也可以将它们指向你自己(只需给出“{\tt .}”或“{\tt s}”作为方向)。
不过要小心,这通常是不明智的。
其他魔杖则是无方向性的——它们不需要一个方向。
在一根魔杖中\zhTransCharge{}的数目是随机的,且当你使用后会减少一。

%.pg
%When the number of charges left in a wand becomes zero, attempts to use the
%wand will usually result in nothing happening.  Occasionally, however, it may
%be possible to squeeze the last few mana points from an otherwise spent wand,
%destroying it in the process.  A wand may be recharged by using suitable
%magic, but doing so runs the risk of causing it to explode.  The chance
%for such an explosion starts out very small and increases each time the
%wand is recharged.
当一根魔杖中剩余的\zhTransCharge{}数量变为零时,
试图使用这根魔杖通常结果是什么都没有发生(nothing happening)。
不过有时,有可能挤出一根已经用完的魔杖中的最后少量\zhTransMana(mana),
并在这一过程中毁掉这根魔杖。
一根魔杖可以使用适当的魔法重新充能(charge),不过这样做有使它爆炸的危险。
这种爆炸的可能性开始时很小,会随着每一次被重新充能而逐渐增加。

%.pg
%In a truly desperate situation, when your back is up against the wall, you
%might decide to go for broke and break your wand.  This is not for the faint
%of heart.  Doing so will almost certainly cause a catastrophic release of
%magical energies.
在真正危急的处境中,此时你的背已经靠在墙上了,你可能会决定不顾一切地冒险,
选择折断你的魔杖。
这不是给怛小的人尝试的。
这样做几乎无疑地会造成魔法能量的灾害性释放。

%.pg
%When you have fully identified a particular wand, inventory display will
%include additional information in parentheses: the number of times it has
%been recharged followed by a colon and then by its current number of charges.
%A current charge count of {\tt -1} is a special case indicating that the wand
%has been cancelled.
当你完全鉴定了一根魔杖时,\zhTransInventory{}显示会在括号中包括附加信息:
它已经被重新充能的次数,后面跟随着一个冒号({\tt :}),然后是它目前的能量次数。
一个 {\tt -1} 的目前能量计数是特殊的情况,
表明该魔杖已经\zhTransCancel(cancel)了。

%.pg
%The command to use a wand is `{\tt z}' (zap).  To break one, use the `{\tt a}'
%(apply) command.
使用一根魔杖的命令为“{\tt z}”(挥动,zap)。
要折断它的话,使用“{\tt a}”(使用,apply)命令。

%.hn 2
%\subsection*{Rings (`{\tt =}')}
\subsection*{戒指(Rings,“{\tt =}”)}

%.pg
%Rings are very useful items, since they are relatively permanent
%magic, unlike the usually fleeting effects of potions, scrolls, and
%wands.
戒指是非常有用的物品,因为它们有着相对永久的魔法效果,
不像药水、卷轴和魔杖通常短暂的效果。

%.pg
%Putting on a ring activates its magic.  You can wear only two
%rings, one on each ring finger.
戴上戒指会激活它的魔法效果。你只可以戴两枚戒指,每一根无名指上一枚。

%.pg
%Most rings also cause you to grow hungry more rapidly, the rate
%varying with the type of ring.
大多数戒指还会使你更快地变得饥饿,变快的速度随着戒指的种类而变。

%.pg
%The commands to use rings are `{\tt P}' (put on) and `{\tt R}' (remove).
使用戒指的命令为“{\tt P}”(戴上,put on)和“{\tt R}”(摘下,remove)。

%.hn 2
%\subsection*{Spellbooks (`{\tt +}')}
\subsection*{咒语书(Spellbooks,“{\tt +}”)}

%.pg
%Spellbooks are tomes of mighty magic.  When studied with the `{\tt r}' (read)
%command, they transfer to the reader the knowledge of a spell (and
%therefore eventually become unreadable) --- unless the attempt backfires.
%Reading a cursed spellbook or one with mystic runes beyond
%your ken can be harmful to your health!
咒语书是有着强大魔法的大部头。
当用“{\tt r}”(阅读,read)命令进行学习时,它们会将一个咒语(spell)的知识转移给阅读者
(因而最终会变得无法阅读)——除非这一尝试没有成功。
阅读一本\zhTransCursed{}咒语书或一本有着超出你知识范围的铭文的咒语书
会对你的健康造成伤害!

%.pg
%A spell (even when learned) can also backfire when you cast it.  If you
%attempt to cast a spell well above your experience level, or if you have
%little skill with the appropriate spell type, or cast it at
%a time when your luck is particularly bad, you can end up wasting both the
%energy and the time required in casting.
当你施放一个咒语(甚至已经学习过的)时,同样可能适得其反。
如果你试图施放一个远高于你的\zhTransExperienceLevel{}的咒语,
或者你的相应咒语类型只有少量技能,或者在你的\zhTransLuck{}非常低时施放,
你可能会浪费掉施放所要求的能量和时间。

%.pg
%Casting a spell calls forth magical energies and focuses them with
%your naked mind.  Some of the magical energy released comes from within
%you, and casting several spells in a row may tire you.
%Casting of spells also requires practice.  With practice, your
%skill in each category of spell casting will improve.  Over time, however,
%your memory of each spell will dim, and you will need to relearn it.
% FIXME naked mind 
念一个咒语会产生魔法能量,并且会用你的意念将它聚集。
一些魔法能量的释放来自你的内部,施放一连串的咒语可能会使你疲备。
咒语的施放同样需要练习。
通过练习,你的各个咒语施放类型的技能会提高。
不过经过一段时间,你对咒语的记忆将会变淡,因而你需要重新学习它。

%.pg
%Some spells are
%directional---you must give a direction in which to cast them.  You can also
%cast them at yourself (just give a `{\tt .}' or `{\tt s}' for the direction).
%Be warned, however, for this is often unwise.  Other spells are
%nondirectional---they don't require a direction.
一些咒语是方向性的——你必须给出施放它们的方向。
你同样可以向你自己施放(只需给出“{\tt .}”或“{\tt s}”作为方向)。
不过小心,这通常是不明智的。
另一些咒语是无方向性的——它们不需要一个方向。

%.pg
%Just as weapons are divided into groups in which a character can become
%proficient (to varying degrees), spells are similarly grouped.
%Successfully casting a spell exercises the skill group; sufficient skill
%may increase the potency of the spell and reduce the risk of spell failure.
%Skill slots are shared with weapons skills.  (See also the section on
%``Weapon proficiency''.)
正如武器被分为不同分类,角色可以在每个分类里变得熟练(达到不同的程度),
咒语也被类似地分了类。
成功地施放一次咒语会煅练这一技能分类;
足够的技能可以增加该咒语的威力,并减少咒语失败的危险。
咒语的技能槽与武器技能一起分享。(另见“武器熟练度”一节。)

%.pg
%Casting a spell also requires flexible movement, and wearing various types
%of armor may interfere with that.
咒语的施放还需要灵活的移动,穿戴许多种类的护甲会妨碍这一点。

%.pg
%The command to read a spellbook is the same as for scrolls, `{\tt r}'
%(read).  The `{\tt +}' command lists your current spells, their levels,
%categories, and chances for failure.
%The `{\tt Z}' (cast) command casts a spell.
阅读一本咒语书的命令与阅读卷轴的相同,“{\tt r}”(阅读,read)。
“{\tt +}”命令列出你目前的咒语,以及它们的等级、类别和施放失败率。
“{\tt Z}”(施放,cast)命令施放一个咒语。

%.hn 2
%\subsection*{Tools (`{\tt (}')}
\subsection*{工具(Tools,“{\tt (}”)}

%.pg
%Tools are miscellaneous objects with various purposes.  Some tools
%have a limited number of uses, akin to wand charges.  For example, lamps burn
%out after a while.  Other tools are containers, which objects can
%be placed into or taken out of.
工具是繁杂的有着各种各样用途的物品。
一些工具有使用次数的限制,与魔法的能量数相近。
例如,经过一段时间后灯(lamp)会烧尽。
其他一些工具是容器,物品可以放入或从里面取出。

%.pg
%The command to use tools is `{\tt a}' (apply).
使用工具的命令为“{\tt a}”(使用,apply)。

%.hn 3
%\subsection*{Containers}
\subsection*{容器(Containers)}

%.pg
%You may encounter bags, boxes, and chests in your travels.  A tool of
%this sort can be opened with the ``{\tt \#loot}'' extended command when
%you are standing on top of it (that is, on the same floor spot),
%or with the `{\tt a}' (apply) command when you are carrying it.  However,
%chests are often locked, and are in any case unwieldy objects.
%You must set one down before unlocking it by
%using a key or lock-picking tool with the `{\tt a}' (apply) command,
%by kicking it with the `{\tt \^{}D}' command,
%or by using a weapon to force the lock with the ``{\tt \#force}''
%extended command.
你可能会在你的旅途中遇到包(bag)、盒子(box)和箱子(chest)。
当你站在这种类型的工具上方时(即在地面上相同的地点),
你可以用“{\tt \#loot}”(掠夺)扩展命令来打开它,
或者在你携带着它时,用“{\tt a}”(使用,apply)命令来打开它。
不过箱子经常是锁上的,且它们无论如何都是笨重的物品。
给它们开锁之前你必须放下它们。你可以用“{\tt a}”(使用,apply)命令来使用
钥匙(key)或者开锁工具(lock-picking tool)来开锁,
或者用“{\tt \^{}D}”命令来把它踢开,
或者用“{\tt \#force}”(用力打开)扩展命令来强行撬开锁。

%.pg
%Some chests are trapped, causing nasty things to happen when you
%unlock or open them.  You can check for and try to deactivate traps
%with the ``{\tt \#untrap}'' extended command.
一些箱子被设置了机关,导致在你给它们开锁或打开它们时会有不愉快的事发生。
你可以用扩展命令“{\tt \#untrap}”(解除陷阱)来检查以及试着解除里面的机关。

%.hn 2
%\subsection*{Amulets (`{\tt "}')}
\subsection*{护身符(Amulets,“{\tt "}”)}

%.pg
%Amulets are very similar to rings, and often more powerful.  Like
%rings, amulets have various magical properties, some beneficial,
%some harmful, which are activated by putting them on.
护身符与戒指非常相似,且经常更加强大。
与戒指一样,护身符有各种魔法属性,一些是有益的,一些是有害的,
这些属性通过将护身符戴上而被激活。

%.pg
%Only one amulet may be worn at a time, around your neck.
一个时刻里只能在你的脖子上戴上一件护身符。

%.pg
%The commands to use amulets are the same as for rings, `{\tt P}' (put on)
%and `{\tt R}' (remove).
使用护身符的命令与使用戒指的相同,
“{\tt P}”(戴上,put on)与“{\tt R}”(移除,remove)。

%.hn 2
%\subsection*{Gems (`{\tt *}')}
\subsection*{宝石(Gems,“{\tt *}”)}

%.pg
%Some gems are valuable, and can be sold for a lot of gold.  They are also
%a far more efficient way of carrying your riches.  Valuable gems increase
%your score if you bring them with you when you exit.
一些宝石是昂贵的,可以出售以获得大量黄金。
它们还是更为有效地携带你的财产的方法。
如果你离开时带着昂贵的宝石,它们会增加你的分数。

%.pg
%Other small rocks are also categorized as gems, but they are much less
%valuable.  All rocks, however, can be used as projectile weapons (if you
%have a sling).  In the most desperate of cases, you can still throw them
%by hand.
其他一些小石头同样被分类为宝石,不过它们是非常不值钱的。
然而,所有石头都可以用作投射武器(如果你有\zhTransSling{}(sling)的话)。
在十分危急的时刻,你还可以用手来扔它们。

%.hn 2
%\subsection*{Large rocks (`{\tt `}')}
\subsection*{巨大的石头(“{\tt `}”)}
%.pg
%Statues and boulders are not particularly useful, and are generally
%heavy.  It is rumored that some statues are not what they seem.
雕像(statue)和\zhTransBoulders(boulder)不是特别有用,并且通常很重。
有传闻称一些雕像并非它们看上去的那样。

%.pg
%Very large humanoids (giants and their ilk) have been known to use boulders
%as weapons.
非常巨大的\zhTransHumanoid(very large humanoid,即\zhTransGiant(giant)和它们的亲属)
以它们用\zhTransBoulders{}作为武器而闻名。

%.hn 2
%\subsection*{Gold (`{\tt \$}')}
\subsection*{黄金(Gold,“{\tt \$}”)}

%.pg
%Gold adds to your score, and you can buy things in shops with it.
%There are a number
%of monsters in the dungeon that may be influenced by the amount of gold
%you are carrying (shopkeepers aside).
黄金会加入到你的分数中,你可以用它们在\zhTransShop{}(shop)里购买东西。
\zhTransDungeon{}中许多怪物会受你所携带的黄金数量所影响
(\zhTransShopkeeper{}(shopkeeper)包括在内)。

%.hn 1
%\section{Conduct}
\section{\zhTransConduct}

%.pg
%As if winning {\it NetHack\/} were not difficult enough, certain players
%seek to challenge themselves by imposing restrictions on the
%way they play the game.  The game automatically tracks some of
%these challenges, which can be checked at any time with the {\tt \#conduct}
%command or at the end of the game.  When you perform an action which
%breaks a challenge, it will no longer be listed.  This gives
%players extra ``bragging rights'' for winning the game with these
%challenges.  Note that it is perfectly acceptable to win the game
%without resorting to these restrictions and that it is unusual for
%players to adhere to challenges the first time they win the game.
好像在{\it NetHack\/}中获胜不够困难一样,一些玩家通过在他们玩游戏的方式上
施加限制来寻求挑战他们自己。
游戏会自动地跟踪一些这类挑战,
这可以在任何时候通过{\tt \#conduct}(\zhTransConduct)命令
或在游戏结束时进行查看。
当你做了一个违反一个挑战的行为时,该挑战将不再被列出。
带着这些挑战赢得游戏给了玩家额外的“吹牛权利”。
注意,没有遵守这些限制而赢得游戏是完全可以接受的,
并且玩家第一次赢得游戏时就带有这些挑战是不常见的。

%.pg
%Several of the challenges are related to eating behavior.  The most
%difficult of these is the foodless challenge.  Although creatures
%can survive long periods of time without food, there is a physiological
%need for water; thus there is no restriction on drinking beverages,
%even if they provide some minor food benefits.
%Calling upon your god for help with starvation does
%not violate any food challenges either.
数个挑战与吃的行为相关。
其中最困难的是绝食(foodless)挑战。
虽然生物可以在没有食物的情况下仍生存很长一段时间,
依然存在着对水的生理上的需求;所有没有在喝饮料上的限制,
即使它们提供一些很小的食物方面的好处。
在饥饿时向你的神祈祷以获得帮助同样不会违反任何食物挑战。

%.pg
%A strict vegan diet is one which avoids any food derived from animals.
%The primary source of nutrition is fruits and vegetables.  The
%corpses and tins of blobs (`b'), jellies (`j'), and fungi (`F') are
%also considered to be vegetable matter.  Certain human
%food is prepared without animal products; namely, lembas wafers, cram
%rations, food rations (gunyoki), K-rations, and C-rations.
%Metal or another normally indigestible material eaten while polymorphed
%into a creature that can digest it is also considered vegan food.
%Note however that eating such items still counts against foodless conduct.
一份严格的素食(strict vegan)饮食会避开任何源自动物的食物。
营养的首要来源是水果和蔬菜。
\zhTransBlobs(blobs,“{\tt b}”)、\zhTransJellies(jellies,“{\tt j}”)
和\zhTransFungi(fungi,“{\tt F}”)的尸体和\zhTransTin{}同样被认为是素食。
一些人类食物是未使用动物产品而制作成的;
即\zhTransLembasWafers(lembas wafers)、\zhTransCramRation(cram ration)、
\zhTransFoodRation(food ration,gunyoki)、\zhTransKRation(K-ration)
和\zhTransCRation(C-ration)。
金属或者其他通常不能吃的材料在\zhTransPolymorph{}为一种可以消化它的生物时
仍然被视作严格素食。
不过注意,吃这类东西仍被认为是违反绝食\zhTransConduct{}的。

%.pg
%Vegetarians do not eat animals;
%however, they are less selective about eating animal byproducts than vegans.
%In addition to the vegan items listed above, they may eat any kind
%of pudding (`P') other than the black puddings,
%eggs and food made from eggs (fortune cookies and pancakes),
%food made with milk (cream pies and candy bars), and lumps of
%royal jelly.  Monks are expected to observe a vegetarian diet.
素食者(Vegetarian)不食用动物;
不过,他们与严格素食者相比在食用动物副产品方向没有那么挑剔。
除了上述列出的严格素食的物品以外,
他们还会吃除了\zhTransBlackPudding(black pudding)以外的
任何种类的布丁(pudding,“{\tt P}”)、蛋(egg)和由蛋制成的食物
(\zhTransFortuneCookie(fortune cookie)和\zhTransPancake(pancake))、
由牛奶制成的食物
(\zhTransCreamPie(cream pie)和\zhTransCandyBar(candy bar))
以及\zhTransLumpOfRoyalJelly(lump of royal jelly)。
\zhTransMonks{}(Monk)预期是遵守素食饮食的。

%.pg
%Eating any kind of meat violates the vegetarian, vegan, and foodless
%conducts.  This includes tripe rations, the corpses or tins of any
%monsters not mentioned above, and the various other chunks of meat
%found in the dungeon.  Swallowing and digesting a monster while polymorphed
%is treated as if you ate the creature's corpse.
%Eating leather, dragon hide, or bone items while
%polymorphed into a creature that can digest it, or eating monster brains
%while polymorphed into a mind flayer, is considered eating
%an animal, although wax is only an animal byproduct.
吃任何种类的肉会违反素食、严格素食和绝食\zhTransConduct。
这其中包括\zhTransTripeRation(tripe ration)、任何上面没有提及的怪物的
尸体或\zhTransTin,以及其他在\zhTransDungeon{}中发现的肉块(chunk of meat)。
当\zhTransPolymorph{}时吞下和消化一个怪物被视作你吃了该怪物的尸体。
在\zhTransPolymorph{}为一种有相应消化能力的怪物时,
吃皮革(leather)、龙皮(dragon hide)或者骨质(bone)物品,
或者在\zhTransPolymorph{}为\zhTransMindFlayer(mind flayer)时
吃下怪物的脑(brain),会被视为吃动物,尽管蜡只是动物副产品而已。

%.pg
%Regardless of conduct, there will be some items which are indigestible,
%and others which are hazardous to eat.  Using a swallow-and-digest
%attack against a monster is equivalent to eating the monster's corpse.
%Please note that the term ``vegan'' is used here only in the context of
%diet.  You are still free to choose not to use or wear items derived
%from animals (e.g. leather, dragon hide, bone, horns, coral), but the
%game will not keep track of this for you.  Also note that ``milky''
%potions may be a translucent white, but they do not contain milk,
%so they are compatible with a vegan diet.  Slime molds or
%player-defined ``fruits'', although they could be anything
%from ``cherries'' to ``pork chops'', are also assumed to be vegan.
即使不考虑\zhTransConduct,仍有一些物品是不能吃的,
另一些如果食用将是危险的。
对一个怪物使用吞下并消化攻击等同于吃该怪物的尸体。
请注意,“严格素食”(vegan)一词在这里仅用在饮食语境中。
你也可以自由地选择不使用或不穿戴源自动物的物品
(例如皮革(leather)、龙皮(dragon hide)、骨(bone)、角(horn)、珊瑚(coral)),
不过游戏不会为你跟踪这些做法。
同样注意“乳状”(milky)药水可能是白色半透明状的,
不过它们不含有牛奶,所以它们是与严格素食相容的。
\zhTransSlimeMold(slime mold)或者玩家定义的“水果”(fruit),
尽管它们可以是从“樱桃”到“猪排骨”的任何东西,仍被假定是严格素食的。

%.pg
%An atheist is one who rejects religion.  This means that you cannot
%{\tt \#pray}, {\tt \#offer} sacrifices to any god,
%{\tt \#turn} undead, or {\tt \#chat} with a priest.
%Particularly selective readers may argue that playing Monk or Priest
%characters should violate this conduct; that is a choice left to the
%player.  Offering the Amulet of Yendor to your god is necessary to
%win the game and is not counted against this conduct.  You are also
%not penalized for being spoken to by an angry god, priest(ess), or
%other religious figure; a true atheist would hear the words but
%attach no special meaning to them.
无神论者(atheist)是拒绝接受宗教信仰的人。
这意味着你不能祈祷({\tt \#pray})、向任何神献祭({\tt \#offer})牺牲、
驱赶({\tt \#turn})\zhTransUndead{}或者与祭司(priest)交谈({\tt \#chat})。
特别挑剔的读者可能会主张玩\zhTransMonks(Monk)或\zhTransPriests(Priest)
角色应该是违反该\zhTransConduct{}的;不过选择留给玩家。
向你的神献祭\zhTransAmuletOfYendor(the Amulet of Yendor)是游戏取胜所必需的,
这不会被认为违反了该\zhTransConduct。
生气的神、祭司或者其他宗教人士主动向你说话同样不会使你受处罚;
一位真正的无神论者会听这些话语,但不会给它们附加特殊意义。

%.pg
%Most players fight with a wielded weapon (or tool intended to be
%wielded as a weapon).  Another challenge is to win the game without
%using such a wielded weapon.  You are still permitted to throw,
%fire, and kick weapons; use a wand, spell, or other type of item;
%or fight with your hands and feet.
大多数玩家使用握持的武器(或者设计作为武器使用的工具)进行战斗。
另一个挑战是不使用这样握持的武器而赢得游戏。
你仍然被允许扔、发射和踢武器;使用魔杖、咒语或者其他类型的物品;
或者用你的手和脚来战斗。

%.pg
%In {\it NetHack\/}, a pacifist refuses to cause the death of any other monster
%(i.e. if you would get experience for the death).  This is a particularly
%difficult challenge, although it is still possible to gain experience
%by other means.
一位反战主义者(pacifist)在{\it NetHack\/}中拒绝造成任何其他怪物的死亡
(即如果你会从该死亡中获得经验)。
这是一个特别难的挑战,不过依然有可能通过其他方法获得经验。

%.pg
%An illiterate character cannot read or write.  This includes reading
%a scroll, spellbook, fortune cookie message, or t-shirt; writing a
%scroll; or making an engraving of anything other than a single ``x'' (the
%traditional signature of an illiterate person).  Reading an engraving,
%or any item that is absolutely necessary to win the game, is not counted
%against this conduct.  The identity of scrolls and spellbooks (and
%knowledge of spells) in your starting inventory is assumed to be
%learned from your teachers prior to the start of the game and isn't
%counted.
一位文盲(illiterate)角色不能阅读或者书写。
这包括阅读卷轴、咒语书、幸运饼干的消息或者 T 恤(t-shirt);
书写卷轴;
或者刻写除了“x”(一位文盲的传统签名)以外的文字。
阅读刻写的文字,或者任何赢得游戏绝对需要的物品,不会视作对该\zhTransConduct
的违反。
在你初始\zhTransInventory{}中的卷轴和咒语书的标识(以及咒语的知识)被认为是
在游戏开始前从你的老师那里学来的,因而不被计算在内。

%.pg
%There are several other challenges tracked by the game.  It is possible
%to eliminate one or more species of monsters by genocide; playing without
%this feature is considered a challenge.  When the game offers you an
%opportunity to genocide monsters, you may respond with the monster type
%``none'' if you want to decline.  You can change the form of an item into
%another item of the same type (``polypiling'') or the form of your own
%body into another creature (``polyself'') by wand, spell, or potion of
%polymorph; avoiding these effects are each considered challenges.
%Polymorphing monsters, including pets, does not break either of these
%challenges.
%Finally, you may sometimes receive wishes; a game without an attempt to
%wish for any items is a challenge, as is a game without wishing for
%an artifact (even if the artifact immediately disappears).  When the
%game offers you an opportunity to make a wish for an item, you may
%choose ``nothing'' if you want to decline.
还有其他数种挑战被游戏跟踪。
有可能通过\zhTransGenocide(genocide)的方式消除一种或多种怪物;
进行游戏时没有这一特性被认为是一种挑战。
当游戏提供一个机会给你灭绝怪物时,如果你想要拒绝的话,
你可以用怪物种类“none”(没有的)作为回应。
通过魔杖、咒语或\zhTransPotionOfPolymorph(potion of polymorph),
你可以将一个物品的外形改变为相同类型的另一种物品(“成堆变形”,polypiling),
或者改变自己身体的形式成另一种生物(“自我变形”,polyself);
避免这些效果分别被认为是一种挑战。
将怪物\zhTransPolymorph,包括\zhTransPets,不会破坏这些挑战中的任意一种。
最后,你有时可能得到许愿(wish);
一次没有尝试许愿任何物品的游戏是一个挑战,
没有许愿\zhTransArtifact(artifact)(甚至该\zhTransArtifact{}立刻消失)
的游戏也同样如此。
当游戏向你提供许愿一个物品的机会时,如果你想要拒绝的话,
可以选择“nothing”(没有的)。

%.hn 1
%\section{Options}
\section{选项}

%.pg
%Due to variations in personal tastes and conceptions of how {\it NetHack\/}
%should do things, there are options you can set to change how {\it NetHack\/}
%behaves.
由于个人偏好的不同,以及在关于{\it NetHack\/}应该如何做一件事观念上的差异,
有多种你可以设置以改变{\it NetHack\/}如何运转的选项(option)。

%.hn 2
%\subsection*{Setting the options}
\subsection*{设置选项}

%.pg
%Options may be set in a number of ways.  Within the game, the `{\tt O}'
%command allows you to view all options and change most of them.
%You can also set options automatically by placing them in the
%``NETHACKOPTIONS'' environment variable or in a configuration file.
%Some versions of {\it NetHack\/} also have front-end programs that allow
%you to set options before starting the game.
选项可以通过多种方式设置。
在游戏中,“{\tt O}”命令允许你查看所有选项,并且可以更改其中大部分。
你还可以将选项放入“NETHACKOPTIONS”环境变量或者配置文件中,以自动进行设置。
一些版本的{\it NetHack\/}还有允许你在开始游戏前设置选项的前端程序。

%.hn 2
%\subsection*{Using the NETHACKOPTIONS environment variable}
\subsection*{使用 NETHACKOPTIONS 环境变量}

%.pg
%The NETHACKOPTIONS variable is a comma-separated list of initial
%values for the various options.  Some can only be turned on or off.
%You turn one of these on by adding the name of the option to the list,
%and turn it off by typing a `{\tt !}' or ``{\tt no}'' before the name.
%Others take a
%character string as a value.  You can set string options by typing
%the option name, a colon or equals sign, and then the value of the string.
%The value is terminated by the next comma or the end of string.
NETHACKOPTIONS 变量是一个逗号({\tt ,})分隔的列表,包含了各种选项的初始值。
一些选项只能打开或关闭。
将这些选项之一的名字加入到列表中会打开它,
在名字前加上“{\tt !}”或者“{\tt no}”(不)则会关闭它。
其他一些选项以字符串作为值。
你可以通过输入选项名,加上冒号({\tt :})或等号({\tt =}),随后接上字符串的值,
来设置一个字符串选项。
该值由下一个逗号或字符串的结束来终止。

%.pg
%For example, to set up an environment variable so that {\it autoquiver\/}
%is on, {\it autopickup\/} is off, the {\it name\/} is set to ``Blue Meanie'',
%and the {\it fruit\/} is set to ``papaya'', you would enter the command
例如,要设置一个环境变量打开{\it autoquiver\/}(自动装备箭筒)、
关闭{\it autopickup\/}(自动捡起)、{\it name\/}(名字)设置为“Blue Meanie”
(蓝色恶魔)以及{\it fruit\/}(水果)设置为“papaya”(番木瓜),
在{\it csh}中时(注意需要转义{\tt !}因为它对于 shell 是特殊的),你应该输入以下命令
%.sd
\begin{verbatim}
    setenv NETHACKOPTIONS "autoquiver,\!autopickup,name:Blue Meanie,fruit:papaya"
\end{verbatim}
%.ed

%\nd in {\it csh}
%(note the need to escape the ! since it's special to the shell), or
\nd 或者在{\it sh\/}或{\it ksh}中时,输入
%.sd
\begin{verbatim}
    NETHACKOPTIONS="autoquiver,!autopickup,name:Blue Meanie,fruit:papaya"
    export NETHACKOPTIONS
\end{verbatim}
%.ed

%\nd in {\it sh\/} or {\it ksh}.

%.hn 2
%\subsection*{Using a configuration file}
\subsection*{使用配置文件}

%.pg
%Any line in the configuration file starting with `{\tt \#}' is treated as a comment.
%Any line in the configuration file starting with ``{\tt OPTIONS=}'' may be
%filled out with options in the same syntax as in NETHACKOPTIONS.
%Any line starting with ``{\tt DUNGEON=}'', ``{\tt EFFECTS=}'',
%``{\tt MONSTERS=}'', ``{\tt OBJECTS=}'', ``{\tt TRAPS=}'', 
%or ``{\tt BOULDER=}''
%is taken as defining the corresponding {\it dungeon},
%{\it effects}, {\it monsters}, {\it objects}, {\it traps\/} or
%{\it boulder\/} option in a different syntax,
%a sequence of decimal numbers giving the character position
%in the current font to be used in displaying each entry.
%A zero in any entry in such a sequence leaves the display of that
%entry unchanged; this feature is not available using the option syntax.
%Such a sequence can be continued to multiple lines by putting a
%`{\tt \verb+\+}' at the end of each line to be continued.
配置文件中任何以“{\tt \#}”开始的行将被视为注释。
配置文件中以“{\tt OPTIONS=}”开始的行可用与 NETHACKOPTIONS 中相同语法的
选项进行填写。
配置文件中任何以“{\tt DUNGEON=}”、“{\tt EFFECTS=}”、“{\tt MONSTERS=}”、
“{\tt OBJECTS=}”、“{\tt TRAPS=}”或“{\tt BOULDER=}”开始的行被视为以一种
不同的语法来设置相应的{\it dungeon}(\zhTransDungeon{})、{\it effects}(效果)、
{\it monsters}(怪物)、{\it objects}(物品)、{\it traps\/}(陷阱)
或{\it boulder\/}(\zhTransBoulders)选项,该语法为一个十进制的数字序列,
每个数字给出在当前字体中用于显示对应项目的字符的位置。
在这样的一个序列中出现的一个零将不会更变相应项目的显示;
这一特性对于使用选项语法是不可用的。
这样的一个序列可以延续成多行,只需在想要延续的每一行的
末尾加上一个“{\tt \verb+\+}”。

%.pg
%If your copy of the game included the compile time AUTOPICKUP\_EXCEPTIONS 
%option, then any line starting with ``{\tt AUTOPICKUP\_EXCEPTION=}'' 
%is taken as defining an exception to the ``{\tt pickup\_types}'' option.
%There is a section of this Guidebook that discusses that.
如果你的游戏拷贝包含了编译时选项 AUTOPICKUP\_EXCEPTIONS(自动捡起例外),
那么任何以“{\tt AUTOPICKUP\_EXCEPTION=}”开头的行将视为是给
“{\it pickup\_types}”(捡起类型)设置了一个例外。
在该指南中有一节会讨论它。

%.pg
%The default name of the configuration file varies on different
%operating systems, but NETHACKOPTIONS can also be set to
%the full name of a file you want to use (possibly preceded by an `{\tt @}').
配置文件的默认名随着操作系统的不同而不同,不过
NETHACKOPTIONS 也可以设置为你想要使用的配置文件的
完整文件名(也许要在之前加上一个“{\tt @}”)。

%.hn 2
%\subsection*{Customization options}
\subsection*{定制选项}

%.pg
%Here are explanations of what the various options do.
%Character strings that are too long may be truncated.
%Some of the options listed may be inactive in your dungeon.
这里是各种选项完成什么的解释。
太长的字符串可能会被截断。
一些列举出的选项可能在你的\zhTransDungeon{}中是非激活的。

\blist{}
%.lp
\item[\ib{align}]
%Your starting alignment ({\tt align:lawful}, {\tt align:neutral},
%or {\tt align:chaotic}).  You may specify just the first letter.
%The default is to randomly pick an appropriate alignment.
%Cannot be set with the `{\tt O}' command.
(\zhTransAlignment\footnote{该括号内的内容为该选项中的单词的含义,下同——翻译。})
你的初始\zhTransAlignment({\tt align:lawful}(\zhTransAlignment:\zhTransLawful)、
{\tt align:neutral}(\zhTransAlignment:\zhTransNeutral)或者
{\tt align:chaotic}(\zhTransAlignment:\zhTransChaotic))。你可以仅指定第一个字母。
默认会自动随机选择一个合适的\zhTransAlignment。
不能使用“{\tt O}”命令进行设置。
%.lp
\item[\ib{autodig}]
%Automatically dig if you are wielding a digging tool and moving into a place
%that can be dug (default false).
(自动挖掘)
如果你握持了一个挖掘工具,当移动到一个可以被挖的地方时会自动挖掘
(默认关闭)。
%.lp
\item[\ib{autopickup}]
%Automatically pick up things onto which you move (default on).
%See ``{\it pickup\_types\/}'' to refine the behavior.
(自动捡起)自动捡起你移动到的地方上的物品(默认打开)。
见“{\it pickup\_types\/}”以精确定制它的行为。
%.lp
\item[\ib{autoquiver}]
%This option controls what happens when you attempt the `f' (fire)
%command with an empty quiver.  When true, the computer will fill
%your quiver with some suitable weapon.  Note that it will not take
%into account the blessed/cursed status, enchantment, damage, or
%quality of the weapon; you are free to manually fill your quiver with
%the `Q' command instead.  If no weapon is found or the option is
%false, the `t' (throw) command is executed instead.  (default false)
(自动装备箭筒)该选项控制当你在箭筒空了时尝试使用“{\tt f}”(发射,fire)命令会
发生什么。当打开时,电脑会用某些合适的武器填充你的箭筒。
注意它不会考虑该武器的\zhTransBlessed/\zhTransCursed{}状态、
\zhTransEnchantments、伤害或者质量;
作为替代,你可以用“{\tt Q}”命令自由地手动填充你的箭筒。
如果没有找到武器或者该选项被关闭了,则会执行“{\tt t}”(扔,throw)命令。(默认关闭)
%.lp
\item[\ib{boulder}]
%Set the character used to display boulders (default is rock class symbol).
(\zhTransBoulders)设置用于显示\zhTransBoulders{}的字符
(默认为石头(rock)类的符号)。
%.lp
\item[\ib{catname}]
%Name your starting cat (ex.\ ``{\tt catname:Morris}'').
%Cannot be set with the `{\tt O}' command.
(猫的名字)命名你出发时携带的猫(例如“{\tt catname:Morris}”)。
不能使用“{\tt O}”命令进行设置。
%.lp character
\item[\ib{character}]
%Pick your type of character (ex.\ ``{\tt character:Monk}'');
%synonym for ``{\it role\/}''.  See ``{\it name\/}'' for an alternate method
%of specifying your role.  Normally only the first letter of
%the value is examined; the string ``{\tt random}'' is an exception.
(角色)选择你角色的种类(例如“{\tt character:Monk}(角色:\zhTransMonks)”);
是“{\it role\/}”(职业)的同义词。
见“{\it name\/}”(名字)获得指定你的职业的另一种方法。
通常只会检查该值的第一个字母;字符串“{\tt random}”(随机)是一个例外。
%.lp
\item[\ib{checkpoint}]
%Save game state after each level change, for possible recovery after
%program crash (default on).
(检查点)在每一次\zhTransDungeon{}层数改变时保存游戏状态,在程序崩溃后以用于可能的恢复
(默认打开)。
%.lp
\item[\ib{checkspace}]
%Check free disk space before writing files to disk (default on).
%You may have to turn this off if you have more than 2 GB free space
%on the partition used for your save and level files.
%Only applies when MFLOPPY was defined during compilation.
(检查空间)在文件写入硬盘前检查空余的硬盘空间(默认打开)。
你可以在用于存档和地层文件的分区有多于 2 GB 空余空间时关闭该选项。
仅在编译时定义了 MFLOPPY 时有效。
%.lp
\item[\ib{cmdassist}]
%Have the game provide some additional command assistance for new 
%players if it detects some anticipated mistakes (default on).
(命令帮助)在检测到一些预期错误时让游戏为新玩家提供一些额外命令帮助
(默认打开)。
%.lp
\item[\ib{confirm}]
%Have user confirm attacks on pets, shopkeepers, and other
%peaceable creatures (default on).
(确认)需要用户在进攻\zhTransPets、\zhTransShopkeeper
和其他无攻击性怪物时进行确认(默认打开)。
%.lp
\item[\ib{DECgraphics}]
%Use a predefined selection of characters from the DEC VT-xxx/DEC
%Rainbow/ANSI line-drawing character set to display the dungeon/effects/traps
%instead of having to define a full graphics set yourself (default off).
%This option also sets up proper handling of graphics
%characters for such terminals, so you should specify it when appropriate
%even if you override the selections with your own graphics strings.
(DEC 图形)使用预先从 DEC VT-xxx/DEC Rainbow/ANSI 画线字符集中选出的字符
来显示\zhTransDungeon{}/动画效果/陷阱,而非由你自己来定义一套完整的图形集(默认关闭)。
该选项还会为这些终端设置合适的图形字符处理方法,
所以当合适时你应该打开它,甚至当你想要用你自己的图形字符串进行替换时。

%.lp
\item[\ib{disclose}]
%Controls options for disclosing various information when the game ends (defaults
%to all possibilities being disclosed).
%The possibilities are:
(展开)控制在游戏结束时各种信息是否展开的选项(默认为展开所有可能的)。
可能的为:

%.sd
%.si
%{\tt i} --- disclose your inventory.\\
%{\tt a} --- disclose your attributes.\\
%{\tt v} --- summarize monsters that have been vanquished.\\
%{\tt g} --- list monster species that have been genocided.\\
%{\tt c} --- display your conduct.
{\tt i} --- 展开你的\zhTransInventory(inventory)。\\
{\tt a} --- 展开你的\zhTransAttributes(attribute)。\\
{\tt v} --- 总结你战胜的(vanquished)怪物。\\
{\tt g} --- 列出被\zhTransGenocide{}的(genocided)怪物种类。\\
{\tt c} --- 显示你的\zhTransConduct(conduct)。
%.ei
%.ed

%Each disclosure possibility can optionally be preceded by a prefix which
%let you refine how it behaves. Here are the valid prefixes:
每一个可能的展开项目可以在前面可选地加上一个前缀,
它可以让你控制项目如何表现。下面是有效的前缀:

%.sd
%.si
%{\tt y} --- prompt you and default to yes on the prompt.\\
%{\tt n} --- prompt you and default to no on the prompt.\\
%{\tt +} --- disclose it without prompting.\\
%{\tt -} --- do not disclose it and do not prompt.
{\tt y} --- 显示提示符,且默认设为是(yes)。\\
{\tt n} --- 显示提示符,且默认设为否(no)。\\
{\tt +} --- 不显示提示符就展开它。\\
{\tt -} --- 不展开它也不显示提示符。
%.ei
%.ed

%(ex.\ ``{\tt disclose:yi na +v -g -c}'')
%The example sets {\it inventory\/} to {\it prompt\/} and default to {\it yes\/}, 
%{\it attributes\/} to {\it prompt\/} and default to {\it no\/}, 
%{\it vanquished\/} to {\it disclose without prompting\/}, 
%{\it genocided\/} to {\it not disclose\/} and not to {\it prompt\/}, and 
%{\it conduct\/} to {\it not disclose\/} and not to {\it prompt\/}.
%Note that the vanquished monsters list includes all monsters killed by
%traps and each other as well as by you. 
(例如“{\tt disclose:yi na +v -g -c}”)
该示例设置{\it \zhTransInventory\/}为{\it 显示提示符\/}且默认为{\it 是\/}、
{\it \zhTransAttributes\/}为{\it 显示提示符\/}且默认为{\it 否\/}、
{\it 战胜的\/}为{\it 不显示提示符就展开\/}、
{\it \zhTransGenocide\/的}为{\it 不展开\/}也{\it 不显示提示符\/}和
{\it \zhTransConduct\/}为{\it 不展开\/}也{\it 不显示提示符\/}。
请注意战胜的怪物包括所有被\zhTransTraps{}和它们相互之间,以及被你杀死的怪物。
%.lp
\item[\ib{dogname}]
%Name your starting dog (ex.\ ``{\tt dogname:Fang}'').
%Cannot be set with the `{\tt O}' command.
(狗的名字)命名你出发时携带的狗(例如“{\tt dogname:Fang}”)。
不能使用“{\tt O}”命令进行设置。
%.lp
\item[\ib{dungeon}]
%Set the graphics symbols for displaying the dungeon (default
%``\verb& |--------||.-|++##& \verb&.##<><>_|\\#{}.}..## #}&'').
%The {\it dungeon\/} option should be
%followed by a string of 1--41
%characters to be used instead of the default map-drawing characters.
%The dungeon map will use the characters you specify instead of the
%default symbols, and default symbols for any you do not specify.
%Remember that you may need to escape some of these characters
%on a command line if they are special to your shell.
(\zhTransDungeon{})设置用于显示\zhTransDungeon{}的图形符号(默认为
“\verb& |--------||.-|++##.##<><>_|\\#{}& \verb&.}..## #}&”)。
{\it dungeon\/}选项后面应该跟随一个有 1--41 个字符的字符串,
它将替代默认的用来绘制地图的字符。
\zhTransDungeon{}的地图会使用你指定的字符而非默认的符号,任何你没有指定的则使用默认符号。
记住,你在命令行中可能需要转义某些字符,如果它们对于你的 shell 是特殊的的话。

%Note that {\it NetHack\/} escape-processes this option string in conventional C
%fashion.  This means that `\verb+\+' is a prefix to take the following
%character literally.  Thus `\verb+\+' needs to be represented as `\verb+\\+'.
%The special escape form
%`\verb+\m+' switches on the meta bit in the following character, and the
%`{\tt \^{}}' prefix causes the following character to be treated as a control
%character.
请注意{\it NetHack\/}会以传统 C 语言风格转义该选项字符。
这意味着“\verb+\+”是使下一个字符使用其字面含义的前缀。
所以“\verb+\+”需要表示为“\verb+\\+”。
“\verb+\m+”这一特殊的转义前缀会更变后面一个字符的 meta 比特,
“{\tt \^{}}”会造成后面一个字符被视作一个控制字符。

%The order of the symbols is:  solid rock, vertical wall, horizontal
%wall, upper left corner, upper right corner, lower left corner, lower
%right corner, cross wall, upward T wall, downward T wall, leftward T
%wall, rightward T wall, no door, vertical open door, horizontal open
%door, vertical closed door, horizontal closed door, iron bars, tree,
%floor of a room, dark corridor, lit corridor, stairs up, stairs down,
%ladder up, ladder down, altar, grave, throne, kitchen sink, fountain, pool or moat,
%ice, lava, vertical lowered drawbridge, horizontal lowered drawbridge,
%vertical raised drawbridge, horizontal raised drawbridge, air, cloud,
%under water.
符号的顺序为:\zhTransSolidRock(solid rock)、垂直的\zhTransWall(wall)、水平的\zhTransWall、左上\zhTransCorner(corner)、右上\zhTransCorner、左下\zhTransCorner、
右下\zhTransCorner、十字形的\zhTransWall、向上的丁字形\zhTransWall、向下的丁字形\zhTransWall、向左的丁字形\zhTransWall、
向右的丁字形\zhTransWall、没有门的\zhTransDoorway(doorway)、打开的垂直\zhTransDoor{}(door)、打开的水平\zhTransDoor、关上的垂直\zhTransDoor、
关上的水平\zhTransDoor、\zhTransIronBars(iron bar)、\zhTransTree(tree)、房间的\zhTransFloor(floor)、黑暗的\zhTransCorridor(corridor)、明亮的\zhTransCorridor、
向上的\zhTransStairs(stair)、向下的\zhTransStairs、向上的\zhTransLadders(ladder)、
向下的\zhTransLadders、\zhTransAltar(altar)、\zhTransGrave(grave)、\zhTransThrone(throne)、厨房\zhTransSink(sink)、
\zhTransFountain(fountain)、\zhTransPool(pool)或\zhTransMoat(moat)、\zhTransIce(ice)、\zhTransLava(lava)、放下的垂直的\zhTransDrawbridge(drawbridge)、
放下的水平的吊桥、
升起的垂直的吊桥、升起的水平的吊桥、\zhTransAir(air)、\zhTransCloud(cloud)、\zhTransUnderWater(under water)。

%You might want to use `{\tt +}' for the corners and T walls for a more
%aesthetic, boxier display.  Note that in the next release, new symbols
%may be added, or the present ones rearranged.
你可能想使用“{\tt +}”显示墙角和丁字形的墙,以获得更美观和厚实的显示。
注意在下一次发布中,可能会增加新的符号,或者目前的会被重新排列。

%Cannot be set with the `{\tt O}' command.
不能使用“{\tt O}”命令进行设置。
%.lp
\item[\ib{effects}]
%Set the graphics symbols for displaying special effects (default
%``\verb&|-\\/*!)(0#@*/-\\& \verb&||\\-//-\\| |\\-/&'').
%The {\it effects\/} option should be
%followed by a string of 1--29
%characters to be used instead of the default special-effects characters.
%This string is subjected to the same processing as the {\it dungeon\/} option.
(效果)设置用于显示特殊效果的图形符号
(默认为“\verb&|-\\/*!)(0#@*/-\\||\\-//-\\& \verb&| |\\-/&”)。
{\it effects\/}选项应跟随着一个用来替代默认特殊效果的含 1--29 个字符的字符串。
该字符串会接受与{\it dungeon\/}选项相同的处理。

%The order of the symbols is:  vertical beam, horizontal beam, left slant,
%right slant, digging beam, camera flash beam, left boomerang, right boomerang,
%four glyphs giving the sequence for magic resistance displays,
%the eight surrounding glyphs for swallowed display,
%nine glyphs for explosions.
%An explosion consists of three rows (top, middle, and bottom) of three
%characters.  The explosion is centered in the center of this $3 \times 3$
%array.
符号的顺序为:垂直的\zhTransBeam、水平的\zhTransBeam、
向左的斜线、向右的斜线、挖掘\zhTransBeam、相机闪光的\zhTransBeam、
向左的\zhTransBoomerang(boomerang)、向右的\zhTransBoomerang、
给出\zhTransMagicResistance{}(magic resistance)连续显示效果的四个符号、
用于被吞食时显示周围环境的八个符号、以及用于爆炸的九个符号。
爆炸包含三行(上、中和下),各含三个字符。
爆炸以该 $3 \times 3$ 序列的中心居中。

%Note that in the next release, new symbols may be added,
%or the present ones rearranged.
注意在下一次发布中,可能会增加新的符号,或者目前的会被重新排列。

%Cannot be set with the `{\tt O}' command.
不能使用“{\tt O}”命令进行设置。
%.lp
\item[\ib{extmenu}]
%Changes the extended commands interface to pop-up a menu of available commands.
%It is keystroke compatible with the traditional interface except that it does
%not require that you hit Enter.  It is implemented only by the tty port 
%(default off), when the game has been compiled to support tty graphics.
改变扩展命令界面为弹出一个可用命令的菜单。
该菜单是与传统界面按键兼容的,除了不需要你按下回车键({\tt Enter})。
它仅在 tty 移植上实现(默认关闭),且只在游戏编译了对 tty 图形的支持时可用。
%.lp
\item[\ib{female}]
%An obsolete synonym for ``{\tt gender:female}''.  Cannot be set with the
%`{\tt O}' command.
(女性)“{\tt gender:female}”(性别:女)的一个废弃的同义词。
不能使用“{\tt O}”命令进行设置。
%.lp
\item[\ib{fixinv}]
%An object's inventory letter sticks to it when it's dropped (default on).
%If this is off, dropping an object shifts all the remaining inventory letters.
(固定\zhTransInventory)当一个物品被放下时,它原来的\zhTransInventory
字母仍为它预留(默认打开)。如果该选项关闭了,放下一个物品会移动所有
剩余的\zhTransInventory{}字母。
%.lp
\item[\ib{fruit}]
%Name a fruit after something you enjoy eating (ex.\ ``{\tt fruit:mango}'')
%(default ``{\tt slime mold}''). Basically a nostalgic whimsy that
%{\it NetHack\/} uses from time to time.  You should set this to something you
%find more appetizing than slime mold.  Apples, oranges, pears, bananas, and
%melons already exist in {\it NetHack}, so don't use those.
(水果)用某个你喜欢吃的东西来命名水果(例如“{\tt fruit:mango}”
(水果:芒果))(默认为“{\tt slime mold}”(\zhTransSlimeMold))。
这基本上是一个{\it NetHack}不时使用的传统幽默。
你应该将它设置为某些你觉得比\zhTransSlimeMold{}更可口的东西。
Apple(苹果)、orange(橘子)、pear(梨)、banana(香蕉)和 melon(甜瓜)
已经在{\it NetHack}中存在了,所以不要使用这些。
%.Ip
\item[\ib{gender}]
%Your starting gender ({\tt gender:male} or {\tt gender:female}).
%You may specify just the first letter.  Although you can
%still denote your gender using the ``{\tt male}'' and ``{\tt female}''
%options, the ``{\tt gender}'' option will take precedence.
%The default is to randomly pick an appropriate gender.
%Cannot be set with the `{\tt O}' command.
(性别)你的初始性别({\tt gender:male}(性别:男)或{\tt gender:female}
(性别:女))。你可以只指定第一个字母。
尽管你仍然可以用“{\it male}”(男)和“{\it female}”(女)选项来表示你的性别,
不过“{\it gender}”选项有优先权。
默认为随机选择一个合适的性别。
不能使用“{\tt O}”命令进行设置。
%.lp
\item[\ib{help}]
%If more information is available for an object looked at
%with the `{\tt /}' command, ask if you want to see it (default on).
%Turning help off makes just looking at things faster, since you aren't
%interrupted with the ``{\tt More info?}'' prompt, but it also means that you
%might miss some interesting and/or important information.
(帮助)如果使用“{\tt /}”命令所查看的一个物品有更多信息可用时,
询问你是否想要阅读它(默认打开)。
关闭帮助会使查看物品更快,因为你不会被“{\tt More info?}”(更多信息?)所打断,
不过这同样意味着你可能错过一些有趣和/或重要的信息。
%.lp
\item[\ib{horsename}]
%Name your starting horse (ex.\ ``{\tt horsename:Trigger}'').
%Cannot be set with the `{\tt O}' command.
(马的名字)命名你出发时携带的马(例如“{\tt horsename:Trigger}”)。
不能使用“{\tt O}”命令进行设置。
%.lp
\item[\ib{IBMgraphics}]
%Use a predefined selection of IBM extended ASCII characters to display the
%dungeon/effects/traps instead of having to define a full graphics set
%yourself (default off).
%This option also sets up proper handling of graphics
%characters for such terminals, so you should specify it when appropriate
%even if you override the selections with your own graphics strings.
(IBM 图形)使用预先选好的 IBM 扩展 ASCII 字符来显示\zhTransDungeon{}/动画效果/陷阱,
而非由你自己来定义一套完整的图形集(默认关闭)。
该选项还会为这些终端设置合适的图形字符处理方法,所以当合适时你应该打开它,
甚至当你想要用你自己的图形字符串进行替换时。
%.lp
\item[\ib{ignintr}]
%Ignore interrupt signals, including breaks (default off).
忽略中断信号,包括暂停(默认关闭)。

%.lp
\item[\ib{legacy}]
%Display an introductory message when starting the game (default on).
(遗留)开始游戏时显示介绍信息(默认打开)。
%.lp
\item[\ib{lit\_corridor}]
%Show corridor squares seen by night vision or a light source held by your
%character as lit (default off).
(照亮走道)将由夜视力看到的或由你的角色所持光源照亮的走道显示为明亮的
(默认关闭)。
%.lp
\item[\ib{lootabc}]
%Use the old `{\tt a}', `{\tt b}', and `{\tt c}' keyboard shortcuts when
%looting, rather than the mnemonics `{\tt o}', `{\tt i}', and `{\tt b}' (default off).
当搜刮物品时使用旧式的“{\tt a}”、“{\tt b}”和“{\tt c}”快捷键,
而非助记符号“{\tt o}”、“{\tt i}”和“{\tt b}”(默认关闭)。
%.lp
\item[\ib{mail}]
%Enable mail delivery during the game (default on).
(邮件)启用在游戏中投递邮件(默认打开)。
%.lp
\item[\ib{male}]
%An obsolete synonym for ``{\tt gender:male}''.  Cannot be set with the
%`{\tt O}' command.
(男性)“{\tt gender:male}”(性别:男)的一个废弃的同义词。
不能使用“{\tt O}”命令进行设置。
%.lp
\item[\ib{menustyle}]
%Controls the interface used when you need to choose various objects (in
%response to the Drop command, for instance).  The value specified should
%be the first letter of one of the following:  traditional, combination,
%partial, or full.  Traditional was the only interface available for
%earlier versions; it consists of a prompt for object class characters,
%followed by an object-by-object prompt for all items matching the selected
%object class(es).  Combination starts with a prompt for object class(es)
%of interest, but then displays a menu of matching objects rather than
%prompting one-by-one.  Partial skips the object class filtering and
%immediately displays a menu of all objects.  Full displays a menu of
%object classes rather than a character prompt, and then a menu of matching
%objects for selection.
(菜单风格)控制当你需要选择多种物品时所使用的界面
(例如回应放下(“{\tt D}”,drop)命令时)。
指定的值应为下列其中之一的第一个字母:traditional(传统)、
combination(混合)、partial(部分)或 full(全部)。
Traditional 是以前版本唯一可用的界面;它包括了询问物品类型字符的提示符,
接着是一个物品接一个物品进行选择的提示符,
用来挑选那些所有符合上一步所选物品类型的物品。
Combination 以用来选择感兴趣的物品类型的提示符开始,
不过其后显示匹配的物品菜单,而不是一个接一个地用提示符来选。
Partial 跳过对物品类型进行过滤,直接显示所有物品的菜单。
Full 显示一个物品类型的菜单,而非字符提示符,接着显示用以选择的匹配物品的菜单。
\item[\ib{menu\_deselect\_all}]
%Menu character accelerator to deselect all items in a menu.
%Implemented by the Amiga, Gem, X11 and tty ports.
%Default `-'.
(菜单,取消全部选中)用于在菜单中取消全部选中的物品的菜单快捷键。
在 Amiga、Gem、X11 和 tty 移植中实现。默认为“{\tt -}”。
\item[\ib{menu\_deselect\_page}]
%Menu character accelerator to deselect all items on this page of a menu.
%Implemented by the Amiga, Gem and tty ports.
%Default `\verb+\+'.
(菜单,取消一页选中)用于取消当前菜单页中所有物品的菜单快捷键。
在 Amiga、Gem 和 tty 移植中实现。默认为“\verb+\+”。
\item[\ib{menu\_first\_page}]
%Menu character accelerator to jump to the first page in a menu.
%Implemented by the Amiga, Gem and tty ports.
%Default `\verb+^+'.
(菜单第一页)用于跳到菜单第一页的菜单快捷键。
在 Amiga、Gem 和 tty 移植中实现。默认为“\verb+^+”。
\item[\ib{menu\_headings}]
%Controls how the headings in a menu are highlighted.
%Values are ``{\tt bold}'', ``{\tt inverse}'', or ``{\tt underline}''.
%Not all ports can actually display all three types.
(菜单标题)控制菜单中的标题如何突出显示。
可能的值有“{\tt bold}”(加粗)、“{\tt inverse}” (反显)
或“{\tt underline}”(下划线)。
不是所有移植都能显示全部这三种类型。
\item[\ib{menu\_invert\_all}]
%Menu character accelerator to invert all items in a menu.
%Implemented by the Amiga, Gem, X11 and tty ports.
%Default `@'.
(菜单,反选全部)用于反选所有菜单物品的菜单快捷键。
在 Amiga、Gem、X11 和 tty 移植中实现。默认为“{\tt @}”。
\item[\ib{menu\_invert\_page}]
%Menu character accelerator to invert all items on this page of a menu.
%Implemented by the Amiga, Gem and tty ports.
%Default `\verb+~+'.
(菜单,反选一页)将当前菜单页所有物品反选的菜单快捷键。
在 Amiga、Gem 和 tty 移植中实现。默认为“\verb+~+”。
\item[\ib{menu\_last\_page}]
%Menu character accelerator to jump to the last page in a menu.
%Implemented by the Amiga, Gem and tty ports.
%Default `\verb+|+'.
(菜单最后一页)用于跳到菜单最后一页的菜单快捷键。
在 Amiga、Gem 和 tty 移植中实现。默认为“\verb+|+”。
\item[\ib{menu\_next\_page}]
%Menu character accelerator to goto the next menu page.
%Implemented by the Amiga, Gem and tty ports.
%Default `\verb+>+'.
(菜单下一页)用于翻到下一菜单页的菜单快捷键。
在 Amiga、Gem 和 tty 移植中实现。默认为“\verb+>+”
\item[\ib{menu\_previous\_page}]
%Menu character accelerator to goto the previous menu page.
%Implemented by the Amiga, Gem and tty ports.
%Default `\verb+<+'.
(菜单上一页)用于翻到上一菜单页的菜单快捷键。
在 Amiga、Gem 和 tty 移植中实现。默认为“\verb+<+”。
\item[\ib{menu\_search}]
%Menu character accelerator to search for a menu item.
%Implemented by the Amiga, Gem and X11 ports.
%Default `:'.
(菜单,搜索)用于搜索菜单中物品的菜单快捷键。
在 Amiga、Gem 和 X11 移植中实现。默认为“{\tt :}”。
\item[\ib{menu\_select\_all}]
%Menu character accelerator to select all items in a menu.
%Implemented by the Amiga, Gem, X11 and tty ports.
%Default `.'.
(菜单,全部选择)用于在菜单中选择所有物品的菜单快捷键。
在 Amiga、Gem、X11 和 tty 移植中实现。默认为“{\tt .}”。
\item[\ib{menu\_select\_page}]
%Menu character accelerator to select all items on this page of a menu.
%Implemented by the Amiga, Gem and tty ports.
%Default `,'.
(菜单,选择一页)用于选择当前菜单页上的所有物品的菜单快捷键。
在 Amiga、Gem 和 tty 移植中实现。默认为“{\tt ,}”。
%.lp
\item[\ib{monsters}]
%Set the characters used to display monster classes (default
%``\verb+abcdefghijklmnopqrstuv+
%\verb+wxyzABCDEFGHIJKLMNOPQRSTUVWXYZ@ '&;:~]+'').
%This string is subjected to the same processing as the {\it dungeon\/} option.
%The order of the symbols is
%ant or other insect, blob, cockatrice,
%dog or other canine, eye or sphere, feline,
%gremlin, humanoid, imp or minor demon,
%jelly, kobold, leprechaun,
%mimic, nymph, orc,
%piercer, quadruped, rodent,
%arachnid or centipede, trapper or lurker above, horse or unicorn,
%vortex, worm, xan or other mythical/fantastic insect,
%light, zruty,
%angelic being, bat or bird, centaur,
%dragon, elemental, fungus or mold,
%gnome, giant humanoid, invisible monster,
%jabberwock, Keystone Kop, lich,
%mummy, naga, ogre,
%pudding or ooze, quantum mechanic, rust monster,
%snake, troll, umber hulk,
%vampire, wraith, xorn,
%apelike creature, zombie,
%human, ghost, golem,
%demon, sea monster, lizard,
%long worm tail, and mimic.
%Cannot be set with the `{\tt O}' command.
(怪物)设置用于显示怪物类别的字符
(默认为“\verb+abcdefghijklmnopqrstuvwxyzABCDEF+
\verb+GHIJKLMNOPQRSTUVWXYZ@ '&;:~]+”)。
该字符串会接受与 {\it dungeon\/} 选项相同的处理。
符号的顺序为\zhTransAnt(ant)或其他昆虫(insect)、
\zhTransBlobs(blob)、\zhTransCockatrice(cockatrice)、
\zhTransDog(dog)或其他犬类(canine)、\zhTransEye(eye)
或\zhTransSphere(sphere)、\zhTransFeline(feline)、
\zhTransGremlin(gremlin)、\zhTransHumanoid(humanoid)、\zhTransImp(imp)
或\zhTransMinorDemon(minor demon)、\zhTransJellies(jelly)、
\zhTransKobold(kobold)、\zhTransLeprechaun(leprechaun)、
\zhTransMimic(mimic)、\zhTransNymph(nymph)、\zhTransOrcs(orc)、
\zhTransPiercer(piercer)、\zhTransQuadruped(quadruped)、
\zhTransRodent(rodent)、\zhTransArachnid(arachnid)
或\zhTransCentipede(centipede)、\zhTransTrapper(trapper)
或\zhTransLurkerAbove(lurker above)、\zhTransHorse(horse)
或\zhTransUnicorn(unicorn)、\zhTransVortex(vortex)、
\zhTransWorm(worm)、\zhTransXan(xan)或其他神话中的/奇幻的昆虫、
\zhTransLight(light)、\zhTransZruty(zruty)、
\zhTransAngelicBeing(angelic being)、\zhTransBat(bat)
或\zhTransBird(bird)、\zhTransCentaur(centaur)、\zhTransDragon(dragon)、
\zhTransElemental(elemental)、\zhTransFungi(fungus)
或\zhTransMold(mold)、\zhTransGnomes(gnome)、
\zhTransGiantHumanoid(giant humanoid)、隐身的怪物(invisible monster)、
\zhTransJabberwock(jabberwock)、\zhTransKeystoneKop(Keystone Kop)、
\zhTransLich(lich)、\zhTransMummy(mummy)、\zhTransNaga(naga)、
\zhTransOrge(ogre)、\zhTransPudding(pudding)或\zhTransOoze(ooze)、
\zhTransQuantumMechanic(quantum mechanic)、
\zhTransRustMonster(rust monster)、
\zhTransSnake(snake)、\zhTransTroll(troll)、
\zhTransUmberHulk(umber hulk)、\zhTransVampire(vampire)、
\zhTransWraith(wraith)、\zhTransXorn(xorn)、
\zhTransApelikeCreature(apelike creature)、\zhTransZombie(zombie)、
\zhTransHumans(human)、\zhTransGhosts(ghost)、\zhTransGolem(golem)、
\zhTransDemons(demon)、\zhTransSeaMonster(sea monster)、
\zhTransLizard(lizard)、\zhTransLongWorm(long worm)的尾巴和\zhTransMimic。
不能使用“{\tt O}”命令进行设置。

%.lp
\item[\ib{msghistory}]
%The number of top line messages to save (and recall with `{\tt \^{}P}')
%(default 20). Cannot be set with the `{\tt O}' command.
(消息历史)顶部行消息保存(和用“{\tt \^{}P}”回想)的条数(默认 20)。
不能使用“{\tt O}”命令进行设置。
%.lp
\item[\ib{msg\_window}]
%Allows you to change the way recalled messages are displayed.
%(It is currently implemented for tty only.) The possible values are:
(消息窗口)允许你改变回想时消息显示的方式。
(目前仅在 tty 实现。)可能的值为:

%.sd
%.si
%{\tt s} --- single message (default, this was the behavior before 3.4.0).\\
%{\tt c} --- combination, two messages as {\it single\/}, then as {\it full\/}.\\
%{\tt f} --- full window, oldest message first.\\
%{\tt r} --- full window, newest message first.
{\tt s} --- 单条(single)消息(默认,这是 3.4.0 以前的行为)。\\
{\tt c} --- 混合(combination),两条消息时与{\it single\/}相同,
更多时与{\it full\/}相同。\\
{\tt f} --- 全部(full)窗口,旧的消息在前。\\
{\tt r} --- 全部窗口,新的消息在前。
%.ei
%.ed

%For backward compatibility, no value needs to be specified (which
%defaults to {\it full\/}), or it can be negated (which defaults
%to {\it single\/}). 
为了向后兼容,不一定要指定值(此时默认为{\it full\/}),
或者它可以被否定(此时默认为{\it single\/})。
%.lp
\item[\ib{name}]
%Set your character's name (defaults to your user name).  You can also
%set your character's role by appending a dash and one or more letters of
%the role (that is, by suffixing one of
%``{\tt -A -B -C -H -K -M -P -Ra -Ro -S -T -V -W}'').
%If ``{\tt -@}'' is used for the role, then a random one will be
%automatically chosen.
%Cannot be set with the `{\tt O}' command.
(名字)设置你的角色名字(默认为你的用户名)。
你可以同时设置你的角色的职业,只需在名字后加上一个连字号({\tt -})和该职业的一个
或多个字母(即加上后缀“{\tt -A -B -C -H -K -M -P -Ra -Ro -S -T -V -W}”
其中之一)。
如果指定“{\tt -@}”作为职业,则一个随机的职业会被自动选取。
不能使用“{\tt O}”命令进行设置。
%.lp
\item[\ib{news}]
%Read the {\it NetHack\/} news file, if present (default on).
%Since the news is shown at the beginning of the game, there's no point
%in setting this with the `{\tt O}' command.
(新闻)阅读{\it NetHack\/}新闻文件,如果有的话(默认打开)。
因为新闻会显示在游戏的开头,所以没有理由使用“{\tt O}”命令进行设置。
%.lp
\item[\ib{null}]
%Send padding nulls to the terminal (default off).
向终端发送填充用的 null(默认关闭)。
%.lp
\item[\ib{number\_pad}]
%Use the number keys to move instead of {\tt [yuhjklbn]} (default 0 or off).
%(number\_pad:2 invokes the old DOS behavior where `{\tt 5}' means `{\tt g}', 
%meta-`{\tt 5}' means `{\tt G}',  and meta-`{\tt 0}' means `{\tt I}'.)
(小键盘)用数字键来移动,而非{\tt [yuhjklbn]}(默认 0 或关闭)。
({\tt number\_pad:2} 使用旧的 DOS 行为,此时“{\tt 5}”意味着“{\tt g}”,
meta-“{\tt 5}”意味着“{\tt G}”,meta-“{\tt 0}”意味着“{\tt I}”。)
%.lp
\item[\ib{objects}]
%Set the characters used to display object classes (default
%``\verb&])[="(%!?+/$*`0_.&'').
%This string is subjected to the same processing as the {\it dungeon\/} option.
%The order of the symbols is
%illegal-object (should never be seen), weapon, armor, ring, amulet, tool,
%food, potion, scroll, spellbook, wand, gold, gem or rock, boulder or statue,
%iron ball, chain, and venom.
%Cannot be set with the `{\tt O}' command.
(物品)设置用于显示物品类别的字符
(默认为“\verb&])[="(%!?+/$*`0_.&”)。
该字符串会接受与 {\it dungeon\/} 选项相同的处理。
符号的顺序为非法的物品(应该永远不会看见)、武器(weapon)、
护甲(armor)、戒指(ring)、护身符(amulet)、工具(tool)、食物(food)、
药水(potion)、卷轴(scroll)、咒语书(spellbook)、魔杖(wand)、
黄金(gold)、宝石(gem)或石头(rock)、\zhTransBoulders(boulder)
或雕像(statue)、铁球(iron ball)、链子(chain)和毒液(venom)。
不能使用“{\tt O}”命令进行设置。
%.lp
\item[\ib{packorder}]
%Specify the order to list object types in (default
%``\verb&")[%?+!=/(*`0_&''). The value of this option should be a string
%containing the symbols for the various object types.  Any omitted types
%are filled in at the end from the previous order.
(背包顺序)指定列举物品类型的顺序(默认为“\verb&")[%?+!=/(*`0_&”)。
该选项的值应该是一个包含了各种物品类型符号的字符串。
任何省略的类型会在最后以前述顺序进行填充。

%.lp
\item[\ib{perm\_invent}]
%If true, always display your current inventory in a window.  This only
%makes sense for windowing system interfaces that implement this feature.
如果打开的话,会在一个窗口中一直显示你的目前\zhTransInventory。
这只对那些实现了该特性的窗口系统界面有意义。
%.lp
\item[\ib{pettype}]
%Specify the type of your initial pet, if you are playing a character class
%that uses multiple types of pets; or choose to have no initial pet at all.
%Possible values are ``{\tt cat}'', ``{\tt dog}'' and ``{\tt none}''.
%Cannot be set with the `{\tt O}' command.
(\zhTransPets{}类型)指定你的初始\zhTransPets{}的类型,
如果你正在玩一个使用多种\zhTransPets{}类型的角色职业的话;
或者选择完全不带初始\zhTransPets。
可能的值为“{\tt cat}”(猫)、“{\tt dog}”(狗)和“{\tt none}”(没有)。
不能使用“{\tt O}”命令进行设置。
%.Ip
\item[\ib{pickup\_burden}]
%When you pick up an item that would exceed this encumbrance
%level (Unburdened, Burdened, streSsed, straiNed, overTaxed,
%or overLoaded), you will be asked if you want to continue.
%(Default `S').
(捡起,负重)当你捡起一个会超过该负重水平
(Unburdened(\zhTransUnburdened)、Burdened(\zhTransBurdened)、
streSsed(\zhTransStressed)、straiNed(\zhTransStrained)、
overTaxed(\zhTransOvertaxed)或 overLoaded(\zhTransOverloaded))的物品时,
你会被询问是否想要继续。(默认为“{\tt S}”)。
%.lp
\item[\ib{pickup\_types}]
%Specify the object types to be picked up when ``{\it autopickup\/}'' 
%is on.  Default is all types.  If your copy of the game has the
%experimental compile time option AUTOPICKUP\_EXCEPTIONS included,
%you may be able to use ``{\it autopickup\_exception\/}'' configuration
%file lines to further refine ``{\it autopickup\/}'' behavior.
(捡起类型)指定当“{\it autopickup\/}”(自动捡起)打开时捡起的物品类型。
默认为所有类型。
如果你的游戏拷贝包含了实验性的编译时选项 AUTOPICKUP\_EXCEPTIONS,
你可以使用“{\tt AUTOPICKUP\_EXCEPTION\/}”(自动捡起例外)配置文件行
来进一步精确定制“{\it autopickup\/}”的行为。
%.lp
\item[\ib{prayconfirm}]
%Prompt for confirmation before praying (default on).
(祈祷确认)祈祷之前显示提示符以进行确认(默认打开)。
%.lp
\item[\ib{pushweapon}]
%Using the `w' (wield) command when already wielding
%something pushes the old item into your alternate weapon slot (default off).
(推武器)当已经握持某件物品时,使用“{\tt w}”(握持,wield)命令会将旧物品推入
你的备用武器槽(默认关闭)。
%.Ip
\item[\ib{race}]
%Selects your race (for example, ``{\tt race:human}'').  Default is random.
%Cannot be set with the `{\tt O}' command.
(种族)选择你的种族(例如,“{\tt race:human}”(种族:\zhTransHumans))。
默认随机。不能使用“{\tt O}”命令进行设置。
%.lp
\item[\ib{rest\_on\_space}]
%Make the space bar a synonym for the `{\tt .}' (rest) command (default off).
(空格键进行休息)使空格键成为“{\tt .}”(休息,rest)命令的同义词(默认关闭)。
%.lp
\item[\ib{role}]
%Pick your type of character (ex.\ ``{\tt role:Samurai}'');
%synonym for ``{\it character\/}''.  See ``{\it name\/}'' for an alternate method
%of specifying your role.  Normally only the first letter of the
%value is examined; `r' is an exception with ``{\tt Rogue}'', {\tt Ranger}'',
%and ``{\tt random}'' values.
(职业)选择你角色的种类(例如“{\tt role:Samurai}”
(职业:\zhTransSamurai));是“{\it character\/}”(角色)的同义词。
见“{\it name\/}”(名字)获得指定你的职业的另一种方法。
通常只有该值的第一个字母会被检查;“{\tt r}”是一个例外,它可以表示
“{\tt Rogue}”(\zhTransRogues)、“{\tt Ranger}”(\zhTransRangers)
和“{\tt random}”(随机)。
%.lp
\item[\ib{runmode}]
%Controls the amount of screen updating for the map window when engaged
%in multi-turn movement (running via {\tt shift}+direction
%or {\tt control}+direction
%and so forth, or via the travel command or mouse click).
%The possible values are:
(跑步模式)控制当进行多回合移动(通过{\tt shift}+方向或{\tt control}+方向等,
或者通过行进命令(“{\tt _}”)或鼠标点击进行跑步)时地图窗口的屏幕更新数量。
可能的值为:

%.sd
%.si
%{\tt teleport} --- update the map after movement has finished;\\
%{\tt run} --- update the map after every seven or so steps;\\
%{\tt walk} --- update the map after each step;\\
%{\tt crawl} --- like {\it walk\/}, but pause briefly after each step.
{\tt teleport} --- (瞬移)当移动结束时更新地图;\\
{\tt run} --- (跑步)移动大约七步后更新地图;\\
{\tt walk} --- (行走)每一步移动后更新地图;\\
{\tt crawl} --- (爬行)与{\it walk\/}相似,不过在每一步后短暂地暂停。
%.ei
%.ed

%This option only affects the game's screen display, not the actual
%results of moving.  The default is {\it run\/}; versions prior to 3.4.1 
%used {\it teleport\/} only.  Whether or not the effect is noticeable will
%depend upon the window port used or on the type of terminal.
该选项只影响游戏的屏幕显示,不影响移动的实际结果。
默认为{\it run\/};早于 3.4.1 的版本只使用{\it teleport\/}。
效果是否明显会取决于使用的窗口移植或终端的类型。
%.lp
\item[\ib{safe\_pet}]
%Prevent you from (knowingly) attacking your pets (default on).
% I think "knowingly" is a typo, which is actually "accidentally"。
(安全,\zhTransPets)防止你(无意中)进攻你的\zhTransPets(默认打开)。
%.lp
\item[\ib{scores}]
%Control what parts of the score list you are shown at the end (ex.\
%``{\tt scores:5top scores/4around my score/own scores}'').  Only the first
%letter of each category (`{\tt t}', `{\tt a}' or `{\tt o}') is necessary.
(分数)控制结束时向你显示分数列表的哪部分
(例如“{\tt scores:5top scores/4around my score/own scores}”
(分数:最高 5 个分数/我的分数两边的各 4 个分数/我自己的分数))。
只有各个类别的第一个字母(“{\tt t}”、“{\tt a}”或“{\tt o}”)是必须的。
%.lp
\item[\ib{showexp}]
%Show your accumulated experience points on bottom line (default off).
在底部行显示你的累积\zhTransExperiencePoints(默认关闭)。
%.lp
\item[\ib{showrace}]
%Display yourself as the glyph for your race, rather than the glyph
%for your role (default off).  Note that this setting affects only
%the appearance of the display, not the way the game treats you.
(显示种族)以你的种族符号来显示你自己,而不使用你职业的符号(默认关闭)。
注意该设置只影响显示的外观,不影响游戏如何对待你。
%.lp
\item[\ib{showscore}]
%Show your approximate accumulated score on bottom line (default off).
(显示分数)在底部行显示你的大致累积分数(默认关闭)。
%.lp
\item[\ib{silent}]
%Suppress terminal beeps (default on).
(安静)抑制终端的响铃(beep)(默认打开)。
%.lp
\item[\ib{sortpack}]
%Sort the pack contents by type when displaying inventory (default on).
(分类,背包)当显示\zhTransInventory{}时按类型对背包内容进行分类(默认打开)。
%.lp
\item[\ib{sound}]
%Enable messages about what your character hears (default on).
%Note that this has nothing to do with your computer's audio capabilities.
%This option is only partly under player control.  The game toggles it
%off and on during and after sleep, for example.
(声音)启用描述你的角色所听到声音的消息(默认打开)。
注意这与你的电脑音频能力无关。该选项仅受玩家部分控制。
例如,在睡觉时游戏会关闭它,睡觉后则打开它。
%.lp
\item[\ib{standout}]
%Boldface monsters and ``{\tt --More--}'' (default off).
(突出)将怪物和“{\tt --More--}”(更多)加粗(默认关闭)。
%.lp
\item[\ib{sparkle}]
%Display a sparkly effect when a monster (including yourself) is hit by an
%attack to which it is resistant (default on).
(闪烁)当一个怪物(包括你自己)被一次攻击击中,且该攻击被免疫时
显示闪烁效果(默认打开)。
%.lp
\item[\ib{suppress\_alert}]
%This option may be set to a NetHack version level to suppress
%alert notification messages about feature changes for that 
%and prior versions (ex.\ ``{\tt suppress\_alert:3.3.1}'')
(抑制警告)这个选项可以设置为一个{\it NetHack}版本,
用来抑制关于该版本与之前版本特性改变的通知消息
(例如“{\tt suppress\_alert:3.3.1}”)。
%.lp
\item[\ib{time}]
%Show the elapsed game time in turns on bottom line (default off).
(时间)在底部行以回合数显示经过的游戏时间(默认关闭)。
%.lp
\item[\ib{timed\_delay}]
%When pausing momentarily for display effect, such as with explosions and
%moving objects, use a timer rather than sending extra characters to the
%screen.  (Applies to ``tty'' interface only; ``X11'' interface always
%uses a timer based delay.  The default is on if configured into the
%program.)
(按时间延迟)当为了显示效果而暂时暂停时,例如显示爆炸和移动中的物品时,
使用定时器而非向屏幕发送多余的字符。
(只应用于“tty”界面;“X11”界面总是使用基于定时器的延迟。
如果配置进程序的话,默认为打开。)
%.lp
\item[\ib{tombstone}]
%Draw a tombstone graphic upon your death (default on).
(墓碑)当你死亡时画出一个墓碑图形(默认打开)。
%.lp
\item[\ib{toptenwin}]
%Put the ending display in a NetHack window instead of on stdout (default off).
%Setting this option makes the score list visible when a windowing version
%of NetHack is started without a parent window, but it no longer leaves
%the score list around after game end on a terminal or emulating window.
(排名前十位胜利)
将结束时的显示放入{\it NetHack}窗口中,而非标准输出(默认关闭)。
设置该选项使得当窗口版本的{\it NetHack}不是由一个父窗口启动时,
分数列表变得可见,不过当在终端或模拟窗口中时,结束游戏后分数列表不再保留。
%.lp
\item[\ib{traps}]
%Set the graphics symbols for displaying traps (default
%``\verb&^^^^^^^^^^^^^^^^^"^^^^&'').
%The {\it traps\/} option should be followed by a string of 1--22
%characters to be used instead of the default traps characters.
%This string is subjected to the same processing as the {\it dungeon\/} option.
(\zhTransTraps)设置用于显示\zhTransTraps{}的图形符号
(默认“\verb&^^^^^^^^^^^^^^^^^"^^^^&”)。
{\it traps\/}选项后面应跟随着一个含 1--22 个字符的字符串,
用来代替默认的\zhTransTraps{}字符。
该字符串会接受与 {\it dungeon\/} 选项相同的处理。

%The order of the symbols is:
%arrow trap, dart trap, falling rock trap, squeaky board, bear trap,
%land mine, rolling boulder trap, sleeping gas trap, rust trap, fire trap,
%pit, spiked pit, hole, trap door, teleportation trap, level teleporter,
%magic portal, web, statue trap, magic trap, anti-magic field, polymorph trap.
符号的顺序为:
\zhTransArrowTrap(arrow trap)、\zhTransDartTrap(dart trap)、
\zhTransFallingRockTrap(falling rock trap)、
\zhTransSqueakyBoard(squeaky board)、\zhTransBearTrap(bear trap)、
\zhTransLandMine(land mine)、
\zhTransRollingBoulderTrap(rolling boulder trap)、
\zhTransSleepingGasTrap(sleeping gas trap)、\zhTransRustTrap(rust trap)、
\zhTransFireTrap(fire trap)、\zhTransPit(pit)、
\zhTransSpikedPit(spiked pit)、\zhTransHole(hole)、
\zhTransTrapDoor(trap door)、\zhTransTeleportationTrap(teleportation trap)、
\zhTransLevelTeleporter(level teleporter)、
\zhTransMagicPortal(magic portal)、\zhTransWeb(web)、
\zhTransStatueTrap(statue trap)、\zhTransMagicTrap(magic trap)、
\zhTransAntiMagicField(anti-magic field)、
\zhTransPolymorphTrap(polymorph trap)。

%Cannot be set with the `{\tt O}' command.
不能使用“{\tt O}”命令进行设置。
%.lp
\item[\ib{travel}]
%Allow the travel command (default on).  Turning this option off will
%prevent the game from attempting unintended moves if you make inadvertent
%mouse clicks on the map window.
(行进)允许行进命令(“{\tt \_}”)(默认打开)。
关闭该选项会防止当你不小心点击了地图时试图进行的无心移动。
%.lp
\item[\ib{verbose}]
%Provide more commentary during the game (default on).
(详细)在游戏中提供更多说明(默认打开)。
%.lp
\item[\ib{windowtype}]
%Select which windowing system to use, such as ``{\tt tty}'' or ``{\tt X11}''
%(default depends on version).
%Cannot be set with the `{\tt O}' command.
(窗口类型)选择使用的窗口系统,如“{\tt tty}”或“{\tt X11}”
(默认取决于版本)。不能使用“{\tt O}”命令进行设置。
\elist

%.hn 2
%\subsection*{Window Port Customization options}
\subsection*{窗口移植定制选项}

%.pg
%Here are explanations of the various options that are
%used to customize and change the characteristics of the
%windowtype that you have chosen.
%Character strings that are too long may be truncated.
%Not all window ports will adjust for all settings listed
%here.  You can safely add any of these options to your 
%config file, and if the window port is capable of adjusting 
%to suit your preferences, it will attempt to do so. If it
%can't it will silently ignore it.  You can find out if an 
%option is supported by the window port that you are currently
%using by checking to see if it shows up in the Options list.
%Some options are dynamic and can be specified during the game
%with the `{\tt O}' command.
这里是用于定制和改变你选择的窗口类型的特征的各种选项。
太长的字符串可能会被截断。不是所有窗口移植都会被这里列出的所有设置调整。
你可以安全地将这些选项任何之一加入你的配置文件,
如果该窗口移植有能力调整以适应你的偏好,则它会试着这样做。
如果它不能则会无声地忽略该选项。
通过查看一个选项是否出现在选项列表中,
你可以找出它是否被你目前所用的窗口移植支持。
一些选项是动态的,可以在游戏中使用“{\tt O}”命令进行指定。

\blist{}
%.lp
\item[\ib{align\_message}]
% Where to align or place the message window (top, bottom, left, or right)
(对齐消息)将消息窗口对齐或放置到哪里
({\tt top}(项部)、{\tt bottom}(底部)、{\tt left}(左部)或 {\tt right}(右部))。
%.lp
\item[\ib{align\_status}]
% Where to align or place the status window (top, bottom, left, or right).
(对齐状态)将状态窗口对齐或放置到哪里
({\tt top}(项部)、{\tt bottom}(底部)、{\tt left}(左部)或 {\tt right}(右部))。
%.lp
\item[\ib{ascii\_map}]
%NetHack should display an ascii map if it can.
(ASCII 地图)当{\it NetHack}有能力时应当显示 ASCII 地图。
%.lp
\item[\ib{color}]
%NetHack should display color if it can for different monsters, 
%objects, and dungeon features
(颜色)当{\it NetHack}有能力时应当为不同的怪物、物品和\zhTransDungeon{}景观显示颜色。
%.lp
\item[\ib{eight\_bit\_tty}]
%Pass eight-bit character values (for example, specified with the {\it
%traps \/} option) straight through to your terminal (default off).
(八位 tty)直接发送八位字符值(例如由{\it traps}(陷阱)选项指定)到你的终端
(默认关闭)。
%.lp
\item[\ib{font\_map}]
%NetHack should use a font by the chosen name for the map window.
(字体,地图){\it NetHack}应该在地图窗口中使用该指定名字对应的字体。
%.lp
\item[\ib{font\_menu}]
%NetHack should use a font by the chosen name for menu windows.
(字体,菜单){\it NetHack}应该在菜单窗口中使用该指定名字对应的字体。
%.lp
\item[\ib{font\_message}]
%NetHack should use a font by the chosen name for the message window.
(字体,消息){\it NetHack}应该在消息窗口中使用该指定名字对应的字体。
%.lp
\item[\ib{font\_status}]
%NetHack should use a font by the chosen name for the status window.
(字体,状态){\it NetHack}应该在状态窗口中使用该指定名字对应的字体。
%.lp
\item[\ib{font\_text}]
%NetHack should use a font by the chosen name for text windows.
(字体,文本){\it NetHack}应该在文本窗口中使用该指定名字对应的字体。
%.lp
\item[\ib{font\_size\_map}]
%NetHack should use this size font for the map window.
(字体大小,地图){\it NetHack}应该在地图窗口中使用该字体大小。
%.lp
\item[\ib{font\_size\_menu}]
%NetHack should use this size font for menu windows.
(字体大小,菜单){\it NetHack}应该在菜单窗口中使用该字体大小。
%.lp
\item[\ib{font\_size\_message}]
%NetHack should use this size font for the message window.
(字体大小,消息){\it NetHack}应该在消息窗口中使用该字体大小。
%.lp
\item[\ib{font\_size\_status}]
%NetHack should use this size font for the status window.
(字体大小,状态){\it NetHack}应该在状态窗口中使用该字体大小。
%.lp
\item[\ib{font\_size\_text}]
%NetHack should use this size font for text windows.
(字体大小,文本){\it NetHack}应该在文本窗口中使用该字体大小。
%.lp
\item[\ib{fullscreen}]
%NetHack should try and display on the entire screen rather than in a window.
(全屏){\it NetHack}应该尝试在整个屏幕中显示,而非在窗口中。
%.lp
\item[\ib{hilite\_pet}]
%Visually distinguish pets from similar animals (default off).
%The behavior of this option depends on the type of windowing you use.
%In text windowing, text highlighting or inverse video is often used;
%with tiles, generally displays a heart symbol near pets.
以视觉上的方式将\zhTransPets{}与相似的动物区分出来(默认关闭)。
这一选项的行为取决于你所使用窗口的类型。
在文本类型的窗口中,经常使用文字高亮或反显;对于瓦片(tile),
通常在\zhTransPets{}附近显示一个心形符号。
%.lp
\item[\ib{large\_font}]
%NetHack should use a large font.
(大号字体){\it NetHack}应该使用大号字体。
%.lp
\item[\ib{map\_mode}]
%NetHack should display the map in the manner specified.
(地图模式){\it NetHack}应该以指定的方式来显示地图。
%.lp
\item[\ib{mouse\_support}]
%Allow use of the mouse for input and travel.
(鼠标支持)允许使用鼠标进行输入和行进。
%.lp
\item[\ib{player\_selection}]
%NetHack should pop up dialog boxes or use prompts for character selection.
(玩家选择){\it NetHack}应该弹出对话框或使用提示符来进行角色选择。
%.lp
\item[\ib{popup\_dialog}]
%NetHack should pop up dialog boxes for input.
(弹出对话框){\it NetHack}应该弹出对话框以进行输入。
%.lp
\item[\ib{preload\_tiles}]
%NetHack should preload tiles into memory.
%For example, in the protected mode MSDOS version, control whether tiles
%get pre-loaded into RAM at the start of the game.  Doing so
%enhances performance of the tile graphics, but uses more memory. (default on).
%Cannot be set with the `{\tt O}' command.
(预加载瓦片){\it NetHack}应该将瓦片(tile)预加载进内存。
例如,在保护模式的 MSDOS 版本中,控制在游戏开始时是否将瓦片预加载进内存。
这样做会增强瓦片图形的性能,不过会使用更多内存(默认打开)。
不能使用“{\tt O}”命令进行设置。
%.lp
\item[\ib{scroll\_amount}]
%NetHack should scroll the display by this number of cells
%when the hero reaches the scroll\_margin.
(滚动量)在英雄到达{\it scroll\_margin}(滚动边缘)时,
{\it NetHack}应该将显示滚动该数目个格子。
%.lp
\item[\ib{scroll\_margin}]
%NetHack should scroll the display when the hero or cursor
%is this number of cells away from the edge of the window.
(滚动边缘)当英雄或光标离窗口的边缘相距该数目个格子时,
{\it NetHack}应该将显示进行滚动。
%.lp
\item[\ib{softkeyboard}]
%Display an onscreen keyboard.  Handhelds are most likely to support this option.
(软键盘)显示一个虚拟键盘。手持设备很可能支持该选项。
%.lp
\item[\ib{splash\_screen}]
%NetHack should display an opening splash screen when it starts up (default yes).
(欢迎界面){\it NetHack}应该在启动时显示一个开头欢迎界面(默认打开)。
%.lp
\item[\ib{tiled\_map}]
%NetHack should display a tiled map if it can.
(瓦片地图){\it NetHack}有能力时应该显示瓦片地图。
%.lp
\item[\ib{tile\_file}]
%Specify the name of an alternative tile file to override the default.
(瓦片文件)指定用以代替默认瓦片文件的文件名。
%.lp
\item[\ib{tile\_height}]
%Specify the preferred height of each tile in a tile capable port.
(瓦片高度)在支持瓦片的移植中指定一个瓦片的首选高度。
%.lp
\item[\ib{tile\_width}]
%Specify the preferred width of each tile in a tile capable port
(瓦片宽度)在支持瓦片的移植中指定一个瓦片的首选宽度。
%.lp
\item[\ib{use\_inverse}]
%NetHack should display inverse when the game specifies it.
(使用反显){\it NetHack}应该在游戏指定反显时显示它。
%.lp
\item[\ib{vary\_msgcount}]
%NetHack should display this number of messages at a time in the message window.
(变化的,消息数目){\it NetHack}应该在消息窗口中一次显示该数目的消息。
%.lp
\item[\ib{windowcolors}]
%NetHack should display windows with the specified foreground/background 
%colors if it can.
(窗口颜色){\it NetHack}应该在有能力时用指定的前景/背景色来显示窗口。
%.lp
\item[\ib{wraptext}]
%NetHack port should wrap long lines of text if they don't fit in 
%the visible area of the window.
(文字换行){\it NetHack}应该在较长的文字行超出窗口的可视区域时将其换行。
\elist

%.hn 2
%\subsection*{Platform-specific Customization options}
\subsection*{平台相关的定制选项}

%.pg
%Here are explanations of options that are used by specific platforms 
%or ports to customize and change the port behavior.
这里是对用于特定平台或移植的选项的解释,
这些选项用于定制和改变移植的行为。

\blist{}
%.lp
\item[\ib{altkeyhandler}]
%Select an alternate keystroke handler dll to load ({\it Win32 tty\/ NetHack\/} only).
%The name of the handler is specified without the .dll extension and without any
%path information.
%Cannot be set with the `{\tt O}' command.
({\tt Alt}键处理器)选择另一个按键处理器 dll 来加载
(仅针对 Win32 tty 移植)。
该处理器的名字指定时不带 .dll 扩展名和任何路径信息。
不能使用“{\tt O}”命令进行设置。
%.lp 
\item[\ib{altmeta}]
%(default on, {\it Amiga NetHack \/} only).
(默认打开,仅针对 Amiga 移植)。
%.lp
\item[\ib{BIOS}]
%Use BIOS calls to update the screen display quickly and to read the keyboard
%(allowing the use of arrow keys to move) on machines with an IBM PC
%compatible BIOS ROM (default off, {\it OS/2, PC\/ {\rm and} ST NetHack\/} only).
在有 IBM PC 兼容的 BIOS ROM 的机器上,
使用 BIOS 调用来快速更新屏幕显示和读取键盘(允许使用方向键来移动)
(默认关闭,仅针对 OS/2、PC 和 ST 移植)。
%.lp 
\item[\ib{flush}]
%(default off, {\it Amiga NetHack \/} only).
(默认关闭,仅针对 Amiga 移植)。
%.lp 
\item[\ib{Macgraphics}]
%(default on, {\it Mac NetHack \/} only).
(默认打开,仅针对 Mac 移植)。
%.lp 
\item[\ib{page\_wait}]
%(default off, {\it Mac NetHack \/} only).
(默认关闭,仅针对 Mac 移植)。
%.lp
\item[\ib{rawio}]
%Force raw (non-cbreak) mode for faster output and more
%bulletproof input (MS-DOS sometimes treats `{\tt \^{}P}' as a printer toggle
%without it) (default off, {\it OS/2, PC\/ {\rm and} ST NetHack\/} only).  
%Note:  DEC Rainbows hang if this is turned on.
%Cannot be set with the `{\tt O}' command.
强制使用原始(raw,非 cbreak)模式,以获得更快的输出和更可靠的输入
(如果没有该选项的话,有时 MS-DOS 会将“{\tt \^{}P}”视为打印键)。
(默认关闭,仅针对 OS/2、PC 和 ST 移植)。
注意:如果该选项打开的话,DEC Rainbow 会僵死。
不能使用“{\tt O}”命令进行设置。
%.lp
\item[\ib{soundcard}]
%(default off, {\it PC NetHack \/} only).
%Cannot be set with the `{\tt O}' command.
(声卡)(默认关闭,仅针对 PC 移植)。
不能使用“{\tt O}”命令进行设置。
%.lp
\item[\ib{subkeyvalue}]
%({\it Win32 tty NetHack \/} only).
%May be used to alter the value of keystrokes that the operating system
%returns to NetHack to help compensate for international keyboard issues.
%OPTIONS=subkeyvalue:171/92
%will return 92 to NetHack, if 171 was originally going to be returned.
%You can use multiple subkeyvalue statements in the config file if needed.
%Cannot be set with the `{\tt O}' command.
(仅针对 Win32 tty 移植)。
可以用来更改操作系统返回给{\it NetHack\/}的键值,以帮助应对国际版键盘问题。
{\tt OPTIONS=subkeyvalue:171/92}会使当最初返回的值为 171 时,
返回 92 给{\it NetHack\/}。
如果需要的话,你可以在配置文件中使用多个{\it subkeyvalue\/}语句。
不能使用“{\tt O}”命令进行设置。
%.lp
\item[\ib{video}]
%Set the video mode used ({\it PC\/ NetHack\/} only).
%Values are {\it autodetect\/}, {\it default\/}, or {\it vga\/}. 
%Setting {\it vga\/} (or {\it autodetect\/} with vga hardware present) will cause
%the game to display tiles. 
%Cannot be set with the `{\tt O}' command.
(视频)设置使用的视频模式(仅针对 PC 移植)。
可能的值为{\tt autodetect\/}(自动检测)、{\tt default\/}(默认)或{\tt vga\/}。
设置为{\it vga\/}(或在有 vga 硬件存在时设置为{\it autodetect\/})
会使游戏显示瓦片(tile)。
不能使用“{\tt O}”命令进行设置。
%.lp
\item[\ib{videocolors}]
\begin{sloppypar}
%Set the color palette for PC systems using NO\_TERMS
%(default 4-2-6-1-5-3-15-12-10-14-9-13-11, {\it PC\/ NetHack\/} only).
%The order of colors is red, green, brown, blue, magenta, cyan,
%bright.white, bright.red, bright.green, yellow, bright.blue,
%bright.magenta, and bright.cyan.
%Cannot be set with the `{\tt O}' command.
(视频颜色)为使用 NO\_TERMS 的 PC 系统设置调色板
(默认{\tt 4-2-6-1-5-3-15-12-} {\tt 10-14-9-13-11},仅针对 PC 移植)。
颜色的顺序为 red(红)、green(绿)、brown(棕)、blue(蓝)、magenta(品红)、
cyan(青)、bright.white(亮白)、bright.red(亮红)、bright.green(亮绿)、
yellow(黄)、bright.blue(亮蓝)、bright.magenta(亮品红)和
bright.cyan(亮青)。
不能使用“{\tt O}”命令进行设置。
\end{sloppypar}
%.lp
\item[\ib{videoshades}]
%Set the intensity level of the three gray scales available
%(default dark normal light, {\it PC\/ NetHack\/} only).
%If the game display is difficult to read, try adjusting these scales;
%if this does not correct the problem, try {\tt !color}.
%Cannot be set with the `{\tt O}' command.
(视频阴影)设置三种可用灰度的亮度水平
(默认为{\tt dark normal light}(暗、中等、亮),仅针对 PC 移植)。
如果游戏的显示难以阅读,尝试调整这些灰度;如果这不解决问题,尝试 {\tt !color}。
不能使用“{\tt O}”命令进行设置。
\elist

%.lp
%.hn 2
%\subsection*{Configuring autopickup exceptions}
\subsection*{配置自动捡起例外}

%.pg
%There is an experimental compile time option called AUTOPICKUP_EXCEPTIONS.  
%If your copy of the game was built with that option defined, you can 
%further refine the behavior of the ``{\tt autopickup}'' option beyond 
%what is available through the ``{\tt pickup\_types}'' option.
{\it NetHack\/}有着一个实验性的编译时选项,称为 AUTOPICKUP\_EXCEPTIONS。
如果你的游戏拷贝编译时定义了该选项,你可以比“{\it pickup\_types}”(捡起类型)
选项更进一步地定制“{\it autopickup}”(自动捡起)选项的行为。

%.pg
%By placing ``{\tt autopickup\_exception}'' lines in your configuration
%file, you can define patterns to be checked when the game is about to
%autopickup something.
通过在你的配置文件中放置一些“{\tt AUTOPICKUP\_EXCEPTION}”行,
你可以定义游戏将要自动捡起某些东西时,需要进行检查的模式。

\blist{}
%.lp
%\item[\ib{autopickup\_exception}]
\item[\tb{AUTOPICKUP\_EXCEPTION}]
%Sets an exception to the `{\it pickup\_types}' option.
%The {\it autopickup\_exception\/} option should be followed by a string of 1--80
%characters to be used as a pattern to match against the singular form
%of the description of an object at your location.
(自动捡起例外)给“{\it pickup\_types}”(捡起类型)选项设置一个例外。
一个含 1--80 个字符的字符串应该跟随在{\tt AUTOPICKUP\_EXCEPTION\/}选项后面,
该字符串被用作一个模式来匹配你所在位置的物品的单数形式描述。

%.pg
%You may use the following special characters in a pattern:
你可以在一个模式中使用下列特殊字符:

%\begin{verbatim}
%    *--- matches 0 or more characters.
%    ?--- matches any single character.
%\end{verbatim}
{\tt *} --- 匹配零个或多个字符。\\
{\tt ?} --- 匹配任意单个字符。

%In addition, some characters are treated specially if they occur as the first 
%character in the specified string pattern, specifically:
另外,如果作为指定字符串模式的第一个字符出现时,一些字符将会被特殊对待,
具体来说:

%.sd
%.si
%{\tt <} --- always pickup an object that matches the pattern that follows.\\
%{\tt >} --- never pickup an object that matches the pattern that follows.
{\tt <} --- 总是捡起匹配其后跟随的模式的物品。\\
{\tt >} --- 从不捡起匹配其后跟随的模式的物品。
%.ei
%.ed

%Can be set with the `{\tt O}' command, but the setting is not preserved
%across saves and restores.
可以使用“{\tt O}”命令进行设置,不过该设置不会在保存和恢复间作保留。
\elist

%.pg
%Here's a couple of examples of autopickup\_exceptions:
下面是一些{\tt AUTOPICKUP\_EXCEPTION}的例子:
\begin{verbatim}
%    autopickup_exception="<*arrow"
%    autopickup_exception=">*corpse"
%    autopickup_exception=">* cursed*"
    AUTOPICKUP_EXCEPTION="<*arrow"
    AUTOPICKUP_EXCEPTION=">*corpse"
    AUTOPICKUP_EXCEPTION=">* cursed*"
\end{verbatim}

%The first example above will result in autopickup of any type of arrow.
%The second example results in the exclusion of any corpse from autopickup.
%The last example results in the exclusion of items known to be cursed from autopickup.
%A `never pickup' rule takes precedence over an `always pickup' rule if both match.
上面第一个例子结果会自动捡起任何类型的\zhTransArrows(arrow)。
第二个例子结果会在自动捡起时排除所有尸体(corpse)。
最后一个例子结果会在自动捡起时排除所有已知为\zhTransCursed(cursed)物品。
当同时匹配时,一个“从不捡起”的规则比一个“总是捡起”的规则要优先。

%.lp
%.hn 2
%\subsection*{Configuring User Sounds}
\subsection*{配置用户声音}

%.pg
%Some platforms allow you to define sound files to be played when a message 
%that matches a user-defined pattern is delivered to the message window.
%At this time the Qt port and the win32tty and win32gui ports support the
%use of user sounds.
一些平台允许你定义声音文件,用来在一条匹配上用户定义的模式的消息被发送到消息
窗口时播放。目前 Qt 移植以及 win32tty 和 win32gui 移植支持使用用户声音。

%.pg
%The following config file entries are relevant to mapping user sounds
%to messages:
下列配置文件条目与消息到用户声音的映射相关:

\blist{}
%.lp
%\item[\ib{SOUNDDIR}]
\item[\tb{SOUNDDIR}]
%The directory that houses the sound files to be played.
(声音目录)存放用来播放的声音的目录。
%.lp
%\item[\ib{SOUND}]
\item[\tb{SOUND}]
%An entry that maps a sound file to a user-specified message pattern.
%Each SOUND entry is broken down into the following parts:
(声音)一个将声音文件与用户指定的消息模式作映射的条目。
每一个 {\tt SOUND} 条目被分成如下几个部分:

%.sd
%.si
%{\tt MESG      } --- message window mapping (the only one supported in 3.4).\\
%{\tt pattern   } --- the pattern to match.\\
%{\tt sound file} --- the sound file to play.\\
%{\tt volume    } --- the volume to be set while playing the sound file.
{\tt MESG      } --- 进行映射的消息窗口(唯一在 3.4 中被支持的)。\\
{\tt pattern   } --- (模式)用于匹配的模式。\\
{\tt sound file} --- (声音文件)用于播放的声音文件。\\
{\tt volume    } --- (音量)播放该声音文件时设置的音量。
%.ei
%.ed
\elist

%.pg
%The exact format for the pattern depends on whether the platform is
%built to use {\it regular expressions \/} or NetHack's own internal pattern 
%matching facility. The {\it regular expressions \/} matching can be much more 
%sophisticated than the internal NetHack pattern matching, but requires 
%3rd party libraries on some platforms.  There are plenty of references 
%available elsewhere for explaining {\it regular expressions \/}. You can verify 
%which pattern matching is used by your port with the 
%\#version command.  
该模式的具体格式取决于该平台编译时使用{\it 正则表达式}(regular expression),
还是使用{\it NetHack}自己的内部模式匹配机制。
{\it 正则表达式}匹配与{\it NetHack}内部模式匹配相比要非常复杂,
不过在一些平台上需要第三方库。
在其他地方有大量可用的资料,用以解释{\it 正则表达式}。
你可以用{\tt \#version}命令来查看哪一种模式匹配被你的移植所使用。

%.pg
%NetHack's internal pattern matching routine uses the following
%special characters in its pattern matching:
{\it NetHack}的内部模式匹配函数在模式匹配中使用下列特殊字符:

%\begin{verbatim}
%    *--- matches 0 or more characters.
%    ?--- matches any single character.
%\end{verbatim}
\begin{itemize}[leftmargin=5em, noitemsep, nolistsep]
  \item[] {\tt *} --- 匹配零个或多个字符。
  \item[] {\tt ?} --- 匹配任意单个字符。
\end{itemize}

%.pg
%Here's an example of a sound mapping using NetHack's internal
%pattern matching facility:
下面是使用{\it NetHack}内部模式匹配机制的声音映射例子:
\begin{verbatim}
    SOUND=MESG "*chime of a cash register*" "gong.wav" 50
\end{verbatim}
%specifies that any message with "chime of a cash register" contained
%in it will trigger the playing of "gong.wav".  You can have multiple
%SOUND entries in your config file.
指定任何包含“{\tt chime of a cash register}”(收银机的钟声)的消息会触发
“gong.wav” 的播放。你可以在你的配置文件中使用多个 {\tt SOUND} 条目。

%.lp
%.hn 2
%\subsection*{Configuring NetHack for Play by the Blind}
\subsection*{配置{\it NetHack}以便盲人进行游戏}

%.pg
%NetHack can be set up to use only standard ASCII characters for making
%maps of the dungeons. This makes the MS-DOS versions of NetHack completely
%accessible to the blind who use speech and/or Braille access technologies.
%Players will require a good working knowledge of their screen-reader's
%review features, and will have to know how to navigate horizontally and
%vertically character by character. They will also find the search
%capabilities of their screen-readers to be quite valuable. Be certain to
%examine this Guidebook before playing so you have an idea what the screen
%layout is like. You'll also need to be able to locate the PC cursor. It is
%always where your character is located. Merely searching for an @-sign will
%not always find your character since there are other humanoids represented
%by the same sign. Your screen-reader should also have a function which
%gives you the row and column of your review cursor and the PC cursor.
%These co-ordinates are often useful in giving players a better sense of the
%overall location of items on the screen.
{\it NetHack}可以设置成只使用标准 ASCII 字符来绘制\zhTransDungeon{}的地图。
这使得 MS-DOS 版本的{\it NetHack}可以完全由借助语音合成器和/或盲文阅读器
辅助技术的盲人使用。
玩家被要求对他们屏幕阅读器的回顾功能有很好的了解,
且需要知道如何水平和垂直地一个字符接一个字符移动。
他们还会发现他们屏幕阅读器的搜索功能是非常有价值的。
在游戏前务必仔细阅读这份指南,以便对屏幕布局是什么样的有所认识。
你还需要有能力定位 PC 光标。它总是在你的角色所在位置。
仅仅搜索一个{\tt @}符号无法总是找出你的角色,因为有
其他的\zhTransHumanoid(humanoid)使用相同的符号来表示。
你的屏幕阅读器应该还有给出你的回顾光标和 PC 光标的行和列的功能。
这些坐标经常是有用的,可以给玩家更好的对于屏幕上物品总体位置的感觉。
%.pg
%While it is not difficult for experienced users to edit the {\it defaults.nh\/}
%file to accomplish this, novices may find this task somewhat daunting.
%Included in all official distributions of NetHack is a file called
%{\it NHAccess.nh\/}.  Replacing {\it defaults.nh\/} with this file will cause
%the game to run in a manner accessible to the blind. After you have gained
%some experience with the game and with editing files, you may want to alter
%settings to better suit your preferences. Instructions on how to do this
%are included in the {\it NHAccess.nh\/} file itself. The most crucial
%settings to make the game accessible are:
虽然有经验的用户可以没有困难地编辑{\it defaults.nh\/}来达到这一点,
新手可能会发现这一任务多少有些令人气馁。
一个名为{\it NHAccess.nh\/}的文件被包括进了所有官方的{\it NetHack}分放中。
以这个文件替换{\it defaults.nh\/}会使游戏以盲人可以使用的方式运行。
在你对游戏及编辑文件获得一些经验后,你可能想要更改设置以更好地适合你的偏好。
关于如何做到如此的指导包含在了{\it NHAccess.nh\/}文件自己里。
使游戏容易使用的最关键设置为:
%.pg
\blist{}
%.lp
\item[\ib{IBMgraphics}]
%Disable IBMgraphics by commenting out this option.
(IBM 图形)注释该选项以禁用 IBMgraphics。
%.lp
\item[\ib{menustyle:traditional}]
%This will assist in the interface to speech synthesizers.
(菜单风格:传统)这会帮助界面到语音的合成器。
%.lp
\item[\ib{number\_pad}]
%A lot of speech access programs use the number-pad to review the screen.
%If this is the case, disable the number\_pad option and use the traditional
%Rogue-like commands.
(小键盘)大量语音辅助程序使用小键盘来回顾屏幕。
如果是这种情形,禁用{\it number\_pad}选项而使用传统的 Rogue-like 命令。
%.lp
\item[\ib{Character graphics}]
%Comment out all character graphics sets found near the bottom of the
%{\it defaults.nh\/} file.  Most of these replace {\it NetHack\/}'s
%default representation of the dungeon using standard ASCII characters
%with fancier characters from extended character sets, and these fancier
%characters can annoy screen-readers.
(字符图形)注释掉所有在{\it defaults.nh\/}文件底部附近的字符图形符。
这些中的大多数会将使用标准 ASCII 字符的{\it NetHack\/}默认\zhTransDungeon{}表示,
替换为扩展字符集中的更花俏的字符,而这些花俏的字符会干扰屏幕阅读器。
\elist

%.hn 1
%\section{Scoring}
\section{分数}

%.pg
%{\it NetHack\/} maintains a list of the top scores or scorers on your machine,
%depending on how it is set up.  In the latter case, each account on
%the machine can post only one non-winning score on this list.  If
%you score higher than someone else on this list, or better your
%previous score, you will be inserted in the proper place under your
%current name.  How many scores are kept can also be set up when
%{\it NetHack\/} is compiled.
{\it NetHack\/}在你的机器上维护一个最高分或最高分取得者的列表,
这取决于它是被如何设置的。
对于后一情形,机器上每一个帐户只能在该列表中张帖一个未获胜分数。
如果你的分数高于列表中其他某人,或比你以前的分数要好,
那么你会以你目前的名字插入到适当的位置。
有多少分数会保留同样可以在{\it NetHack\/}被编译时进行设置。

%.pg
%Your score is chiefly based upon how much experience you gained, how
%much loot you accumulated, how deep you explored, and how the game
%ended.  If you quit the game, you escape with all of your gold intact.
%If, however, you get killed in the Mazes of Menace, the guild will
%only hear about 90\,\% of your gold when your corpse is discovered
%(adventurers have been known to collect finder's fees).  So, consider
%whether you want to take one last hit at that monster and possibly
%live, or quit and stop with whatever you have.  If you quit, you keep
%all your gold, but if you swing and live, you might find more.
你的分数主要取决于你获得了多少经验、你积累了多少战利品、你探索到了多深
和游戏是如何结束的。
如果你退出游戏,你会完整地带着所有你的黄金逃跑。
然而,如果你在\zhTransMazesOfMenace{}(the Mazes of Menace)中被杀死,当发现你的尸体时
协会听说到的只是你黄金的 90\,\%(探险者已知会收取发现者费用)。
所以,考虑你想要最后再攻击那个怪物一次而可能活下来,
或者带着你拥有的退出并停止。
如果你退出了,你会保留你所有的黄金,不过如果你挥舞武器并活了下来,
你可能会发现更多。

%.pg
%If you just want to see what the current top players/games list is, you
%can type
如果你只想查看目前的最高分数玩家/游戏列表如何,你可以在大多数版本中输入
\begin{verbatim}
    nethack -s all
\end{verbatim}
%on most versions.

%.hn 1
%\section{Explore mode}
\section{探索模式}

%.pg
%{\it NetHack\/} is an intricate and difficult game.  Novices might falter
%in fear, aware of their ignorance of the means to survive.  Well, fear
%not.  Your dungeon may come equipped with an ``explore'' or ``discovery''
%mode that enables you to keep old save files and cheat death, at the
%paltry cost of not getting on the high score list.
{\it NetHack\/}是一个复杂而困难的游戏。
新手可能会因恐惧而迟疑,意识到他们对存活手段的无知。
好了,无需害怕。你的\zhTransDungeon{}可能配备了一个“探索”(explore)或“发现”(discovery)
模式,可以允许你保留旧的存档并逃脱死亡,微小的代价则为不会出现在高分列表中。

%.pg
%There are two ways of enabling explore mode.  One is to start the game
%with the {\tt -X}
%switch.  The other is to issue the `{\tt X}' command while already playing
%the game.  The other benefits of explore mode are left for the trepid
%reader to discover.
有两种方式可以启用探索模式。
一种是以 {\tt -X}选项启动游戏。另种是在已经进行游戏时使用“{\tt X}”命令。
其他探索模式的好处将留给惊恐的读者去发现。

%.hn
%\section{Credits}
\section{致谢}
%.pg
%The original %
%{\it hack\/} game was modeled on the Berkeley
%.ux
%UNIX
%{\it rogue\/} game.  Large portions of this paper were shamelessly
%cribbed from %
%{\it A Guide to the Dungeons of Doom}, by Michael C. Toy
%and Kenneth C. R. C. Arnold.  Small portions were adapted from
%{\it Further Exploration of the Dungeons of Doom}, by Ken Arromdee.
最初的{\it hack\/}游戏以 Berkeley UNIX {\it rogue\/}游戏为原型。
这份文档的大部分不知羞耻地抄袭自 Michael C. Toy 和 Kenneth C. R. C. Arnold 的
{\it A Guide to the Dungeons of Doom}(《\zhTransDungeonsOfDoom{}指南》)。
一小部分改写自 Ken Arromdee 的
{\it Further Exploration of the Dungeons of Doom}(《\zhTransDungeonsOfDoom{}的进一步探索》)。

%.pg
%{\it NetHack\/} is the product of literally dozens of people's work.
%Main events in the course of the game development are described below:
{\it NetHack\/}是毫不夸张的大量人工作的成果。
游戏开发的主要事件描述如下:

%.pg
\bigskip
%\nd {\it Jay Fenlason\/} wrote the original {\it Hack\/} with help from {\it
%Kenny Woodland}, {\it Mike Thome}, and {\it Jon Payne}.
{\it Jay Fenlason\/}在{\it Kenny Woodland}、{\it Mike Thome}和
{\it Jon Payne}的帮助下编写了原始的{\it Hack\/}。

%.pg
\medskip
%\nd {\it Andries Brouwer\/} did a major re-write, transforming {\it Hack\/}
%into a very different game, and published (at least) three versions (1.0.1,
%1.0.2, and 1.0.3) for UNIX machines to the Usenet.
{\it Andries Brouwer\/}进行了大程度的重写,将{\it Hack\/}转变为一个非常不
同的游戏,并向新闻组(Usenet)为 UNIX 机器
发布了(至少)三个版本(1.0.1、1.0.2 和 1.0.3)。

%.pg
\medskip
%\nd {\it Don G. Kneller\/} ported {\it Hack\/} 1.0.3 to Microsoft C and MS-DOS,
%producing {\it PC Hack\/} 1.01e, added support for DEC Rainbow graphics in
%version 1.03g, and went on to produce at least four more versions (3.0, 3.2,
%3.51, and 3.6).
{\it Don G. Kneller\/}将{\it Hack\/} 1.0.3 移植到了 Microsoft C 和 MS-DOS,
产生{\it PC Hack\/} 1.01e,还在版本 1.03g 为 DEC Rainbow 图形增加支持,
并且继续制作了至少四个新版本(3.0、3.2、3.51 和 3.6)。

%.pg
\medskip
%\nd {\it R. Black\/} ported {\it PC Hack\/} 3.51 to Lattice C and the Atari
%520/1040ST, producing {\it ST Hack\/} 1.03.
{\it R. Black\/}将{\it PC Hack\/} 3.51 移植到 Lattice C 和 Atari 
520/1040ST,制作出{\it ST Hack\/} 1.03。

%.pg
\medskip
%\nd {\it Mike Stephenson\/} merged these various versions back together,
%incorporating many of the added features, and produced {\it NetHack\/} version
%1.4.  He then coordinated a cast of thousands in enhancing and debugging
%{\it NetHack\/} 1.4 and released {\it NetHack\/} versions 2.2 and 2.3.
{\it Mike Stephenson\/}将这些各种版本重新合并到一起,吸收了许多增加的特性,
制作出{\it NetHack\/}版本 1.4。
接着,他协调一个上千人的团队对{\it NetHack\/} 1.4 进行增强和调试,并发布了
{\it NetHack\/}版本 2.2 和 2.3。

%.pg
\medskip
%\nd Later, Mike coordinated a major rewrite of the game, heading a team which
%included {\it Ken Arromdee}, {\it Jean-Christophe Collet}, {\it Steve Creps},
%{\it Eric Hendrickson}, {\it Izchak Miller}, {\it Eric S. Raymond}, {\it John
%Rupley}, {\it Mike Threepoint}, and {\it Janet Walz}, to produce {\it
%NetHack\/} 3.0c.
随后,{\it Mike}领导一个团队协调进行了游戏的较大重写,
制作出{\it NetHack\/} 3.0c。该团队包括{\it Ken Arromdee}、
{\it Jean-Christophe Collet}、{\it Steve Creps}、{\it Eric Hendrickson}、
{\it Izchak Miller}、{\it Eric S. Raymond}、{\it John Rupley}、
{\it Mike Threepoint}和{\it Janet Walz}。

%.pg
\medskip
%\nd {\it NetHack\/} 3.0 was ported to the Atari by {\it Eric R. Smith}, to OS/2 by
%{\it Timo Hakulinen}, and to VMS by {\it David Gentzel}.  The three of them
%and {\it Kevin Darcy\/} later joined the main development team to produce
%subsequent revisions of 3.0.
{\it NetHack\/} 3.0 由{\it Eric R. Smith}移植到 Atari,
由{\it Timo Hakulinen}移植到 OS/2,以及由{\it David Gentzel} 移植到 VMS。
他们三人与{\it Kevin Darcy\/}后来加入了主要开发团队制作后续的 3.0 修订版。

%.pg
\medskip
%\nd {\it Olaf Seibert\/} ported {\it NetHack\/} 2.3 and 3.0 to the Amiga.  {\it
%Norm Meluch}, {\it Stephen Spackman\/} and {\it Pierre Martineau\/} designed
%overlay code for {\it PC NetHack\/} 3.0.  {\it Johnny Lee\/} ported {\it
%NetHack\/} 3.0 to the Macintosh.  Along with various other Dungeoneers, they
%continued to enhance the PC, Macintosh, and Amiga ports through the later
%revisions of 3.0.
{\it Olaf Seibert\/}移植{\it NetHack\/} 2.3 和 3.0 到 Amiga。
{\it Norm Meluch}、{\it Stephen Spackman\/}和{\it Pierre Martineau\/}
为{\it PC NetHack\/} 3.0 设计了覆盖图代码。
{\it Johnny Lee\/}移植{\it NetHack\/} 3.0 到 Macintosh。
与许许多多\zhTransDungeoneer{}(Dungeoneer)一起,他们在后来的 3.0 的修订版中继续为
PC、Macintosh 和 Amiga 移植作增强。

%.pg
\medskip
%\nd Headed by {\it Mike Stephenson\/} and coordinated by {\it Izchak Miller\/} and
%{\it Janet Walz}, the development team which now included {\it Ken Arromdee},
%{\it David Cohrs}, {\it Jean-Christophe Collet}, {\it Kevin Darcy},
%{\it Matt Day}, {\it Timo Hakulinen}, {\it Steve Linhart}, {\it Dean Luick},
%{\it Pat Rankin}, {\it Eric Raymond}, and {\it Eric Smith\/} undertook a radical
%revision of 3.0.  They re-structured the game's design, and re-wrote major
%parts of the code.  They added multiple dungeons, a new display, special
%individual character quests, a new endgame and many other new features, and
%produced {\it NetHack\/} 3.1.
由{\it Mike Stephenson\/}领导并由{\it Izchak Miller\/}和{\it Janet Walz}
协调,现在包含了{\it Ken Arromdee}、{\it David Cohrs}、
{\it Jean-Christophe Collet}、{\it Kevin Darcy}、{\it Matt Day}、
{\it Timo Hakulinen}、{\it Steve Linhart}、{\it Dean Luick}、{\it Pat Rankin}、
{\it Eric Raymond}和{\it Eric Smith\/}的开发团队进行了对 3.0 的激进修订。
他们重建了游戏的设计,并重写了代码的主要部分。
他们增加了多个\zhTransDungeon{}、一个新的显示、特殊的单独角色\zhTransQuest(quest)、
一个新的\zhTransEndgame(endgame)和许多其他新特性,
制作成了{\it NetHack\/} 3.1。

%.pg
\medskip
%\nd {\it Ken Lorber}, {\it Gregg Wonderly\/} and {\it Greg Olson}, with help
%from {\it Richard Addison}, {\it Mike Passaretti}, and {\it Olaf Seibert},
%developed {\it NetHack\/} 3.1 for the Amiga.
{\it Ken Lorber}、{\it Gregg Wonderly\/}和{\it Greg Olson},
在来自{\it Richard Addison}、{\it Mike Passaretti}和{\it Olaf Seibert}的帮助下,
为 Amiga 开发{\it NetHack\/} 3.1。

%.pg
\medskip
%\nd {\it Norm Meluch\/} and {\it Kevin Smolkowski}, with help from
%{\it Carl Schelin}, {\it Stephen Spackman}, {\it Steve VanDevender},
%and {\it Paul Winner}, ported {\it NetHack\/} 3.1 to the PC.
{\it Norm Meluch\/}和{\it Kevin Smolkowski},在来自{\it Carl Schelin}、
{\it Stephen Spackman}、{\it Steve VanDevender}和{\it Paul Winner}的帮助下,
将{\it NetHack\/} 3.1 移植到 PC。

%.pg
\medskip
%\nd {\it Jon W\{tte} and {\it Hao-yang Wang},
%with help from {\it Ross Brown}, {\it Mike Engber}, {\it David Hairston},
%{\it Michael Hamel}, {\it Jonathan Handler}, {\it Johnny Lee},
%{\it Tim Lennan}, {\it Rob Menke}, and {\it Andy Swanson},
%developed {\it NetHack\/} 3.1 for the Macintosh, porting it for MPW.
%Building on their development, {\it Barton House} added a Think C port.
{\it Jon Wätte}和{\it Hao-yang Wang},在来自{\it Ross Brown}、
{\it Mike Engber}、{\it David Hairston}、{\it Michael Hamel}、
{\it Jonathan Handler}、{\it Johnny Lee}、{\it Tim Lennan}、{\it Rob Menke}
和{\it Andy Swanson}的帮助下,为 Macintosh 开发{\it NetHack\/} 3.1,
将它移植到 MPW。基于他们的开发,{\it Barton House}增加了 Think C 移植。

%.pg
\medskip
%\nd {\it Timo Hakulinen\/} ported {\it NetHack\/} 3.1 to OS/2.
%{\it Eric Smith\/} ported {\it NetHack\/} 3.1 to the Atari.
%{\it Pat Rankin}, with help from {\it Joshua Delahunty},
%was responsible for the VMS version of {\it NetHack\/} 3.1.
%{\it Michael Allison} ported {\it NetHack\/} 3.1 to Windows NT.
{\it Timo Hakulinen\/}将{\it NetHack\/} 3.1 移植到 OS/2。
{\it Eric Smith\/}将{\it NetHack\/} 3.1 移植到 Atari。
{\it Pat Rankin}在来自{\it Joshua Delahunty}的帮助下,
承担了{\it NetHack\/} 3.1 的 VMS 版本。
{\it Michael Allison}将{\it NetHack\/} 3.1 移植到 Windows NT。

%.pg
\medskip
%\nd {\it Dean Luick}, with help from {\it David Cohrs}, developed {\it NetHack\/}
%3.1 for X11.
%{\it Warwick Allison} wrote a tiled version of NetHack for the Atari;
%he later contributed the tiles to the DevTeam and tile support was
%then added to other platforms.
{\it Dean Luick}在来自{\it David Cohrs}的帮助下,
为 X11 开发{\it NetHack\/} 3.1。
{\it Warwick Allison}为 Atari 编写了{\it NetHack\/}的一个瓦片(tile)版本;
他随后将这些瓦片贡献给 DevTeam(开发团队),之后瓦片支持被加入到其他平台。

%.pg
\medskip
%\nd The 3.2 development team, comprised of {\it Michael Allison}, {\it Ken
%Arromdee}, {\it David Cohrs}, {\it Jessie Collet}, {\it Steve Creps}, {\it
%Kevin Darcy}, {\it Timo Hakulinen}, {\it Steve Linhart}, {\it Dean Luick},
%{\it Pat Rankin}, {\it Eric Smith}, {\it Mike Stephenson}, {\it Janet Walz},
%and {\it Paul Winner}, released version 3.2 in April of 1996.
包括了{\it Michael Allison}、{\it Ken Arromdee}、{\it David Cohrs}、
{\it Jessie Collet}、{\it Steve Creps}、{\it Kevin Darcy}、
{\it Timo Hakulinen}、{\it Steve Linhart}、{\it Dean Luick}、{\it Pat Rankin}、
{\it Eric Smith}、{\it Mike Stephenson}、{\it Janet Walz}和{\it Paul Winner}
的 3.2 开发团队于 1996 年四月发布了版本 3.2。

%.pg
\medskip
%\nd Version 3.2 marked the tenth anniversary of the formation of the development
%team.  In a testament to their dedication to the game, all thirteen members
%of the original development team remained on the team at the start of work on
%that release.  During the interval between the release of 3.1.3 and 3.2,
%one of the founding members of the development team, {\it Dr. Izchak Miller},
%was diagnosed with cancer and passed away.  That release of the game was
%dedicated to him by the development and porting teams.
版本 3.2 纪念了开发团队形成十周年。
作为一个他们专注于游戏的证明,所有十三名原来开发团队的成员在这次发布的
工作开始时,仍留在团队里。在 3.1.3 和 3.2 发布之间,
一位开发团队的创立成员 {\it Izchak Miller}博士被诊断出癌症并最终去世。
开发和移植团队将这一游戏版本献给了他。

%.pg
\medskip
%During the lifespan of {\it NetHack\/} 3.1 and 3.2, several enthusiasts
%of the game added
%their own modifications to the game and made these ``variants'' publicly
%available:
在{\it NetHack\/} 3.1 和 3.2 生存期间,数位游戏爱好者向游戏中加入了他们自己
的修改,制作出了这些可以公开获取的“变体”(variant):

%.pg
\medskip
%{\it Tom Proudfoot} and {\it Yuval Oren} created {\it NetHack++},
%which was quickly renamed {\it NetHack$--$}.
%Working independently, {\it Stephen White} wrote {\it NetHack Plus}.
%{\it Tom Proudfoot} later merged {\it NetHack Plus}
%and his own {\it NetHack$--$} to produce {\it SLASH}.
%{\it Larry Stewart-Zerba} and {\it Warwick Allison} improved the spell
%casting system with the Wizard Patch.
%{\it Warwick Allison} also ported NetHack to use the Qt interface.
{\it Tom Proudfoot}和{\it Yuval Oren}创造了{\it NetHack++},
很快又将其重新命名为{\it NetHack$--$}。
{\it Stephen White}独立编写了{\it NetHack Plus}。
{\it Tom Proudfoot}随后将{\it NetHack Plus}和他自己的{\it NetHack$--$}合并,
制作出{\it SLASH}。
{\it Larry Stewart-Zerba}和{\it Warwick Allison}使用 Wizard Patch(巫师补丁)
改进了咒语施放系统。
{\it Warwick Allison}还使用 Qt 界面移植了{\it NetHack}。

%.pg
\medskip
%{\it Warren Cheung} combined {\it SLASH} with the Wizard Patch
%to produce {\it Slash'em\/}, and
%with the help of {\it Kevin Hugo}, added more features.
%Kevin later joined the
%DevTeam and incorporated the best of these ideas into NetHack 3.3.
{\it Warren Cheung}将{\it SLASH}与 Wizard Patch 合并成{\it Slash'em\/},
并在{\it Kevin Hugo}的帮助下,增加了更多特性。
{\it Kevin}后来加入了 DevTeam,并将这些创意中的最好部分加入到了
{\it NetHack} 3.3。

%.pg
\medskip
%The final update to 3.2 was the bug fix release 3.2.3, which was released
%simultaneously with 3.3.0 in December 1999 just in time for the Year 2000.
对 3.2 的最后更新是错误修正版本 3.2.3,后者与 3.3.0 同时释放于 1999 年十二月,
恰好赶在 2000 年之前。

%.pg
\medskip
%The 3.3 development team, consisting of {\it Michael Allison}, {\it Ken Arromdee}, 
%{\it David Cohrs}, {\it Jessie Collet}, {\it Steve Creps}, {\it Kevin Darcy}, 
%{\it Timo Hakulinen}, {\it Kevin Hugo}, {\it Steve Linhart}, {\it Ken Lorber}, 
%{\it Dean Luick}, {\it Pat Rankin}, {\it Eric Smith}, {\it Mike Stephenson}, 
%{\it Janet Walz}, and {\it Paul Winner}, released 3.3.0 in 
%December 1999 and 3.3.1 in August of 2000.
由{\it Michael Allison}、{\it Ken Arromdee}、{\it David Cohrs}、
{\it Jessie Collet}、{\it Steve Creps}、{\it Kevin Darcy}、
{\it Timo Hakulinen}、{\it Kevin Hugo}、{\it Steve Linhart}、{\it Ken Lorber}、
{\it Dean Luick}、{\it Pat Rankin}、{\it Eric Smith}、{\it Mike Stephenson}、
{\it Janet Walz}和{\it Paul Winner}组成的 3.3 开发团队,于 1999 年十二月发布了
3.3.0,于 2000 年八月发布了 3.3.1。

%.pg
\medskip
%Version 3.3 offered many firsts. It was the first version to separate race 
%and profession. The Elf class was removed in preference to an elf race, 
%and the races of dwarves, gnomes, and orcs made their first appearance in 
%the game alongside the familiar human race.  Monk and Ranger roles joined 
%Archeologists, Barbarians, Cavemen, Healers, Knights, Priests, Rogues, Samurai, 
%Tourists, Valkyries and of course, Wizards.  It was also the first version
%to allow you to ride a steed, and was the first version to have a publicly 
%available web-site listing all the bugs that had been discovered.  Despite 
%that constantly growing bug list, 3.3 proved stable enough to last for
%more than a year and a half.
版本 3.3 中出现了许多第一次。
它是将种族和职业分开的第一个版本。
\zhTransElves{}类别由于\zhTransElves{}种族而被移除,
\zhTransDwarves、\zhTransGnomes{}和\zhTransOrcs{}种族首次与
熟悉的\zhTransHumans{}种族一起出现在游戏中。
\zhTransMonks{}和\zhTransRangers{}职业加入\zhTransArcheologists、
\zhTransBarbarians、\zhTransCavemen、\zhTransHealers、\zhTransKnights、
\zhTransPriests、\zhTransRogues、\zhTransSamurai、\zhTransTourists、
\zhTransValkyries{}以及当然的\zhTransWizards。
它还是第一个允许你骑一匹马的版本,以及第一个有一个可以公开访问的网站的版本,
该网站列出了所有已经发现的错误。
尽管错误列表经常增长,3.3 被证明足够稳定以持续一年半。

%.pg
\medskip
%The 3.4 development team initially consisted of 
%{\it Michael Allison}, {\it Ken Arromdee},
%{\it David Cohrs}, {\it Jessie Collet}, {\it Kevin Hugo}, {\it Ken Lorber},
%{\it Dean Luick}, {\it Pat Rankin}, {\it Mike Stephenson}, 
%{\it Janet Walz}, and {\it Paul Winner}, with {\it  Warwick Allison} joining 
%just before the release of NetHack 3.4.0 in March 2002.
3.4 开发团队最初包括{\it Michael Allison}、{\it Ken Arromdee}、
{\it David Cohrs}、{\it Jessie Collet}、{\it Kevin Hugo}、{\it Ken Lorber}、
{\it Dean Luick}、{\it Pat Rankin}、{\it Mike Stephenson}、{\it Janet Walz}
和{\it Paul Winner},{\it  Warwick Allison}恰好在{\it NetHack} 3.4.0 于
2002 年三月发布前加入。

%.pg
\medskip
%As with version 3.3, various people contributed to the game as a whole as
%well as supporting ports on the different platforms that {\it NetHack\/}
%runs on:
与版本 3.3 一样,许多人为游戏作出了贡献,总体上也为{\it NetHack\/}运行的不同平台
上的支持移植作了贡献:

%.pg
\medskip
%\nd{\it Pat Rankin} maintained 3.4 for VMS.
{\it Pat Rankin}为 VMS 维护 3.4。

%.pg
\medskip
%\nd {\it Michael Allison} maintained NetHack 3.4 for the MS-DOS platform.
%{\it Paul Winner} and {\it Yitzhak Sapir} provided encouragement.
{\it Michael Allison}为 MS-DOS 平台维护{\it NetHack\/} 3.4。
{\it Paul Winner}和{\it Yitzhak Sapir}提供了鼓励。

%.pg
\medskip
%\nd {\it Dean Luick}, {\it Mark Modrall}, and {\it Kevin Hugo} maintained and
%enhanced the Macintosh port of 3.4.
{\it Dean Luick}、{\it Mark Modrall}和{\it Kevin Hugo}维护和增强
3.4 的 Macintosh 移植。

%.pg
\medskip
%\nd {\it Michael Allison}, {\it David Cohrs}, {\it Alex Kompel}, {\it Dion Nicolaas}, and 
%{\it Yitzhak Sapir} maintained and enhanced 3.4 for the Microsoft Windows platform.
%{\it Alex Kompel} contributed a new graphical interface for the Windows port. 
%{\it Alex Kompel} also contributed a Windows CE port for 3.4.1.
{\it Michael Allison}、{\it David Cohrs}、{\it Alex Kompel}、
{\it Dion Nicolaas}和{\it Yitzhak Sapir}为 Microsoft Windows 平台维护和增强
3.4。{\it Alex Kompel}为 Windows 移植贡献了一个新的图形界面。
{\it Alex Kompel}还为 3.4.1 作了 Windows CE 移植。

%.pg
\medskip
%\nd {\it Ron Van Iwaarden} maintained 3.4 for OS/2.
{\it Ron Van Iwaarden}为 OS/2 维护 3.4。

%.pg
\medskip
%\nd {\it Janne Salmij\"{a}rvi} and {\it Teemu Suikki} maintained
%and enhanced the Amiga port of 3.4 after {\it Janne Salmij\"{a}rvi} resurrected
%it for 3.3.1.
{\it Janne Salmij\"{a}rvi}和{\it Teemu Suikki}在{\it Janne Salmij\"{a}rvi}
将 3.3.1 的 Amiga 移植复活后对 3.4 作维护和增强。

%.pg
\medskip
%\nd {\it Christian ``Marvin'' Bressler} maintained 3.4 for the Atari after he
%resurrected it for 3.3.1.
{\it Christian ``Marvin'' Bressler}在他将 Atari 的 3.3.1 复活后,
为其维护 3.4。

%.pg
\medskip
%\nd There is a NetHack web site maintained by {\it Ken Lorber} at 
%http:{\tt /}{\tt /}www.nethack.org{\tt /}.
在 \url{http://www.nethack.org/} 有一个由{\it Ken Lorber}维护的
{\it NetHack}网站。

%.pg
\bigskip
%\nd From time to time, some depraved individual out there in netland sends a
%particularly intriguing modification to help out with the game.  The Gods of
%the Dungeon sometimes make note of the names of the worst of these miscreants
%in this, the list of Dungeoneers:
不时地,一些网络世界上堕落的人发送来一份特别有趣的修改,以帮助这个游戏。
\zhTransDungeon{}的众神(The Gods of the Dungeon)有时会将这些恶人中最坏的人的名字记录到
这份\zhTransDungeoneer{}名单(the list of Dungeoneers)中:

%.sd
\begin{center}
\begin{tabular}{lll}
%TABLE_START
Adam Aronow & Izchak Miller & Mike Stephenson\\
Alex Kompel & J. Ali Harlow & Norm Meluch\\
Andreas Dorn & Janet Walz & Olaf Seibert\\
Andy Church & Janne Salmij\"{a}rvi & Pasi Kallinen\\
Andy Swanson & Jean-Christophe Collet & Pat Rankin\\
Ari Huttunen & Jochen Erwied & Paul Winner\\
Barton House & John Kallen & Pierre Martineau\\
Benson I. Margulies & John Rupley & Ralf Brown\\
Bill Dyer & John S. Bien & Ray Chason\\
Boudewijn Waijers & Johnny Lee & Richard Addison\\
%Bruce Cox & Jon W\{tte & Richard Beigel\\
Bruce Cox & Jon Wätte & Richard Beigel\\
Bruce Holloway & Jonathan Handler & Richard P. Hughey\\
Bruce Mewborne & Joshua Delahunty & Rob Menke\\
Carl Schelin & Keizo Yamamoto & Robin Johnson\\
Chris Russo & Ken Arnold & Roderick Schertler\\
David Cohrs & Ken Arromdee & Roland McGrath\\
David Damerell & Ken Lorber & Ron Van Iwaarden\\
David Gentzel & Ken Washikita & Ronnen Miller\\
David Hairston & Kevin Darcy & Ross Brown\\
Dean Luick & Kevin Hugo & Sascha Wostmann\\
Del Lamb & Kevin Sitze & Scott Bigham\\
Deron Meranda & Kevin Smolkowski & Scott R. Turner\\
Dion Nicolaas & Kevin Sweet & Stephen Spackman\\
Dylan O'Donnell & Lars Huttar & Stephen White\\
Eric Backus & Malcolm Ryan & Steve Creps\\
Eric Hendrickson & Mark Gooderum & Steve Linhart\\
Eric R. Smith & Mark Modrall & Steve VanDevender\\
Eric S. Raymond & Marvin Bressler & Teemu Suikki\\
Erik Andersen & Matthew Day & Tim Lennan\\
Frederick Roeber & Merlyn LeRoy & Timo Hakulinen\\
Gil Neiger & Michael Allison & Tom Almy\\
Greg Laskin & Michael Feir & Tom West\\
Greg Olson & Michael Hamel & Warren Cheung\\
Gregg Wonderly & Michael Sokolov & Warwick Allison\\
Hao-yang Wang & Mike Engber & Yitzhak Sapir\\
Helge Hafting & Mike Gallop\\
Irina Rempt-Drijfhout & Mike Passaretti
%TABLE_END  Do not delete this line.
\end{tabular}
\end{center}
%.ed

%\vfill
%\begin{flushleft}
%\small
%Microsoft and MS-DOS are registered trademarks of Microsoft Corporation.\\
%%%Don't need next line if a UNIX macro automatically inserts footnotes.
%UNIX is a registered trademark of AT\&T.\\
%Lattice is a trademark of Lattice, Inc.\\
%Atari and 1040ST are trademarks of Atari, Inc.\\
%AMIGA is a trademark of Commodore-Amiga, Inc.\\
%%.sm
%Brand and product names are trademarks or registered trademarks
%of their respective holders.
%\end{flushleft}

\newpage

\appendix
% Now here are stuffs added by the translator. 
以下部分为附录,均由翻译者添加。
附录中包括了对本文档的说明、译名对照表以及{\it NetHack General Public License}。

\section{关于本文档}
本文档翻译自{\it Eric S. Raymond}等人编写的官方{\it NetHack} Guidebook。
原标题为“A Guide to the Mazes of Menace: Guidebook for {\it NetHack}”。
翻译由{\it Roy Clark}(kralcyor)于 2015 年完成。
此翻译作品并非由 DevTeam 发布,且翻译者与 DevTeam 没有关系。

原文档作为{\it NetHack}的一部分,使用{\it NetHack General Public License}
(NGPL)许可。本文档使用相同的协议。
{\it NetHack General Public License}可以在本文档的附录
\ref{appendix:NGPL} 找到。
鉴于 NPGL 本身版权限制,翻译者无法对其进行翻译。
简要地说,你拥有依照协议自由地对{\it NetHack}(或本文档)获取、
分发和修改等的权利,条件为你修改的{\it NetHack}(或本文档)不能限制他人获得这些
协议中赋予的权利。具体使用(尤其在分发修改版本时)请参考和遵照 NPGL 条款。

本文档完全没有担保,不会对你在\zhTransDungeon{}世界里面或者外面发生的任何事情负责。

本文档意在帮助有一定英语基础,但对英文阅读有点吃力或者不希望大量阅读的玩家,
阅读 Guidebook 和进行原版游戏。
所以,(如你所见)本文档(有点不厌其烦地)大量地以在括号中备注的形式,
将原始英语单词标注出来。
另外,在附录 \ref{appedix:En-Zh} 中还提供了译名对照表,以供参考。

本文档使用\LaTeX 写作,源文件可以从
\url{http://kralcyor.info/nethack/guidebook-zh/src/} 获取。
%编译文本档所用命令为
%{\tt xelatex -papersize=a4 -jobname=nethack-Guidebook-zh_v0.1 guidebook-zh.tex }。
本文档的其他格式文件可以从
\url{http://kralcyor.info/nethack/guidebook-zh/} 获取。

如果你发现了翻译错误,或者有建议或问题,可以通过
\href{mailto:kralcyor@kralcyor.info}{kralcyor@kralcyor.info}
与翻译者联系。

本文档的版本为 0.1,对应于{\it NetHack} 3.4.3 版本中的 Guidebook。

依照 NPGL,任何人获取和使用本文档不会被收取费用。
不过,如果觉得本文档对你有用,你可以考虑对翻译者进行捐助。
有关向翻译者捐助的信息可以在 \url{http://kralcyor.info/donation.html} 找到。

\twocolumn
\begin{strip}
\section{译名对照表}
\label{appedix:En-Zh}
\end{strip}

\begin{supertabular}{ll}
% This file is available under NetHack General Public License. 
% The License is included in the file `license'. 
% This file is generated from zhTrans-template by generate_zhTrans_snippets.sh
% You may not modify this file directly. 
% Sun Nov 22 17:30:35 CST 2015

air	&	空气	\\
akly	&	阿克利斯标枪	\\
Alignment	&	阵营	\\
altar	&	祭坛	\\
the Amulet of Yendor	&	壬铎护身符	\\
amulet	&	护身符	\\
angelic being	&	类天使生物	\\
anti-magic field	&	反魔法力场	\\
ant	&	蚂蚁	\\
apelike creature	&	类猿生物	\\
arachnid	&	蛛形纲动物	\\
Archeologist	&	考古学家	\\
Armor Class	&	护甲等级	\\
armor	&	护甲	\\
arrow	&	箭	\\
arrow trap	&	箭陷阱	\\
artifact	&	神器	\\
attribute	&	属性	\\
banded mail	&	带链甲	\\
Barbarian	&	野蛮人	\\
basic	&	基本的	\\
bat	&	蝙蝠	\\
beam	&	光线	\\
bear trap	&	捕熊器	\\
beast	&	动物	\\
bec-de-corbin	&	渡鸦嘴战锤	\\
bird	&	鸟类	\\
black pudding	&	黑布丁	\\
blessed	&	被祝福的	\\
blessing	&	祝福	\\
blind	&	失明	\\
blob	&	液滴	\\
Bones	&	尸骨	\\
bonuse	&	加成	\\
boomerang	&	回飞镖	\\
boulder	&	巨石	\\
bow	&	弓	\\
broadsword	&	大刀	\\
bronze plate mail	&	黄铜板链甲	\\
Burdened	&	负重的	\\
cancel	&	作废	\\
candy bar	&	方糖块	\\
Caveman	&	穴居人	\\
centaur	&	半人马	\\
centipede	&	蜈蚣	\\
chain mail	&	锁子甲	\\
Chaotic	&	混乱	\\
charge	&	能量	\\
Charisma	&	魅力值	\\
cloud	&	云	\\
cockatrice	&	鸡蛇	\\
computer fantasy game	&	电脑幻想游戏	\\
conduct	&	品行	\\
Confused	&	迷糊	\\
Constitution	&	体格值	\\
corner	&	墙角	\\
corridor	&	走道	\\
cram ration	&	克元配给	\\
C-ration	&	C 配给	\\
cream pie	&	奶油派	\\
credit card	&	信用卡	\\
credit	&	信用量	\\
crossbow bolt	&	驽箭	\\
crossbow	&	驽	\\
crystal plate mail	&	水晶板链甲	\\
cursed	&	被诅咒的	\\
curse	&	诅咒	\\
dagger	&	匕首	\\
dai-sho	&	大小	\\
dart trap	&	飞镖陷阱	\\
demons	&	恶魔	\\
Dexterity	&	敏捷值	\\
Divine intervention	&	神的干预	\\
dog	&	狗	\\
doorway	&	门口	\\
door	&	门	\\
dragon scale mail	&	龙鳞甲	\\
dragon	&	龙	\\
drawbridge	&	吊桥	\\
dungeoneer	&	地牢探索者	\\
Dungeon Level	&	地牢层数	\\
the Dungeons of Doom	&	毁灭地牢	\\
dungeon	&	地牢	\\
Dwarf	&	矮人	\\
dwarvish mithril-coat	&	矮人秘银衣	\\
elemental	&	元素	\\
elven mithril-coat	&	精灵秘银衣	\\
Elf	&	精灵	\\
enchanted	&	被施过法术的	\\
enchantment	&	魔法值	\\
endgame	&	最终试炼	\\
experience level	&	经验等级	\\
experience point	&	经验值	\\
Experience	&	经验	\\
expert	&	内行的	\\
eye	&	眼	\\
Fainting	&	昏晕	\\
falling rock trap	&	落石陷阱	\\
feline	&	猫科动物	\\
fire trap	&	烈火陷阱	\\
floor	&	地面	\\
food ration	&	食物配给	\\
food	&	食物	\\
fortune cookie	&	幸运饼干	\\
fountain	&	泉	\\
fungi	&	真菌	\\
gem	&	宝石	\\
genocide	&	灭绝	\\
ghost	&	鬼魂	\\
giant humanoid	&	巨型类人生物	\\
giant	&	巨人	\\
Gnome	&	侏儒	\\
Gnomish Mines	&	侏儒矿坑	\\
Gold	&	黄金	\\
golem	&	魔像	\\
grand master	&	特级大师	\\
grave	&	坟墓	\\
gremlin	&	小妖精	\\
Healer	&	医治师	\\
Hit Point	&	生命值	\\
hole	&	洞	\\
holy water	&	圣水	\\
horse	&	马	\\
humanoid	&	类人生物	\\
Human	&	人类	\\
Hunger Status	&	饥饿状态	\\
ice box	&	冰箱	\\
ice	&	冰	\\
imp	&	小恶魔	\\
Intelligence	&	智力值	\\
inventory	&	清单	\\
invisible	&	隐身的	\\
iron bar	&	铁条	\\
jabberwock	&	甲伯沃基	\\
jellies	&	胶冻	\\
Keystone Kop	&	启斯东警察	\\
Knight	&	骑士	\\
kobold	&	狗头人	\\
K-ration	&	K 配给	\\
ladder	&	梯子	\\
land mine	&	地雷	\\
lava	&	熔岩	\\
Lawful	&	守序	\\
leather armor	&	皮甲	\\
leather jacket	&	皮夹克	\\
lembas wafer	&	兰巴斯饼干	\\
leprechaun	&	矮妖精	\\
level teleporter	&	层间瞬移陷阱	\\
lich	&	巫妖	\\
light	&	光	\\
little dog	&	小狗	\\
lizard	&	蜥蜴	\\
long worm	&	长蠕虫	\\
lucern hammer	&	琉森锤	\\
Luck	&	幸运值	\\
lump of royal jelly	&	蜂王浆团块	\\
lurker above	&	顶部潜伏者	\\
mace	&	狼牙棒	\\
magic portal	&	魔法入口	\\
magic resistance	&	魔法免疫	\\
magic trap	&	魔法陷阱	\\
mail daemon	&	邮件精灵	\\
mana	&	玛那	\\
martial arts	&	战争艺术	\\
master	&	大师	\\
MAUD	&	毛德	\\
the Mazes of Menace	&	恐吓迷宫	\\
mimic	&	拟物怪	\\
mind flayer	&	夺心魔	\\
minor demon	&	低级恶魔	\\
moat	&	护城河	\\
mold	&	霉菌	\\
Monk	&	僧侣	\\
mummy	&	木乃伊	\\
naga	&	那伽	\\
Neutral	&	中立	\\
Nippon	&	日本	\\
none	&	没有的	\\
Northland	&	北国	\\
nymph	&	仙女	\\
ooze	&	泥浆	\\
the Oracle of Delphi	&	德尔斐的祭司	\\
orcish chain mail	&	半兽人锁子甲	\\
orcish ring mail	&	半兽人环甲	\\
Orc	&	半兽人	\\
ogre	&	食人魔	\\
Overloaded	&	过载的	\\
Overtaxed	&	负担过重的	\\
pancake	&	薄煎饼	\\
pets	&	宠物	\\
pick-axe	&	镐	\\
piercer	&	锥状怪	\\
pit	&	坑	\\
plate mail	&	板链甲	\\
polearm	&	长兵器	\\
polymorph trap	&	变形陷阱	\\
Polymorph	&	变形	\\
pony	&	矮种马	\\
pool	&	水池	\\
potion of polymorph	&	变形药水	\\
potion	&	药水	\\
Power	&	能量	\\
Priest	&	祭司	\\
pudding	&	布丁	\\
quadruped	&	四足动物	\\
quantum mechanic	&	量子力学	\\
quest	&	任务	\\
Ranger	&	漫游者	\\
Rank	&	级别	\\
restricted	&	受限制的	\\
ring mail	&	环甲	\\
ring	&	戒指	\\
rodent	&	啮齿动物	\\
Rogue	&	流氓	\\
rolling boulder trap	&	滚动巨石陷阱	\\
rust monster	&	锈蚀怪	\\
rust trap	&	锈蚀陷阱	\\
Samurai	&	武士	\\
Satiated	&	饱足	\\
scale mail	&	鳞甲	\\
scroll of identify	&	鉴定卷轴	\\
scroll of mail	&	邮件卷轴	\\
scroll	&	卷轴	\\
sea monster	&	海怪	\\
shield	&	盾	\\
shopkeeper	&	黑店老板	\\
shop	&	黑店	\\
shortest-path algorithm	&	最短路径算法	\\
shortest path	&	最短路径	\\
sink	&	水池	\\
skilled	&	熟练的	\\
sleeping gas trap	&	催眠陷阱	\\
slime mold	&	黏菌	\\
sling	&	投石器	\\
snake	&	蛇	\\
Sokoban	&	推箱子	\\
solid rock	&	坚硬的石头	\\
spear	&	矛	\\
spellbook	&	咒语书	\\
sphere	&	球	\\
spiked pit	&	尖刺坑	\\
splint mail	&	条板甲	\\
squeaky board	&	发出吱吱声的地板	\\
staff	&	棍棒	\\
stair	&	楼梯	\\
statue trap	&	雕像陷阱	\\
statue	&	雕像	\\
steeds	&	坐骑	\\
Strained	&	勉强的	\\
Strength	&	力量值	\\
Stressed	&	受重压的	\\
studded leather armor	&	带钉皮甲	\\
Stunned	&	失去平衡	\\
sword	&	剑	\\
teleportation trap	&	瞬移陷阱	\\
text adventure games	&	文字冒险游戏	\\
throne	&	王座	\\
Time	&	时间	\\
tin	&	罐头	\\
tool	&	工具	\\
Tourist	&	游客	\\
trap door	&	陷阱门	\\
trapper	&	陷阱怪	\\
trap	&	陷阱	\\
tree	&	树	\\
tripe ration	&	肚儿配给	\\
troll	&	巨怪	\\
two-handed sword	&	双手剑	\\
umber hulk	&	土巨怪	\\
Unburdened	&	不负重的	\\
uncursed	&	未被诅咒的	\\
undead	&	不死生物	\\
under water	&	水面底下	\\
{\it Unearthed Arcana}	&	《被发掘出的奥秘》	\\
unholy water	&	邪恶之水	\\
unicorn	&	独角兽	\\
unskilled	&	不熟练的	\\
Valkyrie	&	女武神	\\
the Valley of Gehennom	&	地狱之谷	\\
vampire	&	吸血鬼	\\
vortex	&	旋涡	\\
wall	&	墙	\\
wand	&	魔杖	\\
weapon	&	武器	\\
web	&	蜘蛛网	\\
Wisdom	&	智慧值	\\
Wizard	&	巫师	\\
worm	&	蠕虫	\\
wraith	&	死灵	\\
xan	&	三虫	\\
xorn	&	索尔石怪	\\
zombie	&	僵尸	\\
zruty	&	祖鲁提	\\

\end{supertabular}

\onecolumn

\section{\it NetHack General Public License}
\label{appendix:NGPL}

以下是本文档所使用的协议。
请注意,该协议有其自己的版本许可(第一段第一句,“{\it 任何人被允许复制和分发这份协议的一字不差的副本,
但更改是不被允许的。}”),与本文档的许可并不同,请依照它的许可进行使用。

\VerbatimInput{license}
%\lstinputlisting[xleftmargin=.01\textwidth]{license}

\end{document}
